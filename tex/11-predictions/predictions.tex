\section{Testable Predictions and Observational Signatures}
  \label{sec:testable-predictions-and-observational-signatures}

  Before detailing specific observational signatures, it is important to clarify
  the epistemic status of the numerical estimates presented in this section.
  Values such as the $\sim 8$--$10\%$ correction to the Hubble constant or the
  $\sim 10^{-10}\,\mathrm{yr}^{-1}$ drift in effective observables are not proposed
  as precision predictions.
  They should be understood as order-of-magnitude consistency estimates derived
  from the geometric coupling between the $\chi$ field and the effective relaxation
  fraction $\Omega_\chi$.

  Their role is to demonstrate that the Cosmochrony framework operates within a
  phenomenologically relevant regime, capable of addressing current observational
  tensions without fine-tuning or the introduction of additional dynamical degrees
  of freedom.

  \subsection{Hubble Constant from $\chi$ Dynamics}
    \label{subsec:hubble-constant-from-$chi$-dynamics}

    In Cosmochrony, the Hubble parameter is not introduced as a free cosmological
    parameter but follows directly from the relaxation dynamics of the $\chi$ field:
    \begin{equation}
      H(t) = \frac{\dot{\chi}}{\chi}.
    \end{equation}

    Assuming a maximal relaxation rate $\dot{\chi} \simeq c$, the present-day value
    can be expressed as
    \begin{equation}
      H_0 \simeq \frac{c}{\chi(t_0)}.
    \end{equation}

    This relation implies a direct correspondence between the observed Hubble
    constant and the characteristic relaxation scale of $\chi$ at the current epoch.
    Early-universe probes (e.g.\ CMB-based inferences) and late-time distance ladder
    measurements effectively sample $\chi$ at different stages of its relaxation,
    naturally leading to systematically different inferred values of $H_0$.

  \subsection{Redshift Drift}
    \label{subsec:redshift-drift}

    The monotonic relaxation of $\chi$ implies a slow temporal evolution of
    cosmological redshifts.
    This induces a redshift drift that differs quantitatively from that predicted by
    $\Lambda$CDM, particularly at intermediate redshifts.

    A characteristic estimate for the drift rate is
    \begin{equation}
      \dot{z} \sim H_0 (1+z) - \frac{c}{\chi(t)},
    \end{equation}
    corresponding to a secular variation of order
    $\Delta z \sim 10^{-10}\,\mathrm{yr}^{-1}$ at $z \sim 1$.
    This differs from standard $\Lambda$CDM expectations at the $\sim 10\%$ level in
    this regime.

    Future high-precision spectroscopic facilities, such as extremely large
    telescopes equipped with ultra-stable spectrographs, may be capable of probing
    this effect, providing a direct observational discriminator between geometric
    relaxation and dark-energy-driven acceleration.

  \subsection{Gravitational Wave Propagation}
    \label{subsec:gravitational-wave-propagation}

    In the Cosmochrony framework, gravitational waves correspond to propagating
    modulations of the $\chi$ field.
    In regions of high excitation density, such as near compact objects, the local
    slowdown of $\chi$ relaxation is expected to induce partial absorption or
    dispersion of these modulations.

    Order-of-magnitude estimates suggest that gravitational waves passing within
    $\sim 10\,GM/c^2$ of a compact object could experience attenuation at the level of
    $\sim 10\%$.
    Such effects would manifest primarily in the late-time ringdown phase of binary
    mergers as a frequency-dependent deviation from general relativistic templates.

    While current detectors lack the sensitivity to resolve such effects, future
    space-based observatories with high signal-to-noise ratios may provide the
    necessary precision.

  \subsection{Spin and Topological Signatures}
    \label{subsec:spin-and-topological-signatures}

    If particle spin originates from topologically nontrivial configurations of the
    $\chi$ field, as proposed in this work, then spin-related phenomena may exhibit
    subtle geometric signatures beyond standard quantum mechanical descriptions.

    In particular, ultra-high-precision interference experiments sensitive to
    $4\pi$ rotational symmetry may probe deviations associated with the internal
    topology of localized $\chi$ excitations.
    Such effects are expected to be extremely small but conceptually distinctive.

  \subsection{Absence of Dark Energy Signatures}
    \label{subsec:absence-of-dark-energy-signatures}

    Because cosmic acceleration emerges in Cosmochrony without invoking a dark energy
    component, the framework predicts the absence of dynamical dark energy signatures,
    such as evolving equations of state or clustering behavior.

    Observations consistent with a purely geometric origin of acceleration would
    favor this interpretation over models requiring additional energy components.

    \paragraph{Discriminating observational signatures.}
      While the absence of primordial tensor modes alone is not a decisive
      discriminator, Cosmochrony predicts that the lack of an inflationary phase should
      manifest through correlated deviations in large-scale cosmological observables.
      These include suppressed power at low CMB multipoles, specific angular
      correlations in polarization, and the absence of an inflationary tensor imprint
      at large angular scales.
      It is the combination of these features, rather than any single parameter, that
      provides a potential observational discriminator with respect to standard
      inflationary cosmologies.

\subsection{Summary}
  \label{subsec:summary2}

  Cosmochrony yields a set of observationally testable signatures across cosmology,
  gravitation, and quantum phenomena.
  While most predictions reproduce established observations, several imply
  quantitative deviations that may be accessible to future high-precision
  measurements, providing concrete avenues for empirical scrutiny.
