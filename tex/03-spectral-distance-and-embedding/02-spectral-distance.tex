\subsection{Spectral distance}
  \label{subsec:spectral-distance}

  From spectral proximity measures, one may define an effective distance by monotonic
  transformations.
  A convenient choice is
  \begin{equation}
    d_{\mathrm{spec}}(i,j)
    \;=\;
    -\log\!\left(
             \frac{K(i,j;\alpha)}{\sqrt{K(i,i;\alpha)\,K(j,j;\alpha)}}
    \right),
  \end{equation}
  which defines a symmetric, non-negative quantity vanishing when $i=j$.
  This definition is purely spectral and does not invoke any geometric or physical
  interpretation.

  The resulting distance is generically non-injective:
  distinct relational configurations may induce identical spectral distances, and
  multiple embeddings may correspond to the same distance matrix.
  This non-uniqueness is a structural feature of spectral reconstruction rather than a
  limitation of the formalism.

  Explicit constructions of local embeddings, intrinsic dimension selection,
  and the breakdown of manifold reconstruction outside the projectable regime
  are presented in Appendix~\ref{subsec:emergent-coordinates}.

  \paragraph{Emergent spectral dimension.}
    In Eq.~(2), the exponent $d$ should not be interpreted as a fixed or postulated
    topological dimension.
    Within the present framework, $d$ is an \emph{emergent spectral quantity}
    characterizing the asymptotic scaling of the eigenvalue counting function
    $N(\lambda)$ in the low-energy regime.
    Operationally, $d$ is extracted from the slope of $\log N(\lambda)$ versus
    $\log \lambda$ and may take non-integer values at finite spectral resolution.

    In particular, the convergence of $d$ toward an integer value reflects the
    stabilization of the relational substrate into a smooth effective geometric
    regime.
    Numerical evidence presented in this work indicates that, under admissibility
    and regularity conditions, the substrate converges toward a four-dimensional
    spectral behavior, $d \to 4$, without this value being imposed \emph{a priori}.
    This convergence is observed to persist across multiple spectral decades, indicating
    that the four-dimensional behavior is not a transient finite-size effect but a stable
    property of the admissible relational regime.

    At smaller spectral scales, deviations from integer dimensionality encode
    local curvature and connectivity distortions, providing a direct bridge
    between spectral observables and effective geometric structure.

  \paragraph{Stability of the spectral dimension.}
    Numerical evaluations of the spectral counting function, performed on distinct
    relational realizations and resolutions of the substrate, indicate that the extracted
    spectral dimension $d_s$ remains stable and converges toward $d_s \simeq 4$ over a broad
    range of spectral scales.
    Deviations appear only near the ultraviolet cutoff, where the notion of effective
    geometry itself ceases to be applicable.

    This stability across scales and discretizations supports the interpretation of
    $d_s$ as an emergent property of the relational organization rather than as an imposed
    dimensional parameter.

  \paragraph{From graph distance to effective geodesics.}
    The operational distance defined via shortest weighted paths on the relational
    graph plays the role of an effective geodesic distance.
    While combinatorial distances count edge hops, the weighted distance incorporates
    local variations of connectivity through edge weights, thereby encoding the
    inhomogeneous relational structure of the substrate.

    In the continuum limit, the local density of nodes contributing to admissible
    paths controls the scaling of volumes and distances.
    This node density acts as the discrete analogue of the metric determinant
    $\sqrt{-g}$, governing how relational neighborhoods are mapped onto effective
    geometric volumes.
    Geometric notions thus arise from connectivity statistics rather than from
    a postulated metric field.
