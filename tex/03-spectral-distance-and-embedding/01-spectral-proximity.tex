\subsection{Spectral proximity}
  \label{subsec:spectral-proximity}

  Given the Laplacian spectrum $\{\lambda_n, \phi_n\}$, one may define spectral kernels
  that quantify the relational proximity between nodes or abstract elements of the
  relational system.
  A generic example is provided using heat-kernel–type constructions.
  \begin{equation}
    K(i,j;\alpha) \;=\; \sum_{n} e^{-\alpha \lambda_n}\,\phi_n(i)\,\phi_n(j),
  \end{equation}
  where $\alpha$ is the spectral scale parameter.
  Such kernels measure the degree of connectivity between elements $i$ and $j$ through
  the spectrum of $\Delta$, without reference to any metric distance.

  Spectral kernels of this type are widely used in clustering and manifold
  learning to extract geometric information from graph spectra~\cite{VonLuxburg2007}.
  They are invariant under the relabeling of nodes and do not presuppose embedding.
  They provided a natural notion of relational proximity that depends solely on the
  spectral properties of the Laplacian.
