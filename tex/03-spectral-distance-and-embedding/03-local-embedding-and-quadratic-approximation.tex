\subsection{Local embedding and quadratic approximation}
  \label{subsec:local-embedding-and-quadratic-approximation}

  When the relational system admits a sufficiently regular spectral structure, the
  spectral distance matrix may be locally approximated by a low-dimensional embedding.
  In such regimes, one may introduce local coordinates $x^\mu$ as auxiliary variables
  parametrizing the embedding space.

  To leading order, the spectral distance between nearby elements admits a quadratic
  approximation of the form
  \begin{equation}
    d_{\mathrm{spec}}(i,j)^2
    \;\approx\;
    g_{\mu\nu}(x)\,\Delta x^\mu \Delta x^\nu ,
  \end{equation}
  where $g_{\mu\nu}$ is a symmetric tensor encoding the local geometry of the embedding.
  This tensor is not postulated as a fundamental object, but arises as a local summary
  of spectral distances.

  Local quadratic approximations of distance functions and their relation
  to effective metric tensors are standard results in metric geometry~\cite{Burago2001}.

  The introduction of $g_{\mu\nu}$ should be understood as a descriptive convenience,
  valid only in regimes where a smooth spectral embedding exists.
  Outside such regimes, no metric interpretation is assumed or required.
