\section{Introduction}
  \label{sec:introduction}

  A central open problem in gravitational physics concerns the emergence of
  spacetime geometry from more primitive, non-metric structures.
  In general relativity, the metric tensor is postulated to be a fundamental dynamical
  field that encodes both the causal and metrical properties of spacetime.
  In contrast, many approaches to background-independent gravity suggest that
  spacetime geometry should arise as an effective description, reconstructed
  from relational or pre-geometric data.

  This question has motivated a wide range of frameworks, including canonical and
  covariant approaches to quantum gravity, causal set theory, loop-based
  constructions, and programs based on spectral or non-commutative geometry.
  While these approaches differ in their technical implementation, they share a
  common challenge: how to recover a smooth pseudo-Riemannian geometry from
  fundamentally non-metric degrees of freedom, without introducing a background
  manifold or metric by assumption.

  In this study, we address this reconstruction problem from a deliberate
  minimalist and operational perspective.
  Rather than postulating a spacetime metric or continuum structure from the
  outset, we consider relational systems described by connectivity data encoded in
  discrete or coarse-grained graphs.
  The central object of our analysis is the spectrum of a Laplacian operator
  defined for such relational structures.
  We investigate the conditions under which spectral information alone is sufficient
  to define the effective notions of distance, curvature, and geometry.

  Our approach is motivated by the observation that spectral data provide a natural
  bridge between discrete relational systems and continuum geometry.
  At appropriate limits, the spectrum of graph Laplacians is known to encode
  geometric information analogous to that of the differential operators on smooth
  manifolds.
  The question that we pursue here is whether this correspondence can be made explicit
  and operational, yielding an effective metric structure without assuming one at
  the fundamental level.

  We developed a framework in which admissible relational configurations were
  characterized by spectral consistency conditions.
  Effective geometric quantities were then reconstructed through spectral distances
  and embeddings, defined purely in terms of the Laplacian spectrum.
  No coordinates, metric tensors, or variational principles were assumed a priori.
  Instead, such structures emerge only in regimes in which the relational system admits
  a stable and well-defined spectral continuum limit.

  Within this setting, we demonstrate that familiar geometric notions can be recovered in a
  controlled manner.
  In particular, we demonstrate how an effective metric description arises locally,
  how the curvature can be defined operationally, and how known solutions of general
  relativity can be approximated within appropriate limits.
  As a concrete benchmark, we demonstrate the recovery of a Schwarzschild-type geometry
  from purely spectral and relational inputs.

  The scope of this study is intentionally restricted.
  We did not address the dynamics of matter fields, quantum statistics, or
  cosmological evolution.
  Our aim is not to propose a complete theory of quantum gravity but to isolate and
  analyze the geometric reconstruction problem in its simplest and most transparent
  form.

  The present manuscript follows a theoretical structure in which methodological constructions and results are developed
  jointly, as is customary in foundational and theoretical physics.
  The structure is as follows.
  Section~\ref{sec:relational-substrate-and-spectral-structure} introduces the relational graph framework and
  the associated Laplacian operator.
  Section~\ref{sec:spectral-distance-and-embedding} develops spectral notions of distance and
  embedding.
  Section~\ref{sec:continuum-limit-and-emergent-metric} analyzes the emergence of an effective continuum
  geometry and metric structure.
  Section~\ref{sec:operational-geometry} discusses operational definitions of curvature,
  and Section~\ref{sec:gr-limit-schwarzschild-type-recovery} presents the recovery of a
  Schwarzschild-type effective geometry.
  Technical results and auxiliary derivations are collected in the appendices.
