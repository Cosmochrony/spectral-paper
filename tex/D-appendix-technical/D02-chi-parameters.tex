\subsection{Estimates of $\chi$-Field Parameters}
  \label{subsec:chi-parameters}

  This section compiles order-of-magnitude estimates for the effective parameters
  governing the $\chi$-field dynamics.
  These estimates are not presented as fundamental constants,
  but as \emph{consistency ranges} inferred from observational constraints
  and from the requirement that the emergent description reproduces known
  gravitational and cosmological scales.

  The parameters considered here include:
  \begin{itemize}
    \item the \textbf{connectivity scale} $K_0$ entering the collective coupling $K_{ij}$,
    \item the \textbf{characteristic $\chi$ scale} $\chi_c$,
    \item effective parameters $(\lambda,\eta)$ controlling solitonic stabilization,
    \item the maximal relaxation speed $c$.
  \end{itemize}

  \subsubsection{Connectivity Scale $K_0$ and Characteristic Scale $\chi_c$}

    In the continuum limit, the effective gravitational coupling emerges from the
    collective stiffness of the $\chi$ network.
    Dimensional consistency and matching to the Newtonian limit yield the relation
    \begin{equation}
      G = \frac{c^4}{16 \pi K_0 \chi_c^2},
      \label{eq:G-emergent}
    \end{equation}
    which links the emergent gravitational constant $G$ to two independent scales:
    \begin{itemize}
      \item $K_0$, with dimensions $[\mathrm{length}]^{-2}$, characterizing the maximal
      local connectivity of the network,
      \item $\chi_c$, the characteristic magnitude of $\chi$ over which macroscopic
      geometric effects become significant.
    \end{itemize}

    Equation~\eqref{eq:G-emergent} admits two physically distinct regimes:

    \paragraph{Planck-scale normalization.}
      If $\chi_c$ is associated with the Planck length
      $\ell_P \simeq 1.6 \times 10^{-35}\,\mathrm{m}$,
      one finds
      \begin{equation}
        K_0 \sim 10^{93}\,\mathrm{m}^{-2}.
      \end{equation}
      In this regime, the $\chi$ network is extremely stiff,
      and gravitational effects are interpreted as arising from microscopic connectivity
      close to quantum-gravitational scales.

    \paragraph{Cosmological-scale normalization.}
      If instead $\chi_c$ is identified with the present Hubble scale
      $c/H_0 \simeq 1.4 \times 10^{26}\,\mathrm{m}$,
      the inferred connectivity is
      \begin{equation}
        K_0 \sim 10^{-52}\,\mathrm{m}^{-2}.
      \end{equation}
      This corresponds to a much softer network,
      where large-scale cosmological relaxation dominates the effective geometry.

      Both regimes are internally consistent at the level of dimensional analysis;
      discriminating between them requires additional observational input.

  \subsubsection{Effective Solitonic Parameters $(\lambda,\eta)$}

    Localized particle-like excitations are modeled as stabilized configurations
    of the $\chi$ field.
    In generic solitonic models, the characteristic mass scale satisfies
    \begin{equation}
      m_{\text{soliton}} \propto \sqrt{\lambda}\,\eta^3,
    \end{equation}
    where $\lambda$ controls nonlinear self-interaction strength
    and $\eta$ sets the characteristic field amplitude.

    Matching the electron mass scale provides an indicative constraint:
    \begin{equation}
      \lambda \sim 10^{-116}\,\mathrm{m}^{-2}
    \end{equation}
    for $\eta \sim \mathcal{O}(1)$ in naturalized units.
    This extremely small value should not be interpreted as a fundamental constant,
    but as evidence that $\lambda$ is an emergent or environment-dependent parameter,
    analogous to effective couplings in condensed-matter systems.

    Mass hierarchies between particle species (e.g.\ $m_p/m_e \simeq 1836$)
    would then arise from differences in topological class,
    effective $\eta$, or collective stabilization mechanisms,
    rather than from a single universal coupling.

  \subsubsection{Relaxation Speed and Cosmological Constraints}

    The maximal relaxation speed $c$ is identified with the invariant speed
    of relativistic kinematics.
    At the cosmological level, the background evolution implies
    \begin{equation}
      H(t) \simeq \frac{\dot{\chi}}{\chi},
    \end{equation}
    so that at the present epoch
    \begin{equation}
      \chi(t_0) \simeq \frac{c}{H_0}
      \sim 4 \times 10^{26}\,\mathrm{m}.
    \end{equation}

    This identification reproduces the observed age of the Universe
    $t_0 \sim \chi(t_0)/c \simeq 13.8\,\mathrm{Gyr}$
    without introducing additional cosmological parameters.

  \subsubsection{Observational Constraints}

    Current observations impose indirect constraints on the allowed parameter space:
    \begin{itemize}
      \item \textbf{CMB anisotropies} constrain large-scale $\chi$ fluctuations,
      disfavouring values of $\chi_c$ that would over-amplify low-$\ell$ modes.
      \item \textbf{Hubble tension} can be interpreted as probing different effective
      $\chi$ regimes at low and high redshift.
      \item \textbf{Gravitational-wave propagation} constrains variations of $K_0$
      in strong-field environments to remain subdominant.
    \end{itemize}

  \subsubsection{Summary and Status}

    Table~\ref{tab:chi-parameters} summarizes the indicative ranges discussed above.
    These values should be understood as \emph{consistency windows}, not predictions.

    \begin{table}[h]
      \centering
      \caption{Indicative ranges for effective $\chi$-field parameters}
      \label{tab:chi-parameters}
      \begin{tabular}{|c|c|c|}
        \hline
        Parameter & Planck-scale regime & Cosmological-scale regime \\ \hline
        $K_0$ & $\sim 10^{93}\,\mathrm{m}^{-2}$ & $\sim 10^{-52}\,\mathrm{m}^{-2}$ \\ \hline
        $\chi_c$ & $\sim 10^{-35}\,\mathrm{m}$ & $\sim 10^{26}\,\mathrm{m}$ \\ \hline
        $\lambda$ & $\ll 1$ (effective) & $\ll 1$ (effective) \\ \hline
        $\eta$ & $\mathcal{O}(1)$ & $\gg 1$ \\ \hline
      \end{tabular}
    \end{table}

    A first-principles derivation of these parameters from the microscopic
    $\chi$ dynamics remains an open problem and is identified as a key target
    for future analytical and numerical work.
