\subsection{Estimates of $\chi$-Field Parameters}
  \label{subsec:chi-parameters}

  The quantities introduced in this section, including effective coupling matrices and spectral
  modes, should be understood as properties of a projected relaxation operator acting on a finite
  function space.

  They characterize how localized $\chi$ configurations respond to perturbations within a given
  resolution scale, rather than representing fundamental degrees of freedom of the theory.
  The parameters considered here characterize the effective response of the $\chi$
  relaxation dynamics in regimes where a coarse-grained geometric description applies.
  They do not correspond to fundamental constants of the theory, but to emergent or
  environment-dependent scales encoding how localized structures constrain relaxation.

  The relevant parameters include:
  \begin{itemize}
    \item the \textbf{effective coupling scale} $K_0$ entering the projected kernel $K_{ij}$,
    \item the \textbf{characteristic $\chi$ scale} $\chi_c$ at which macroscopic geometric
    effects become significant,
    \item effective solitonic parameters $(\lambda,\eta)$ controlling stabilization
    mechanisms,
    \item the maximal relaxation speed $c$.
  \end{itemize}

  \subsubsection{Effective Coupling Scale $K_0$ and Characteristic Scale $\chi_c$}

    In weak-field regimes admitting an effective geometric description, the gravitational
    coupling emerges from the collective stiffness of the $\chi$ relaxation dynamics.
    Dimensional consistency and matching to the Newtonian limit yield the relation
    \begin{equation}
      G = \frac{c^4}{16 \pi K_0 \chi_c^2},
      \label{eq:G-emergent}
    \end{equation}
    which links the effective gravitational constant $G$ to two emergent scales.

    The parameter $K_0$, with dimensions $[\mathrm{length}]^{-2}$, characterizes the maximal
    strength of the collective relaxation response in a homogeneous background.
    It should be understood as a property of the projected relaxation operator introduced
    for numerical and phenomenological purposes, not as a microscopic connectivity.

    The scale $\chi_c$ sets the characteristic magnitude of $\chi$ over which structural
    variations significantly modulate relaxation and thereby induce macroscopic geometric
    effects.
    It marks the breakdown scale of the homogeneous relaxation regime, beyond which
    localized configurations noticeably slow the relaxation flow.

    Equation~\eqref{eq:G-emergent} admits two physically distinct normalization regimes:

    \paragraph{Planck-scale normalization.}
      If $\chi_c$ is associated with the Planck length
      $\ell_P \simeq 1.6 \times 10^{-35}\,\mathrm{m}$,
      one finds
      \begin{equation}
        K_0 \sim 10^{93}\,\mathrm{m}^{-2}.
      \end{equation}
      In this regime, the effective relaxation dynamics is extremely stiff, and gravitational
      phenomena are interpreted as arising from structural constraints operating near the
      threshold where classical spacetime descriptions cease to apply.

    \paragraph{Cosmological-scale normalization.}
      If instead $\chi_c$ is identified with the present Hubble scale
      $c/H_0 \simeq 1.4 \times 10^{26}\,\mathrm{m}$,
      the inferred effective coupling scale is
      \begin{equation}
        K_0 \sim 10^{-52}\,\mathrm{m}^{-2}.
      \end{equation}
      This regime corresponds to a much softer collective response, dominated by large-scale
      cosmological relaxation.
      Both normalizations are internally consistent at the level of dimensional analysis;
      discriminating between them requires additional observational input.

  \subsubsection{Effective Solitonic Parameters $(\lambda,\eta)$}

    Localized particle-like excitations are modeled as stabilized configurations of the
    $\chi$ field.
    Their persistence against relaxation is controlled by effective parameters governing
    nonlinear self-organization and structural amplitude.

    In generic solitonic models, the characteristic energy scale satisfies
    \begin{equation}
      m_{\text{soliton}} \propto \sqrt{\lambda}\,\eta^3,
    \end{equation}
    where $\lambda$ controls the strength of nonlinear stabilization and $\eta$ sets the
    typical amplitude of the configuration.
    Within Cosmochrony, these parameters are not universal constants but encode how a given
    topological configuration resists the global relaxation flow.

    Matching the electron mass scale provides an indicative constraint,
    \begin{equation}
      \lambda \sim 10^{-116}\,\mathrm{m}^{-2}
    \end{equation}
    for $\eta \sim \mathcal{O}(1)$ in naturalized units.
    Such an extremely small value should not be interpreted as fine-tuning, but as evidence
    that $\lambda$ is an emergent parameter sensitive to the internal structure and
    environment of the excitation, in close analogy with effective couplings in
    condensed-matter systems.

    Mass hierarchies between particle species (e.g.\ $m_p/m_e \simeq 1836$) would then arise
    from differences in topological class, effective amplitudes, or collective stabilization
    mechanisms, rather than from a single universal coupling.

  \subsubsection{Relaxation Speed and Cosmological Constraints}

    The maximal relaxation speed $c$ is identified with the invariant speed of relativistic
    kinematics.
    At the cosmological level, the homogeneous relaxation dynamics imply
    \begin{equation}
      H(t) \simeq \frac{\dot{\chi}}{\chi},
    \end{equation}
    so that at the present epoch
    \begin{equation}
      \chi(t_0) \simeq \frac{c}{H_0}
      \sim 4 \times 10^{26}\,\mathrm{m}.
    \end{equation}

    This identification reproduces the observed age of the Universe,
    $t_0 \sim \chi(t_0)/c \simeq 13.8\,\mathrm{Gyr}$,
    without introducing additional cosmological parameters or modifying late-time
    dynamics.

  \subsubsection{Observational Constraints}

    Current observations impose indirect constraints on the allowed effective parameter
    space:
    \begin{itemize}
      \item \textbf{CMB anisotropies} constrain large-scale $\chi$ fluctuations, disfavouring
      values of $\chi_c$ that would excessively amplify low-$\ell$ modes.
      \item \textbf{The Hubble tension} may be interpreted as probing different effective
      $\chi$ relaxation regimes at low and high redshift.
      \item \textbf{Gravitational-wave propagation} constrains variations of the effective
      coupling scale $K_0$ in strong-field environments to remain subdominant.
    \end{itemize}

  \subsubsection{Summary and Status}

    Table~\ref{tab:chi-parameters} summarizes the indicative ranges discussed above.
    These values define consistency windows rather than predictions.

    \begin{table}[h]
      \centering
      \caption{Indicative ranges for effective $\chi$-field parameters}
      \label{tab:chi-parameters}
      \begin{tabular}{|c|c|c|}
        \hline
        Parameter & Planck-scale regime & Cosmological-scale regime \\ \hline
        $K_0$ & $\sim 10^{93}\,\mathrm{m}^{-2}$ & $\sim 10^{-52}\,\mathrm{m}^{-2}$ \\ \hline
        $\chi_c$ & $\sim 10^{-35}\,\mathrm{m}$ & $\sim 10^{26}\,\mathrm{m}$ \\ \hline
        $\lambda$ & $\ll 1$ (effective) & $\ll 1$ (effective) \\ \hline
        $\eta$ & $\mathcal{O}(1)$ & $\gg 1$ \\ \hline
      \end{tabular}
    \end{table}

    A first-principles derivation of these effective parameters from the underlying
    $\chi$ relaxation dynamics remains an open problem and is identified as a central
    target for future analytical and numerical work.
