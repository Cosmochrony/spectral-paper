\subsection{Simulation Algorithms for $\chi$-Field Dynamics}
  \label{subsec:simulation-algorithms}

  This appendix presents numerical methods used to explore the dynamical behavior of the
  $\chi$ field under the relaxation principles introduced in the main text.
  Its purpose is neither to define the fundamental ontology of Cosmochrony nor to privilege
  a specific microscopic substrate, but to provide a controlled computational framework in
  which the consistency, stability, and collective behavior of $\chi$ configurations can be
  investigated.

  The simulations described here serve four complementary goals:
  \begin{enumerate}
    \item to verify the internal consistency of the bounded relaxation dynamics,
    \item to test the spontaneous formation and persistence of localized excitations,
    \item to study the response of the field to perturbations and external constraints,
    \item to extract preliminary spectral signatures associated with stable configurations.
  \end{enumerate}

  All numerical constructions should therefore be understood as \emph{effective
representations}.
  They do not assume a fundamental spacetime geometry, nor do they fix a unique microscopic
  realization of the theory.
  Their role is analogous to lattice simulations in QCD: exploratory, non-fundamental, yet
  indispensable for probing non-linear and collective regimes.

  \paragraph{Discrete representation and numerical substrate.}
    The $\chi$ field is represented on a discrete set of sites $\{i\}$, each carrying a scalar
    value $\chi_i(\lambda)$, where $\lambda$ is the monotonic relaxation parameter introduced in
    Section~\ref{subsec:parameter-independent-relaxation}.
    Connectivity between sites is encoded by a local coupling matrix $K_{ij}$.

    This discrete structure should be regarded strictly as a numerical scaffold.
    It does not represent a physical lattice, causal set, or spacetime discretization.
    Its sole purpose is to approximate local correlations and relaxation flows without
    presupposing a background geometry.
    Different graph topologies (regular grids, random graphs, locally weighted networks) lead
    to qualitatively similar behavior, indicating that the observed phenomena are robust with
    respect to discretization choices.

    The simulations are therefore compatible with the relational formulation developed in
    Appendix~\ref{app:relational_formulation}, while remaining technically independent of it.

  \paragraph{Relaxation update rule.}
    The discrete evolution of the field is governed by a bounded relaxation rule inspired by
    the minimal kinematic constraint
    (Section~\ref{subsec:minimal-kinematic-constraint}):
    \begin{equation}
      \label{eq:discrete-dynamics}
      \frac{d\chi_i}{d\lambda}
      =
      c \sqrt{1 - \frac{1}{c^2}
      \sum_{j \sim i} K_{ij} (\chi_i - \chi_j)^2 } .
    \end{equation}

    This update rule enforces strict monotonicity of $\chi$, a universal upper bound on the
    local relaxation rate, and suppression of gradient-driven instabilities.
    Time stepping is implemented using explicit adaptive schemes with stability control.
    Alternative discretizations respecting the same bound lead to equivalent qualitative
    behavior, confirming that the results do not depend on a fine-tuned algorithmic choice.

  \paragraph{Formation of localized configurations.}
    Starting from generic initial conditions, the simulations consistently exhibit the
    emergence of localized regions where spatial variations of $\chi$ remain persistently
    large.
    These regions resist the global relaxation flow and remain stable over many relaxation
    times.

    Such configurations are interpreted as numerical counterparts of the solitonic
    excitations discussed in
    Section~\ref{sec:particles-as-localized-excitations-of-the-chi-field}.
    They arise dynamically without being imposed by hand and do not require fine-tuned
    initial conditions.
    Perturbative tests show that small disturbances around these configurations decay rather
    than grow, confirming their dynamical stability.

  \paragraph{Spectral analysis and mode separation.}
    To probe the internal structure of stable configurations, the relaxation operator is
    linearized around a stationary $\chi$ background.
    The resulting eigenvalue problem defines a discrete spectrum of modes characterizing the
    response of the configuration to small perturbations.

    A systematic spectral scan reveals a clear separation between a small number of
    low-lying modes associated with collective, coherent deformations of the configuration,
    and a dense set of higher modes that are rapidly damped by the relaxation dynamics.
    This mode separation is a robust numerical feature and does not depend sensitively on the
    discretization or boundary conditions.

    These eigenmodes quantify intrinsic relaxation scales associated with the configuration.
    At this stage, they are not identified with observed particle masses.
    Rather, they provide a structural fingerprint of each stable excitation, capturing how
    strongly it resists deformation.
    The possible connection between these spectral hierarchies and physical mass spectra is
    discussed conceptually in Appendix~\ref{subsec:spectral_mass}, without invoking
    numerical matching.

  \paragraph{Interpretation and scope.}
    The existence of discrete spectral hierarchies is a robust numerical result.
    Their role in the present work is structural rather than predictive: they demonstrate
    that the relaxation dynamics of $\chi$ naturally supports stable localized excitations,
    bounded and causal evolution, and internally organized mode spectra.

  \paragraph{Limitations.}
    The simulations presented here do not include quantum fluctuations, backreaction beyond
    the effective level, or fully relativistic covariance.
    They are not intended to provide numerical predictions for particle physics or precision
    cosmology.

  \paragraph{Conclusion.}
    This appendix shows that the relaxation dynamics of the $\chi$ field can be implemented
    numerically in a stable and controlled manner, without invoking a background geometry or
    additional degrees of freedom.
    The emergence of localized excitations and discrete spectral mode separation provides
    strong support for the conceptual foundations of Cosmochrony and establishes a solid
    basis for future quantitative developments.
