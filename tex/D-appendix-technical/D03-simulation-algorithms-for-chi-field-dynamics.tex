\subsection{Simulation Algorithms for $\chi$-Field Dynamics}
  \label{subsec:simulation-algorithms}

  The numerical simulations presented here do not assume an underlying network or lattice structure
  for the $\chi$ field.
  Instead, they rely on a finite basis representation chosen for numerical stability and diagnostic
  clarity, analogous to spectral, finite-element, or wavelet-based methods commonly used in
  continuum field theories.

  The apparent graph-like structure appearing in the implementation reflects this choice of basis
  and sampling, not a physical discretization of the $\chi$ substrate.

  The numerical simulations presented in this appendix pursue four complementary goals:
  \begin{enumerate}
    \item to verify the internal consistency of the bounded relaxation dynamics,
    \item to test the spontaneous formation and persistence of localized configurations,
    \item to study the response of the $\chi$ field to perturbations and imposed constraints,
    \item to extract structural spectral signatures associated with stable configurations.
  \end{enumerate}

  These simulations implement finite-dimensional representations of the fundamentally
  continuous $\chi$ relaxation dynamics.
  They do not assume a fundamental spacetime geometry, nor do they select a unique
  microscopic realization of the theory.
  Their role is analogous to numerical approximations commonly used in continuum field
  theories to probe non-linear and collective regimes that are analytically inaccessible.

  \paragraph{Numerical representation and computational substrate.}
    For numerical purposes, the $\chi$ field is represented on a finite set of degrees of
    freedom $\{\chi_i(\lambda)\}$, where the index $i$ labels elements of a chosen numerical
    basis and $\lambda$ denotes the monotonic relaxation parameter introduced in
    Section~\ref{subsec:parameter-independent-relaxation}.
    The interactions between these degrees of freedom are encoded through a coupling
    matrix $K_{ij}$, which represents a finite projection of the effective relaxation kernel.

    This construction should not be interpreted as a physical lattice, causal structure,
    or discretized spacetime.
    Different choices of numerical bases or sampling strategies (regular grids, irregular
    samplings, weighted connectivity graphs) lead to qualitatively similar behavior,
    indicating that the observed phenomena reflect intrinsic properties of the relaxation
    dynamics rather than artifacts of a particular representation.

  \paragraph{Relaxation update rule.}
    The numerical evolution of the field follows a bounded relaxation rule inspired by the
    minimal kinematic constraint discussed in
    Section~\ref{subsec:minimal-kinematic-constraint}.
    In the chosen representation, the evolution equation takes the form
    \begin{equation}
      \label{eq:discrete-dynamics}
      \frac{d\chi_i}{d\lambda}
      =
      c \sqrt{1 - \frac{1}{c^2}
      \sum_{j} K_{ij} (\chi_i - \chi_j)^2 } .
    \end{equation}

    This update rule enforces strict monotonicity of $\chi$, a universal upper bound on the
    local relaxation rate, and the suppression of gradient-driven instabilities.
    Time stepping is implemented using adaptive schemes with stability control.
    Alternative numerical implementations respecting the same bound produce equivalent
    qualitative behavior, confirming that the results are not sensitive to algorithmic
    details.

  \paragraph{Formation of localized configurations.}
    Starting from generic initial conditions, the simulations robustly exhibit the
    spontaneous emergence of localized configurations in which structural variations of
    $\chi$ remain persistently large.
    These configurations locally resist the global relaxation flow and remain stable over
    many relaxation intervals.

    Such configurations are interpreted as numerical counterparts of the solitonic
    excitations discussed in
    Section~\ref{sec:particles-as-localized-excitations-of-the-chi-field}.
    They arise dynamically without being imposed by hand and do not require fine-tuned
    initial conditions.
    Perturbative tests indicate that small disturbances around these configurations decay
    rather than grow, confirming their dynamical stability within the relaxation framework.

  \paragraph{Spectral analysis and response modes.}
    To probe the internal structure of stable configurations, the effective relaxation
    operator is linearized around a stationary background configuration.
    The resulting eigenvalue problem defines a discrete set of response modes describing
    how the configuration reacts to small perturbations within the chosen numerical
    representation.

    A systematic spectral analysis reveals a clear separation between a small number of
    low-lying modes associated with coherent, collective deformations of the configuration,
    and a dense set of higher modes that are rapidly damped by the relaxation dynamics.
    This separation is a robust numerical feature and does not depend sensitively on the
    choice of basis, resolution, or boundary conditions.

    These response modes quantify intrinsic relaxation scales associated with a given
    configuration.
    At this stage, they are not identified with observed particle masses.
    Rather, they provide a structural fingerprint characterizing the degree of internal
    organization and resistance to deformation of each stable excitation.
    Possible connections between such spectral hierarchies and physical mass spectra are
    discussed conceptually in
    Appendix~\ref{subsec:spectral_mass}, without invoking numerical matching.

  \paragraph{Interpretation and scope.}
    The appearance of discrete spectral hierarchies in the simulations is a robust and
    reproducible numerical result.
    Within the present work, their role is structural rather than predictive: they
    demonstrate that the bounded relaxation dynamics of $\chi$ naturally supports stable
    localized configurations, causal and monotonic evolution, and internally organized
    response spectra.

  \paragraph{Limitations.}
    The simulations presented here do not include quantum fluctuations, higher-order
    backreaction effects, or fully relativistic covariance.
    They are not intended to provide quantitative predictions for particle physics or
    precision cosmology.

  \paragraph{Conclusion.}
    This appendix demonstrates that the relaxation dynamics of the $\chi$ field can be
    implemented numerically in a stable and controlled manner using finite-dimensional
    representations, without invoking a background geometry or additional fundamental
    degrees of freedom.
    The spontaneous emergence of localized configurations and the associated separation of
    response modes provide strong numerical support for the conceptual foundations of
    Cosmochrony and establish a solid basis for future quantitative investigations.
