\subsection{Collective Gravitational Coupling and Operational Geometry}
  \label{subsec:collective-coupling}

  The $\chi$ field is fundamentally continuous, and its dynamics is defined through non-linear,
  non-perturbative relaxation constraints that do not admit a closed-form spectral decomposition.
  As a consequence, any explicit investigation of stability, mode structure, or response to localized
  excitations necessarily requires a finite-dimensional projection of the underlying functional
  operator.

  In this appendix, we introduce such a projection as a numerical representation of the
  \emph{continuous} relaxation operator governing $\chi$.
  This construction does not reflect any discretization of the underlying substrate, but provides
  a computationally tractable surrogate for an operator whose exact functional spectrum is
  analytically inaccessible.

  \paragraph{Collective coupling and effective description.}
    Localized excitations of the $\chi$ field resist its global relaxation and thereby
    reduce the local relaxation rate.
    When many such excitations are present, their influence combines collectively,
    leading to a macroscopic modulation of the relaxation dynamics.

    At the effective level, this collective influence can be summarized by a coupling
    kernel $K_{ij}$, understood as a finite representation of the response of the
    $\chi$ relaxation flow to structural variations.
    The indices $i$ and $j$ label elements of a chosen numerical or functional basis
    used to represent configurations of $\chi$; they do not correspond to fundamental
    spatial locations.
    The kernel depends only on differences of $\chi$ between configurations and
    characterizes the stiffness of the relaxation dynamics, without presupposing any
    background notion of distance, adjacency, or metric structure.

  \paragraph{Effective gravitational potential and weak-field regime.}
    In regimes where variations of $\chi$ are weak and smoothly distributed, the
    collective relaxation dynamics admit a coarse-grained description in which
    spatial organization becomes meaningful.
    In this regime, variations of the local relaxation rate may be parametrized
    by an effective gravitational potential $\Phi$, defined operationally through
    the relative slowdown of $\chi$:
    \begin{equation}
      \frac{\partial_t \chi}{c} \simeq 1 + \frac{\Phi}{c^2}.
    \end{equation}

    This definition introduces $\Phi$ not as a fundamental field, but as a convenient
    summary of how localized excitations collectively constrain the relaxation flow.
    Under this identification, and in the weak-structure limit, the effective dynamics
    reduce to a Poisson-like relation,
    \begin{equation}
      \nabla^2 \Phi = 4\pi G \rho,
    \end{equation}
    where $\rho$ denotes the effective density of localized, relaxation-resistant
    configurations.
    The constant $G$ appears here as an emergent coupling parameter relating the
    collective response of the $\chi$ field to the density of such excitations.
    As in other effective descriptions of gravity, its numerical value is fixed
    empirically rather than derived from first principles at this level.

  \paragraph{Operational emergence of geometry.}
    Because Cosmochrony does not assume a fundamental spacetime metric, spatial geometry
    is defined operationally through the propagation and attenuation of variations
    in the $\chi$ field.
    Two configurations are considered close if perturbations of $\chi$ propagate
    efficiently between them, and distant otherwise.

    In the weak-gradient regime, this operational notion induces an effective spatial
    geometry that coincides with the standard Newtonian and post-Newtonian descriptions.
    Spacetime curvature therefore arises as a macroscopic descriptor of how localized
    excitations collectively modulate the relaxation and propagation properties of
    $\chi$, rather than as a primitive geometric attribute.

  \paragraph{Scope and limitations.}
    The construction presented in this subsection is restricted to quasi-static,
    weak-field regimes in which a smooth geometric description provides a faithful
    summary of the underlying relaxation dynamics.
    It does not address strong-field configurations, fully relativistic corrections,
    or quantum fluctuations of the $\chi$ field.

    Its purpose is to establish that classical gravitational behavior can be recovered
    consistently as an effective manifestation of the collective relaxation dynamics
    of $\chi$, without introducing a fundamental metric structure or an independent
    gravitational interaction.
