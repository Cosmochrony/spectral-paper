\subsection{Collective Gravitational Coupling and Operational Geometry}
  \label{subsec:collective-coupling}

  This subsection develops an effective and operational description of gravitational
  phenomena within the Cosmochrony framework.
  Its purpose is to demonstrate explicitly how standard Newtonian gravity and its
  geometric interpretation emerge from the collective dynamics of the $\chi$ field
  in regimes where a quasi-static and weak-gradient approximation is valid.

  The constructions introduced here are not fundamental.
  They constitute a technical representation of the underlying continuous dynamics,
  designed to connect the relational formulation of Cosmochrony to familiar
  gravitational phenomenology.
  The fully relational and metric-free foundation of the theory is developed
  independently in Appendix~\ref{app:relational_formulation}.

  \paragraph{Collective coupling and effective description.}
    Localized excitations of the $\chi$ field resist its global relaxation and thereby
    slow the local evolution rate $\partial_t \chi$.
    When many such excitations are present, their influence superposes collectively,
    leading to a macroscopic modification of the relaxation dynamics.

    In an effective description, this collective influence may be encoded in a coupling
    kernel $K_{ij}$, which characterizes how variations of $\chi$ between neighboring
    regions affect the local relaxation rate.
    The kernel depends only on differences of $\chi$ and captures the local stiffness
    of the field, without presupposing any background notion of distance or metric.

  \paragraph{Continuum limit and emergence of the Newtonian potential.}
    In regimes where variations of $\chi$ are small and slowly varying, the collective
    dynamics may be approximated by a continuum description.
    In this limit, spatial variations of the relaxation rate can be expressed in terms
    of an effective gravitational potential $\Phi$, defined operationally through the
    relative slowdown of $\chi$:
    \begin{equation}
      \frac{\partial_t \chi}{c} \simeq 1 + \frac{\Phi}{c^2}.
    \end{equation}

    Under this identification, the collective relaxation dynamics reduce to a Poisson-like
    equation,
    \begin{equation}
      \nabla^2 \Phi = 4\pi G \rho,
    \end{equation}
    where $\rho$ denotes the density of localized excitations.
    The gravitational constant $G$ emerges here as an effective parameter relating the
    strength of the collective coupling to the density of excitations.
    Its numerical value is not predicted from first principles in the present framework,
    but must be fixed empirically, as in other effective theories of gravity.

  \paragraph{Operational notion of geometry.}
    Because Cosmochrony does not assume a fundamental spacetime metric, spatial geometry
    is defined operationally through the propagation of $\chi$ variations.
    Two regions are considered close if perturbations of $\chi$ propagate efficiently
    between them, and distant otherwise.

    In the weak-gradient regime, this operational notion induces an effective spatial
    geometry that coincides with the standard Newtonian and post-Newtonian descriptions.
    Spacetime curvature therefore appears as a macroscopic summary of how localized
    excitations collectively modulate the relaxation and propagation of the $\chi$ field,
    rather than as a primitive geometric property.

  \paragraph{Scope and limitations.}
    The construction presented in this subsection is restricted to quasi-static,
    weak-field regimes where an effective spatial description is meaningful.
    It does not address strong-field dynamics, relativistic corrections beyond leading
    order, or quantum fluctuations of the $\chi$ field.

    Its role is to establish that classical gravitational behavior can be recovered
    consistently from the collective dynamics of $\chi$ without introducing a
    fundamental metric or an independent gravitational interaction.
