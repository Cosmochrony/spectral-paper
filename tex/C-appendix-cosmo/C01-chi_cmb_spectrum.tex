\subsection{Low-\texorpdfstring{$\ell$}{ℓ} CMB Power Suppression from Global \texorpdfstring{$\chi$}{χ} Relaxation}
  \label{app:lowell_attenuation}

  One of the most persistent anomalies of the cosmic microwave background (CMB) concerns the
  suppression of temperature anisotropy power at the largest angular scales
  ($\ell \lesssim 30$), particularly in the quadrupole and octupole moments.
  Within the standard $\Lambda$CDM framework, these deviations are generally
  attributed to cosmic variance, and no deterministic physical mechanism is
  associated with their occurrence.

  In the Cosmochrony framework, however, the large-scale modes of the CMB probe
  the global configuration space of the $\chi$ field rather than local stochastic
  perturbations alone.
  Because $\chi$ undergoes a monotonic relaxation constrained by a maximal
  relaxation speed and by global connectivity, the longest-wavelength modes
  correspond to collective configurations that are not freely adjustable.

  \paragraph{Structural attenuation of global modes.}
    At very low multipoles, the associated angular modes span regions comparable
    to the full causal domain of the $\chi$ field.
    As a consequence, these modes are subject to global relaxation constraints:
    their amplitude is attenuated relative to the scale-invariant expectation,
    not as a result of random fluctuation, but due to the finite relaxation budget
    available for the largest coherent configurations of $\chi$.

    This effect is deterministic in origin but statistical in manifestation.
    Cosmochrony does not predict exact multipole amplitudes; instead, it predicts
    a systematic suppression envelope affecting the lowest $\ell$ modes, whose
    precise realization depends on the detailed configuration of $\chi$ at
    last scattering.

  \paragraph{Illustrative comparison with observations.}
    Figure~\ref{fig:cmb_lowell_unsmoothed} displays the observed CMB temperature
    power spectrum at low multipoles, without aggressive smoothing, together with
    a schematic attenuation envelope representative of the Cosmochrony mechanism.
    This figure is intended to illustrate the qualitative structural deviation
    from scale invariance implied by global $\chi$ relaxation constraints.
    It does \emph{not} constitute an independent multipole-by-multipole prediction.

    \begin{figure}[htbp]
      \centering
      \includegraphics[width=0.85\textwidth]{C-appendix-cosmo/cmb_lowell_unsmoothed}
      \caption{Observed CMB temperature power spectrum at low multipoles
        ($\ell \lesssim 30$), shown without heavy smoothing.
        The shaded region illustrates a qualitative attenuation envelope expected
        from global $\chi$ relaxation constraints in Cosmochrony.
        Unlike $\Lambda$CDM, where low-$\ell$ suppression is treated as a statistical
        accident, Cosmochrony interprets it as a structural consequence of the finite
        relaxation capacity of the largest-scale $\chi$ configurations.
        The envelope is illustrative and does not represent a parameter-fitted
        prediction.}
      \label{fig:cmb_lowell_unsmoothed}
    \end{figure}

  \paragraph{Conceptual distinction from $\Lambda$CDM.}
    In $\Lambda$CDM, deviations at low $\ell$ are explained \emph{a posteriori}
    as realizations of cosmic variance around an ensemble mean.
    In Cosmochrony, by contrast, the ensemble itself is constrained:
    the global relaxation dynamics of $\chi$ limit the admissible configuration
    space for the longest-wavelength modes.
    This introduces a qualitative, physically grounded distinction between
    large-scale and small-scale fluctuations.

  \paragraph{Scope and limitations.}
    The present analysis does not replace detailed Boltzmann calculations nor
    does it attempt to reproduce the full angular power spectrum.
    Its purpose is to identify a robust qualitative signature of Cosmochrony:
    a deterministic suppression tendency affecting the lowest multipoles,
    arising from the global relaxation structure of the fundamental field.
    Quantitative refinement of this effect is deferred to future numerical
    studies of $\chi$ dynamics.
