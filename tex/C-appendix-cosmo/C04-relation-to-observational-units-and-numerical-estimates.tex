\subsection{Relation to Observational Units and Numerical Estimates}
  \label{subsec:observational-estimates}

  This subsection provides order-of-magnitude estimates that connect the $\chi$ framework
  to observed cosmological quantities.
  The purpose is not to perform parameter fitting or derive precise numerical predictions,
  but to assess the internal consistency of Cosmochrony and its compatibility with
  empirical scales.

  \subsubsection{Normalization of the $\chi$ Field}
    \label{subsec:normalization-of-the-chi-field}

    To relate $\chi$ to observable quantities, a normalization must be specified.
    We identify the present-day value $\chi(t_0)$ with the characteristic cosmological
    length scale governing large-scale expansion.
    Operationally, $\chi(t_0)$ may be interpreted as the cumulative geometric scale
    associated with the global relaxation of $\chi$ up to the present epoch.

    This identification does not assume a unique physical origin for $\chi(t_0)$,
    but provides a minimal normalization consistent with the interpretation
    $a(t) \propto \chi(t)$.

  \subsubsection{Emergent Gravitational Coupling}
    \label{subsec:K0-chic-constraints}

    In the effective description, the gravitational constant $G$ emerges from the
    constitutive relation governing the coupling between neighboring $\chi$
    configurations.
    While the microscopic parameters $K_0$ and $\chi_c$ are not fixed individually,
    their product is constrained by matching the observed value of $G$:
    \begin{equation}
      K_0 \chi_c^2 \sim \frac{c^4}{16 \pi G}.
    \end{equation}

    This relation fixes the overall stiffness scale of the effective $\chi$ network,
    while leaving open whether the characteristic scale $\chi_c$ is associated with
    microscopic (e.g.\ Planckian) or macroscopic (cosmological) physics.
    The present framework does not require committing to a specific choice at this stage.

  \subsubsection{Hubble Constant}
    \label{subsec:hubble-constant}

    At the homogeneous level, the Hubble parameter is given by
    \begin{equation}
      H(t) = \frac{\dot{\chi}}{\chi}.
    \end{equation}
    Assuming that the present universe is close to the maximal relaxation regime,
    $\dot{\chi}(t_0) \simeq c$, the present Hubble constant follows as
    \begin{equation}
      H_0 \simeq \frac{c}{\chi(t_0)}.
    \end{equation}

    Using the observed value $H_0 \approx 70~\mathrm{km\,s^{-1}\,Mpc^{-1}}$ yields
    \begin{equation}
      \chi(t_0) \sim 4 \times 10^{26}~\mathrm{m},
    \end{equation}
    consistent with the observed Hubble radius.
    This correspondence arises without introducing additional cosmological parameters.

  \subsubsection{Age of the Universe}
    \label{subsec:age-of-the-universe}

    Integrating the homogeneous relaxation relation $\dot{\chi} \simeq c$ gives
    \begin{equation}
      \chi(t) \simeq c t + \chi_{\mathrm{init}},
    \end{equation}
    where $\chi_{\mathrm{init}}$ denotes the effective value of $\chi$ at the onset
    of the relaxation regime relevant for cosmological observations.

    Neglecting $\chi_{\mathrm{init}}$ relative to present values leads to
    \begin{equation}
      t_0 \simeq \frac{\chi(t_0)}{c} \sim 4 \times 10^{17}~\mathrm{s},
    \end{equation}
    corresponding to approximately $13.8$ billion years, in agreement with
    standard cosmological estimates.

  \subsubsection{Redshift Interpretation}
    \label{subsec:redshift-interpretation}

    In Cosmochrony, cosmological redshift arises from the relative change of the
    $\chi$ field between emission and observation:
    \begin{equation}
      1 + z = \frac{\chi(t_{\mathrm{obs}})}{\chi(t_{\mathrm{emit}})}.
    \end{equation}

    This relation reproduces standard redshift phenomenology while attributing it
    to geometric scaling induced by $\chi$ relaxation rather than to recessional
    motion within a pre-existing spacetime.

  \subsubsection{Cosmic Microwave Background Scale}
    \label{subsec:cosmic-microwave-background-scale}

    At recombination ($z_{\mathrm{rec}} \simeq 1100$), the characteristic value of
    $\chi$ was correspondingly smaller:
    \begin{equation}
      \chi(t_{\mathrm{rec}}) \simeq \frac{\chi(t_0)}{1 + z_{\mathrm{rec}}}.
    \end{equation}

    Fluctuations imprinted at that epoch are subsequently stretched by the
    monotonic growth of $\chi$, providing a natural geometric interpretation of
    the observed angular scales in the cosmic microwave background.

  \subsubsection{Orders of Magnitude and Robustness}
    \label{subsec:orders-of-magnitude-and-robustness}

    All numerical estimates presented in this subsection rely solely on observed
    cosmological quantities and the bounded relaxation dynamics of $\chi$.
    No fine-tuning of parameters or detailed model fitting is assumed.

    While a fully quantitative cosmological model remains to be developed,
    these estimates demonstrate that Cosmochrony naturally reproduces the
    correct orders of magnitude for key observables, including the Hubble
    constant, cosmic age, redshift scaling, and CMB characteristic scales.

  \subsubsection{Summary}
    \label{subsec:summary}

    The $\chi$ framework admits a consistent normalization in observational units
    and yields cosmologically relevant scales without introducing additional
    fundamental parameters.
    These order-of-magnitude relations support the internal consistency of
    Cosmochrony and motivate further quantitative investigation.
