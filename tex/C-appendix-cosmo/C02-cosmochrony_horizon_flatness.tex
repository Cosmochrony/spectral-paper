\subsection{Resolution of the Horizon and Flatness Problems Without Inflation}
  \label{subsec:cosmochrony_horizon_flatness}

  In standard cosmology, the horizon and flatness problems arise from extrapolating a
  metric-based notion of causality back to the earliest stages of the universe.
  Inflation resolves these issues by postulating a brief phase of accelerated expansion,
  during which regions now widely separated were once in causal contact.

  Cosmochrony adopts a fundamentally different perspective.
  In this framework, spacetime geometry and its associated causal structure are not
  fundamental, but emerge from the relaxation dynamics of the scalar field $\chi$.
  As a result, causal connectivity is not constrained by a pre-existing metric at early
  stages, but is encoded directly in the global configuration of $\chi$ prior to the
  emergence of an effective spacetime description.

  \paragraph{Horizon problem.}
    The horizon problem is resolved because large-scale correlations do not need to be
    established through signal propagation within spacetime.
    Instead, they originate from the fact that $\chi$ constitutes a single, continuous
    dynamical substrate whose relaxation precedes and gives rise to spacetime itself.
    Regions that later appear causally disconnected in the emergent metric description
    may therefore share common field configurations inherited from earlier stages of
    $\chi$ evolution.

    In this sense, Cosmochrony replaces inflationary causal contact with
    \emph{pre-geometric connectivity}:
    correlations are established at the level of the fundamental field rather than through
    superluminal expansion or fine-tuned initial conditions imposed on a metric background.

  \paragraph{Flatness problem.}
    The flatness problem is similarly addressed without invoking an inflationary phase.
    In Cosmochrony, the effective spatial curvature reflects gradients and inhomogeneities
    in the relaxation rate of $\chi$.
    As $\chi$ relaxes monotonically toward a homogeneous state on large scales,
    curvature terms are dynamically diluted.
    Near-flat spatial geometry therefore emerges as a natural attractor of the relaxation
    dynamics, rather than as the result of exponential expansion.

    This mechanism does not require fine-tuning of initial curvature parameters.
    Instead, flatness reflects the tendency of the $\chi$ field to minimize large-scale
    geometric tension as relaxation progresses.

  \paragraph{Implications for primordial correlations.}
    Because large-scale coherence arises from the global structure of $\chi$ rather than
    from inflationary amplification of vacuum fluctuations, Cosmochrony allows for
    departures from strict scale invariance at the largest angular scales.
    In particular, constraints on the longest-wavelength modes of $\chi$ may lead to a
    suppression or modulation of power at low multipoles in the cosmic microwave background.

    Such effects are not interpreted here as definitive predictions, but as structural
    tendencies of the framework that may offer observational discrimination from
    inflation-based scenarios.

  \paragraph{Status of the description.}
    The arguments presented in this section establish that the horizon and flatness
    problems do not arise as fundamental inconsistencies within Cosmochrony.
    A quantitative derivation of the primordial power spectrum, including detailed
    predictions for CMB anisotropies, requires dedicated numerical simulations of
    $\chi$-field relaxation and lies beyond the scope of the present work.

    Nevertheless, the framework provides a conceptually consistent and inflation-free
    resolution of large-scale causal coherence and near-flat geometry, rooted in the
    pre-geometric dynamics of a single scalar field.
