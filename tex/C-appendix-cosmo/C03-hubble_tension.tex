\subsection{Evolution of the Hubble Parameter and the Hubble Tension}
  \label{app:hubble_tension}

  In Cosmochrony, the cosmological expansion is governed by the relaxation dynamics
  of the scalar field $\chi$, rather than by a balance between matter, radiation,
  and a dark energy component.
  The Hubble parameter therefore reflects the instantaneous relaxation rate of
  $\chi$ relative to its global value.

  \paragraph{Global expansion rate.}
    At the background level, the scale factor is proportional to $\chi$,
    \[
      a(t) \propto \chi(t),
    \]
    so that the Hubble parameter is given by
    \[
      H(t) = \frac{\dot{\chi}}{\chi}.
    \]
    In a perfectly homogeneous configuration, $\nabla \chi = 0$ and the relaxation
    equation reduces to $\dot{\chi} = c$, yielding
    \[
      H(t) = \frac{c}{\chi(t)}.
    \]
    This relation defines the \emph{global} expansion rate in Cosmochrony.
    Its precise redshift dependence beyond the homogeneous limit depends on how
    inhomogeneities and relaxation gradients contribute to the average dynamics,
    and is not assumed here to follow an exact power law at all epochs.

  \paragraph{Relaxation budget and effective expansion.}
    To quantify the influence of inhomogeneities, we introduce the dimensionless
    \emph{relaxation budget parameter}
    \begin{equation}
      \Omega_\chi \equiv \langle \beta^2 \rangle,
      \qquad \beta \equiv \frac{|\nabla \chi|}{c},
      \label{eq:chi-relaxation}
    \end{equation}
    which measures the fraction of the relaxation capacity stored in spatial
    gradients of $\chi$.
    In the late universe, these gradients are dominated by localized solitonic
    excitations and therefore track the matter distribution.

    The effective global expansion rate is then reduced according to
    \[
      \bar{H} = \frac{c}{\chi}\sqrt{1 - \Omega_\chi}.
    \]
    Empirically, $\Omega_\chi$ is constrained to be close to the observed matter
    fraction, $\Omega_\chi \simeq \Omega_m \approx 0.3$, providing a natural
    suppression of the expansion rate without invoking dark energy.

  \paragraph{Local expansion and the Hubble tension.}
    In an inhomogeneous universe, the relaxation budget varies locally.
    In a region with density contrast
    \[
      \delta = \frac{\rho - \bar{\rho}}{\bar{\rho}},
    \]
    we adopt the minimal closure relation
    \[
      \beta_{\text{loc}}^2 = \Omega_\chi (1+\delta),
    \]
    corresponding to a mean-field scaling between matter density and $\chi$-gradient
    energy.

    The local Hubble parameter then becomes
    \begin{equation}
      H_{\text{loc}}
      = \bar{H}
      \sqrt{\frac{1 - \Omega_\chi (1+\delta)}{1 - \Omega_\chi}}.
    \end{equation}
    In underdense regions ($\delta < 0$), the available relaxation capacity is higher,
    leading to $H_{\text{loc}} > \bar{H}$.

  \paragraph{Numerical consistency.}
    For $\Omega_\chi \approx 0.31$ and a local underdensity consistent with the KBC
    void ($\delta \approx -0.4$ on scales of $\sim 300$~Mpc), we obtain
    \[
      \frac{H_{\text{loc}}}{\bar{H}} \approx 1.08,
    \]
    corresponding to an $8\%$ enhancement of the locally inferred Hubble constant.
    This magnitude is sufficient to account for the observed Hubble tension between
    local distance-ladder measurements and global CMB-based inferences.

  \paragraph{Interpretation and status.}
    In Cosmochrony, the Hubble tension does not signal missing energy components or
    inconsistencies in early-universe physics.
    Instead, it arises as a non-linear environmental effect associated with the
    redistribution of the $\chi$-field relaxation budget in an inhomogeneous universe.

    While the framework naturally predicts a separation between local and global
    expansion rates, a fully quantitative determination of $H(z)$ across all
    redshifts requires numerical simulations of $\chi$ dynamics and is left for
    future work.
    The resolution of the Hubble tension, however, follows directly from the
    relaxation-based interpretation of cosmological expansion and constitutes a
    distinctive and testable signature of the theory.
