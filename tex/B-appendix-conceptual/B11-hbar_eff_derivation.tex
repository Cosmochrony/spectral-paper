\subsection{Spectral Stability and the Emergence of \(\hbar_{\mathrm{eff}}\)}
  \label{sec:hbar_eff_derivation}

  In Cosmochrony, the effective Planck constant \(\hbar_{\mathrm{eff}}\) is not
  introduced as a fundamental quantum postulate.
  Instead, it emerges as a scaling parameter linking spectral stability of
  \(\chi\)-field solitons to effective spacetime observables.

  \subsubsection*{Fundamental scales of the \(\chi\) dynamics}

    The \(\chi\) field is characterized by three independent dynamical scales:
    \begin{itemize}
      \item \(K_0\): maximal relaxation stiffness, with dimensions \([L^{-2}]\),
      \item \(\chi_c\): correlation length at which solitonic configurations stabilize,
      \item \(c\): maximal relaxation speed.
    \end{itemize}

    From these, one may define a natural unit of action associated with the relaxation dynamics,
    \begin{equation}
      \hbar_{\chi}
      \;\equiv\;
      \frac{c^3}{K_0 \chi_c} ,
    \end{equation}
    which has the dimensions of action and is independent of the standard Planck
    constant.

    \noindent
    Here and in the following, $K_0$ and $\chi_c$ denote the \emph{bare substrate parameters},
    i.e.\ universal invariants characterizing the rigidity and correlation capacity of the
    $\chi$ field. The scale-dependent values discussed in~\ref{sec:appendix-technical} arise only after
    coarse-graining and do not enter the definition of $\hbar_\chi$.

  \subsubsection*{Spectral origin of effective quantization}

    Quantization in Cosmochrony follows from the discrete spectrum of the stability
    operator \(\Delta_G^{(0)}\).
    For a solitonic excitation with eigenvalue \(\lambda_n\), the characteristic
    frequency of small oscillations scales as
    \begin{equation}
      \nu_n \;\sim\; \frac{c}{\chi_c}\,\sqrt{\lambda_n}\,\mathcal{N}_n^{1/2} .
    \end{equation}

    Identifying the rest energy with the product of this frequency and an effective
    action scale yields
    \begin{equation}
      E_n = \hbar_{\mathrm{eff}}\,\nu_n ,
    \end{equation}
    from which \(\hbar_{\mathrm{eff}}\) emerges as a geometric and spectral quantity,
    not as an independent constant.

  \subsubsection*{Regime-dependent scaling}

    The effective value of \(\hbar_{\mathrm{eff}}\) depends on the scale at which the
    system is probed.
    In regimes where the characteristic spacetime scale
    \(\ell_{\mathrm{spacetime}}\) is comparable to \(\chi_c\),
    \begin{equation}
      \hbar_{\mathrm{eff}} \approx \hbar_{\chi} ,
    \end{equation}
    recovering standard quantum behavior.

    At macroscopic scales \(\ell_{\mathrm{spacetime}} \gg \chi_c\),
    \begin{equation}
      \hbar_{\mathrm{eff}}
      \approx
      \hbar_{\chi}
      \left( \frac{\chi_c}{\ell_{\mathrm{spacetime}}} \right)^2 ,
    \end{equation}
    leading to a strong suppression of quantum effects and the emergence of classical
    behavior.

  \subsubsection*{Consistency with quantum phenomenology}

    In the microscopic regime, where \(\hbar_{\mathrm{eff}} \approx \hbar\), standard
    quantization relations
    \(E = \hbar \nu\) are recovered as effective descriptions.
    This agreement is not postulated but follows from the scaling behavior of
    \(\hbar_{\mathrm{eff}}\) once the projected regime matches laboratory scales.

    \paragraph{Numerical constraints.}

      Reproducing particle-scale quantum behavior requires
    \begin{equation}
      K_0 \chi_c^2 \sim \hbar ,
      \end{equation}
      which constrains the admissible values of the relaxation stiffness and correlation
      length.
      These constraints are consistent with soliton stability and do not require fine tuning.

\subsection{Renormalization of Substrate Parameters}
  \label{subsec:renormalization}

  To maintain consistency between the fundamental definition of $\hbar_\chi$
  and the scale-dependent observations in Appendix D, we distinguish between:
  \begin{itemize}
    \item \textbf{Bare Parameters ($K_0, \chi_c$):} Universal invariants of the $\chi$
    substrate that determine the fundamental quantum of action $\hbar_\chi$.
    \item \textbf{Effective Parameters ($K_{\text{eff}}, \chi_{\text{eff}}$):}
    Environment-dependent values emerging from the coarse-graining of relaxation constraints, as detailed in
    Section~\ref{sec:appendix-technical}.
  \end{itemize}

  The universality of $\hbar$ and the spectral invariant $\alpha_{\text{spec}}$ (formerly $\alpha$
  in Section B.9) stems from their dependence on the ratio of these bare quantities, which remains invariant under
  projective scaling.

  \noindent
  In particular, dimensionless coupling constants such as the electromagnetic fine-structure constant
  $\alpha_{\mathrm{EM}}$ do not inherit any arbitrariness from the substrate parameters.
  Within the Cosmochrony framework, the electric charge $e$ is not treated as a free gauge parameter, but as a property
  of localized solitonic configurations.
  The associated transmittance is not an adjustable quantity but a geometric invariant of the soliton's spectral
  embedding relative to the projection fiber~$\Pi$.
  As a result, the dependence on the substrate rigidity $K_0$ cancels out in dimensionless ratios, ensuring their
  invariance within a fixed relaxation epoch.

  \paragraph{Summary.}

    Within Cosmochrony, both inertial mass and effective quantization emerge from the same spectral stability structure
    of the \(\chi\) relaxation dynamics.
    The Planck constant appears not as a fundamental input, but as a scale-dependent parameter encoding the projection
    from relational dynamics to spacetime-based observables.
