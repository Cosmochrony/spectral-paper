\subsection{Soliton Energy and Structural Mass Scaling}
  \label{subsec:soliton_energy_mass}

  \paragraph{Status of this analysis.}
    This section presents a \emph{quantitative but non-numerical} analysis of the mass
    associated with localized solitonic configurations of the $\chi$ field.
    The goal is not to reproduce the observed particle mass spectrum, but to identify
    robust scaling relations, hierarchy constraints, and structural dependencies that
    emerge independently of microscopic details.
    A fully quantitative derivation of particle masses would require a complete
    effective field description, including gauge interactions and renormalization
    effects, which lies beyond the scope of the present work.

  \paragraph{Mass as integrated resistance to relaxation.}
    Within Cosmochrony, the mass of a localized excitation is interpreted as a measure
    of the total resistance it presents to the global relaxation of the $\chi$ field.
    Once an effective geometric description applies, this resistance may be summarized
    by an effective energy functional of the configuration,
    \begin{equation}
      M_{\mathrm{eff}} \;\propto\;
      \int_{\mathcal{V}}
      \left[
        \mathcal{T}(\nabla \chi)
        + \mathcal{U}(\chi)
      \right]
      \, d^3x ,
    \end{equation}
    where $\mathcal{T}$ encodes gradient-induced resistance and $\mathcal{U}$ represents
    nonlinear self-interaction terms stabilizing the soliton.
    This expression should be understood as an effective diagnostic measure rather than
    as a fundamental Hamiltonian.

  \paragraph{Scaling with soliton size and internal structure.}
    For a localized configuration characterized by a typical spatial scale $\ell$ and
    a characteristic amplitude $\Delta \chi$, dimensional analysis yields
    \begin{equation}
      M_{\mathrm{eff}} \;\sim\;
      \ell^3
      \left[
        \frac{(\Delta \chi)^2}{\ell^2}
        + V(\Delta \chi)
      \right].
    \end{equation}

    For simple one-dimensional kink-like solitons, the balance between gradient resistance
    and nonlinear self-interaction fixes the soliton width $\xi$ in terms of an effective
    curvature stiffness parameter $\lambda$ and the characteristic field scale $\chi_c$.
    In this case, the effective mass scale may be written schematically as
    \begin{equation}
      M_{\mathrm{eff}} \;\sim\; \sqrt{\lambda}\,\xi\,\chi_c^2 ,
    \end{equation}
    where $\lambda$ should be interpreted as an emergent, configuration-dependent quantity
    rather than as a fundamental coupling constant.

  In the regime where stabilization results from a balance between gradient resistance
    and nonlinear self-interaction, the soliton size $\ell$ is dynamically fixed, leading
    to a finite and stable effective mass.
    Importantly, different classes of solitons (kinks, vortices, knotted configurations)
    exhibit distinct scaling behaviors with respect to $\ell$ and $\Delta \chi$, implying
    that mass hierarchies arise structurally rather than from fine-tuning.

  \paragraph{Topological classes and mass hierarchy.}
    The effective mass depends not only on the size of the excitation but also on its
    topological class.
    Configurations with higher winding or linking numbers necessarily involve increased
    internal gradients, resulting in systematically larger resistance to relaxation.
    As a consequence, masses associated with different topological families obey
    ordering relations of the form
    \begin{equation}
      M_{n+1} \;>\; M_n ,
    \end{equation}
    where $n$ labels a topological invariant.
    This establishes a natural mechanism for discrete mass hierarchies without requiring
    the introduction of ad hoc parameters.

  \paragraph{Spectral interpretation.}
    From a spectral perspective, localized excitations correspond to bound modes of the
    linearized relaxation operator around a solitonic background.
    The effective mass is then controlled by the lowest nontrivial eigenvalue associated
    with the configuration,
    \begin{equation}
      M_{\mathrm{eff}} \;\sim\; \lambda_{\min}^{-1},
    \end{equation}
    where $\lambda_{\min}$ denotes the smallest positive eigenvalue governing the
    stability of the soliton.
    This formulation highlights that mass is fundamentally a spectral property of the
    $\chi$ dynamics rather than a free parameter.

  \paragraph{Robustness and universality.}
    The scaling relations derived above depend only on generic properties of the $\chi$
    field—locality, monotonic relaxation, and nonlinear stabilization—and are therefore
    expected to be robust against modifications of the microscopic details of the model.
    While numerical values of particle masses cannot be fixed at this level, the existence
    of discrete, ordered, and stable mass scales emerges as a structural prediction of
    the framework.

  \paragraph{Order-of-magnitude consistency.}
    Although the present analysis does not aim to reproduce the observed particle mass
    spectrum, it is instructive to examine whether the structural parameters entering the
    solitonic energy scale admit values compatible with known particle masses.

    For a simple kink-like soliton of characteristic width $\lambda^{-1}$ and amplitude
    set by the local relaxation scale $\chi_c$, the effective rest energy scales as
    \begin{equation}
      E_{\text{sol}} \sim \chi_c^2 \lambda ,
    \end{equation}
    up to dimensionless shape-dependent factors of order unity.

    Identifying this energy with the electron rest mass,
    $E_{\text{sol}} \sim m_e c^2 \approx 0.511\,\mathrm{MeV}$,
    and expressing all quantities in natural units ($\hbar = c = 1$),
    one finds that reproducing the electron mass requires an effective coupling of order
    \begin{equation}
      \lambda \sim 10^{-44},
    \end{equation}
    for $\chi_c$ normalized near the Planck scale.

    Such an extremely small value should not be interpreted as a fundamental parameter.
    Rather, it strongly suggests that the effective coupling $\lambda$ is dynamically
    generated through collective relaxation and topological constraints of the $\chi$ field,
    rather than being a bare microscopic constant.

  \paragraph{Summary.}
    Solitonic configurations of the $\chi$ field naturally possess finite effective
    masses determined by their size, internal structure, and topological class.
    Rather than predicting specific numerical values, Cosmochrony constrains the possible
    scaling and hierarchy of masses through geometric and spectral principles.
    This structural quantitativity provides a sound foundation for future extensions
    toward a fully predictive mass spectrum.
