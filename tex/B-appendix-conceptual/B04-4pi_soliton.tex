\subsection{Example: \(4\pi\)-Periodic Soliton and Spinorial Behavior}
  \label{subsec:4pi_soliton}

  This section provides an \emph{illustrative geometric construction} showing how
  spin-$\tfrac{1}{2}$--like behavior may emerge from a scalar $\chi$ configuration,
  without introducing a fundamental spinor field.

  The construction should be understood as an effective and topological model,
  intended to demonstrate plausibility rather than to constitute a complete
  microscopic derivation of fermionic degrees of freedom.

  \subsubsection{Phase-Twisted Solitonic Configuration}

    For convenience, the scalar field $\chi$ is represented in a complex form,
    \begin{equation}
      \chi(x) = \eta \tanh(\kappa x)\, e^{i \theta(x)},
    \end{equation}
    where the complex phase does \emph{not} represent an independent internal degree
    of freedom but serves as a parametrization of the internal oscillatory structure
    of a localized $\chi$ excitation.
    The underlying physical field remains real.

    Choosing
    \begin{equation}
      \theta(x) = \frac{x}{2},
    \end{equation}
    implies that the configuration returns to its original state only after a
    $4\pi$ variation of the phase,
    \begin{equation}
      \theta(x+4\pi)=\theta(x)+2\pi,
    \end{equation}
    while a $2\pi$ variation produces a sign inversion.

  \subsubsection{Topological Interpretation}

    This $4\pi$ periodicity reflects a topological obstruction analogous to that
    encountered in spinorial representations.
    Although the spatial configuration may appear unchanged after a $2\pi$ rotation,
    the internal state of the excitation is not.
    Only a $4\pi$ rotation restores full equivalence.

    This behavior mirrors the double-cover structure
    $\mathrm{SU}(2)\!\to\!\mathrm{SO}(3)$
    characteristic of spin-$\tfrac{1}{2}$ systems, without postulating a fundamental
    spinor field.
    Instead, the spinorial behavior arises from the topology of the solitonic
    configuration itself.

  \subsubsection{Relation to Fermionic Statistics}

    At the effective level, such $4\pi$-periodic excitations naturally acquire a
    minus sign under $2\pi$ rotations.
    In multi-excitation configurations, this topological property suggests an
    antisymmetric exchange behavior, providing a geometric basis for fermion-like
    statistics.

    This result does not constitute a proof of the spin--statistics theorem.
    Rather, it demonstrates that fermionic transformation properties and exclusion
    behavior can consistently emerge from topologically constrained scalar-field
    excitations.

  \subsubsection{Conceptual Scope}

    The present construction is intentionally minimal.
    It aims to show that:
    \begin{itemize}
      \item spinorial behavior does not require a fundamental spinor ontology,
      \item fermion-like properties may arise from topological constraints,
      \item scalar-field dynamics can support nontrivial exchange behavior.
    \end{itemize}

    A fully relational formulation of these topological properties, independent of
    any embedding geometry, is discussed in
    Appendix~\ref{app:relational_formulation}.
