\subsection{Relation to Classical Limits}
  \label{subsec:classical-limits}

  In regimes where the $\chi$ field varies slowly over large scales and localized
  excitations are dilute, the dynamics may be linearized around a homogeneous
  background configuration.
  In this limit, fluctuations of $\chi$ propagate as weak disturbances on an
  effectively flat geometric background.

  The resulting phenomenology reproduces the operational content of classical and
  quantum field theories formulated on Minkowski spacetime: wave propagation,
  superposition, and approximate locality emerge as effective properties of the
  coarse-grained $\chi$ dynamics.
  This correspondence should be understood as an \emph{effective recovery}, not as
  an ontological reduction of Cosmochrony to standard quantum field theory.

  Conversely, in regimes of high excitation density or strong spatial variation of
  $\chi$, nonlinear effects dominate.
  Large gradients slow the local relaxation rate, inducing effective curvature,
  time dilation, and horizon-like behavior.
  These regimes reproduce the phenomenology associated with curved spacetime and
  gravitational collapse, while remaining governed by the same underlying scalar
  dynamics.

  The classical limit in Cosmochrony therefore does not correspond to a separate
  theoretical layer, but to a dynamical regime in which collective behavior and
  coarse-graining suppress relational and topological effects.
  Classical spacetime and standard field dynamics emerge as stable, approximate
  descriptions valid when $\chi$ admits a smooth geometric interpretation.
