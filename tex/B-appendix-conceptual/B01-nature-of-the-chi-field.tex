\subsection{Nature of the $\chi$ Field}
  \label{subsec:nature-chi}

  The field $\chi$ is postulated as a real scalar quantity admitting an effective description
  as a smooth field $\chi(x^\mu)$ on a four-dimensional differentiable manifold
  once a stable geometric regime is reached.
  This manifold should not be regarded as fundamental, but as an emergent representation
  of the relational structure induced by $\chi$ itself.

  Unlike conventional scalar fields in quantum field theory, $\chi$ does not represent
  a matter degree of freedom propagating \emph{within} spacetime.
  Rather, it encodes the local geometric scale from which spacetime notions
  such as distance, duration, and causal structure arise as effective concepts.

  Operationally, $\chi$ may be interpreted as a proper wavelength field whose monotonic
  relaxation defines both temporal ordering and spatial separation.
  In this sense, spacetime coordinates serve only as convenient labels for
  the collective evolution of $\chi$, not as fundamental background entities.
