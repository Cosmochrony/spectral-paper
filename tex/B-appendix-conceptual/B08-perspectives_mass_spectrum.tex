\subsection{Perspectives: Towards a Derivation of the Proton-to-Electron Mass Ratio}
  \label{subsec:perspectives_mass_spectrum}

  The proton-to-electron mass ratio ($m_p/m_e \simeq 1836$) is one of the most
  precisely measured dimensionless constants in physics.
  Within the Cosmochrony framework, the aim of this section is not to derive this
  value from first principles, but to clarify how such a ratio could emerge from
  the spectral and topological structure of localized $\chi$-solitons.
  The discussion below should therefore be understood as a minimal spectral
  ansatz, intended to identify the relevant mechanisms and constraints rather than
  to provide a complete microscopic calculation.

  \subsubsection{Spectral Stability Hypothesis}

    Let $\chi_{\mathrm{sol}}$ denote a stationary localized configuration of the
    $\chi$ field.
    Small perturbations $\delta\chi$ around this background are governed, at the
    coarse-grained level, by a linear stability operator $\mathcal{L}_{\mathrm{sol}}$,
    defined as the second variation of an effective localization functional.
    Normal modes satisfy the eigenvalue problem
    \begin{equation}
      \mathcal{L}_{\mathrm{sol}} \psi_n = \lambda_n \psi_n .
    \end{equation}

    In regimes where an effective wave description applies, these modes exhibit
    oscillation frequencies $\omega_n = c\sqrt{\lambda_n}$.
    Identifying the lowest characteristic frequency with the rest energy of the
    excitation leads to the effective relation
    \begin{equation}
      m_n = \sqrt{\lambda_n}\,\chi_c ,
    \end{equation}
    where $\chi_c$ denotes a characteristic length scale associated with the spatial
    extent of the solitonic configuration.

    \paragraph{Dimensional consistency and effective scales.}
      The relation above should be understood as an effective parametrization.
      If $\chi_c$ is a pure length scale, then $\lambda_n$ must implicitly incorporate
      the appropriate conversion factors (e.g.\ powers of $c$ and an effective action
      scale) for $m_n$ to carry dimensions of mass.
      Equivalently, $\chi_c$ may be viewed as absorbing such factors, including the
      emergent scale $\hbar_{\mathrm{eff}}/c$ discussed in earlier sections.
      A first-principles derivation of these effective scales from the microscopic
      $\chi$ dynamics lies beyond the scope of the present work.
      Throughout the remainder of this section, the relation
      $m_n=\sqrt{\lambda_n}\,\chi_c$ is used as a convenient coarse-grained description,
      without implying that $\chi_c$ is a universal fundamental constant.

  \subsubsection{Elementary versus Composite Spectral Structures}

    A crucial distinction must be made between elementary and composite excitations
    within the spectral stability framework.
    Elementary particles, such as leptons, are expected to correspond to
    topologically elementary solitonic configurations whose inertial mass is
    dominated by a single lowest stability eigenvalue.
    By contrast, baryonic excitations are composite objects, whose mass reflects the
    combined contribution of several coupled stability modes associated with a bound
    configuration.

    In this view, mass ratios between elementary and composite particles cannot, in
    general, be expressed as the ratio of two single eigenvalues of the same operator.
    Rather, they take the schematic form
    \begin{equation}
      \frac{m_{\mathrm{comp}}}{m_{\mathrm{elem}}}
      \;\sim\;
      \frac{\sum_k \sqrt{\lambda^{(\mathrm{comp})}_k}}
      {\sqrt{\lambda^{(\mathrm{elem})}_0}},
      \label{eq:mass_ratio_schematic}
    \end{equation}
    where $\lambda^{(\mathrm{elem})}_0$ denotes the fundamental stability mode of an
    elementary soliton, and $\{\lambda^{(\mathrm{comp})}_k\}$ label the low-lying modes
    contributing to a composite bound structure.

  \subsubsection{Ansatz for the Proton as a Composite Soliton}

    As an exploratory working hypothesis, inspired by but not equivalent to Skyrme-type
    models, we consider the proton as a composite solitonic excitation.
    Specifically, we assume:
    \begin{itemize}
      \item The electron corresponds to a fundamental soliton with topological charge
      $Q_e = 1$ and eigenvalue $\lambda_e$.
      \item The proton corresponds to a bound state of three such elementary solitons,
      with total topological charge $Q_p = 3$, supplemented by an additional
      collective binding mode with eigenvalue $\lambda_{\mathrm{bind}}$.
    \end{itemize}

    The choice $Q_p=3$ is motivated by the observed threefold constituent structure
    of baryons, but is not derived here from a classification of solitonic topological
    sectors.
    Other composite configurations are not excluded by the present framework.

  \subsubsection{Mass Ratio from Spectral Scaling}

    Under the above assumptions, the effective eigenvalue associated with the proton
    may be written schematically as
    \begin{equation}
      \lambda_p \;\approx\; \lambda_{\mathrm{bind}} + 3\lambda_e ,
    \end{equation}
    leading to the mass ratio
    \begin{equation}
      \frac{m_p}{m_e}
      \;\approx\;
      \sqrt{\frac{\lambda_{\mathrm{bind}} + 3\lambda_e}{\lambda_e}} .
    \end{equation}

    In the binding-dominated regime $\lambda_{\mathrm{bind}} \gg \lambda_e$, this
    expression reduces to
    \begin{equation}
      \frac{m_p}{m_e} \;\approx\; \sqrt{\frac{\lambda_{\mathrm{bind}}}{\lambda_e}} .
    \end{equation}
    Matching the observed value $m_p/m_e \simeq 1836$ therefore imposes the spectral
    constraint
    \begin{equation}
      \frac{\lambda_{\mathrm{bind}}}{\lambda_e}
      \;\sim\; 3.4 \times 10^{6}.
    \end{equation}

    This relation is not derived here but identified as a target condition on the
    relative spectral scales of elementary and composite solitonic sectors.
    Whether such a hierarchy can arise naturally from specific topological
    connectivities and stability operators remains an open problem.

  \subsubsection{Open Questions and Research Directions}

    Several key questions must be addressed to turn this ansatz into a predictive
    framework:
    \begin{itemize}
      \item What topological features of composite solitons determine the magnitude
      of $\lambda_{\mathrm{bind}}$?
      \item Does a universal scaling law $\lambda_{\mathrm{bind}} =
      f(Q_p,Q_e,\chi_c)$ exist?
      \item Is the ratio $m_p/m_e$ stable under perturbations of the effective
      potential $V(\chi)$?
      \item Can the choice $Q_p=3$ be derived from a systematic classification of
      solitonic topological sectors?
    \end{itemize}

  \subsubsection{Summary}

    Within the Cosmochrony framework, the proton-to-electron mass ratio is interpreted
    not as a fundamental input, but as an emergent constraint on the relative spectral
    organization of elementary and composite solitonic excitations.
    The present analysis provides a consistent toy model that identifies the
    conditions such a framework must satisfy, while leaving their explicit
    realization to future analytical and numerical work.

\subsection{Role of $V(\chi)$ and Outlook}
  \label{subsec:role_vchi_mass}

  The effective potential $V(\chi)$ is expected to play a secondary role in mass
  generation, primarily by controlling fine splittings within a given solitonic
  sector rather than by setting the overall mass scale.

  \subsubsection{Eigenvalue Splittings and Fine Structure}

    In general, one may write
    \begin{equation}
      V(\chi) = \sum_n \lambda_n (\chi - \chi_c)^n ,
    \end{equation}
    where the coefficients $\lambda_n$ encode nonlinear interactions between
    solitonic modes.
    Differences such as the neutron--proton mass splitting could arise from small
    electromagnetic or topological corrections to this potential.
    No quantitative prediction is attempted here in the absence of an explicit form
    for $V(\chi)$.

  \subsubsection{Future Work}

    Key directions for future investigation include:
    \begin{itemize}
      \item Deriving the effective potential $V(\chi)$ from the underlying relaxation
      dynamics of the $\chi$ field.
      \item Constructing and classifying composite solitonic configurations and their
      associated stability operators.
      \item Performing numerical simulations to test whether large spectral
      hierarchies can arise without fine tuning.
    \end{itemize}

    In this sense, topology constrains the structure of the stability spectrum, while
    $V(\chi)$ controls fine splittings.
    The emergence of observed mass hierarchies is thus framed as a concrete but open
    spectral-geometric problem within the Cosmochrony framework.
