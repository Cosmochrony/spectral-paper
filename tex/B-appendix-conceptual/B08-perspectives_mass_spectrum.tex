\subsection{Perspectives: Towards a Derivation of the Mass Spectrum}
  \label{subsec:perspectives_mass_spectrum}

  While the identification of particles as topological $\chi$-solitons provides a
  geometric and dynamical origin for inertial mass—understood as resistance to
  relaxation—the explicit derivation of the observed particle mass spectrum
  remains an open challenge.

  In Cosmochrony, particle masses are not expected to arise from arbitrary
  coupling constants or symmetry-breaking scales, but from the intrinsic stability
  properties of localized configurations within the relaxation substrate.
  This motivates a spectral approach, in which masses emerge as characteristic
  frequencies associated with small fluctuations around solitonic backgrounds.

  \subsubsection{Spectral Stability Hypothesis}

    Let $\chi_{\mathrm{sol}}$ denote a stationary localized configuration.
    Small perturbations $\delta\chi$ around this background are governed, at the
    coarse-grained level, by a linear stability operator $\mathcal{L}_{\mathrm{sol}}$,
    defined as the second variation of an effective localization functional.
    Normal modes satisfy the eigenvalue problem
    \begin{equation}
      \mathcal{L}_{\mathrm{sol}} \psi_n = \lambda_n \psi_n .
    \end{equation}

    In regimes where an effective wave description applies, these modes exhibit
    oscillation frequencies $\omega_n = c\sqrt{\lambda_n}$.
    Identifying the lowest characteristic frequency with the rest energy of the
    excitation yields the relation
    \begin{equation}
      m_n = \frac{\hbar_{\mathrm{eff}}}{c}\sqrt{\lambda_n},
    \end{equation}
    where $\hbar_{\mathrm{eff}}$ acts as an effective conversion factor between
    geometric frequency and observed inertial mass.
    In this framework, $\hbar$ is not postulated as a fundamental constant, but
    emerges phenomenologically from the coarse-grained description of relaxation
    dynamics.

  \subsubsection{Geometric Origin of Mass Hierarchies}

    Within this spectral perspective, mass hierarchies correspond to the structure
    of the low-lying spectrum of $\mathcal{L}_{\mathrm{sol}}$.
    Different particle species arise from topologically inequivalent solitonic
    sectors, while generational structure may reflect higher-order stable modes
    associated with a single underlying configuration.

    This approach transforms the problem of particle masses into a question of
    spectral geometry: determining which network connectivities and stability
    conditions give rise to discrete, hierarchically separated eigenvalues.

  \subsubsection{Programmatic Outlook}

    Developing this proposal into a predictive framework requires:
    \begin{enumerate}
      \item constructing explicit classes of solitonic configurations and their
      associated stability operators,
      \item analyzing the dependence of the low-lying spectrum on topological and
      geometric parameters,
      \item and implementing numerical studies of large relaxation networks to test
      whether realistic mass hierarchies emerge without fine tuning.
    \end{enumerate}

    The purpose of the present section is not to claim a derivation of the Standard
    Model spectrum, but to establish a concrete and non-arbitrary route by which such
    a derivation could, in principle, be achieved within the Cosmochrony framework.
