\subsection{Validity of Geometric Descriptions}
  \label{subsec:validity-of-geometry}

  The emergence of curvature and metric structure discussed in the previous sections
  does not imply that geometric descriptions are universally applicable.
  Rather, geometry appears as an effective language, valid only in regimes where the
  relational structure admits a smooth and locally stable continuum approximation.

  In such regimes, geometric relations exhibit a remarkable robustness.
  Local curvature, geodesic deviation, and horizon structure depend only weakly on
  microscopic details of the underlying relational system.
  This explains the universality of geometric behavior observed across a wide range
  of physical situations.

  Outside these regimes, the failure of geometric descriptions should not be
  interpreted as a breakdown of geometric laws.
  Instead, it reflects the loss of applicability of the geometric language itself.
  When local injectivity or smoothness conditions are violated, the notion of
  spacetime ceases to be operationally meaningful, and no geometric description is
  assumed.

  From this perspective, geometric relations function as internal consistency
  conditions of emergent spacetime descriptions rather than as fundamental laws
  governing an underlying substrate.
  Geometry is exact where it applies and silent elsewhere.
