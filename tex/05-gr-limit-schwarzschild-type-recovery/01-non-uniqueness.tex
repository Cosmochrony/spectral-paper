\subsection{Non-uniqueness of Spectral Reconstruction}
  \label{subsec:non-uniqueness}

  The reconstruction of an effective metric from relational spectral data is
  generically non-unique.
  This non-uniqueness is not a consequence of approximation or incomplete information,
  but a structural feature of spectral geometry.

  Spectral reconstruction proceeds by extracting geometric descriptors from a restricted
  set of spectral features, typically after the application of admissibility or
  regularity criteria.
  As a result, distinct relational configurations may share identical low-resolution
  spectral content and therefore give rise to the same effective metric description.
  Conversely, a given relational structure may admit multiple continuum embeddings that
  are spectrally equivalent within the admissible window.

  Operational notions of distance further reinforce this ambiguity.
  Distance is defined in terms of correlation efficiency or spectral proximity rather
  than through an underlying coordinate separation.
  Different choices of spectral filtering, coarse-graining scale, or embedding procedure
  may therefore lead to metrically equivalent but geometrically distinct descriptions.

  Importantly, this ambiguity does not compromise physical consistency.
  All equivalent metric descriptions agree on observable geometric properties within
  their shared domain of validity.
  The effective metric should thus be understood as a representative of an equivalence
  class of geometries compatible with the same relational spectral data.

  Non-uniqueness is therefore an intrinsic aspect of emergent geometry in relational
  frameworks.
  It reflects the fact that geometric descriptions encode only a subset of the underlying
  relational structure and should not be interpreted as faithful microscopic
  representations.
