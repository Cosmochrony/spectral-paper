\subsection{Non-uniqueness of Spectral Reconstruction}
  \label{subsec:non-uniqueness}

  The reconstruction of an effective metric from relational spectral data is
  generically non-unique.
  This non-uniqueness is not a consequence of approximation or incomplete information,
  but a structural feature of spectral geometry.

  Spectral reconstruction proceeds by extracting geometric descriptors from a restricted
  set of spectral features, typically after the application of admissibility or
  regularity criteria.
  As a result, distinct relational configurations may share identical low-resolution
  spectral content and therefore give rise to the same effective metric description.
  Conversely, a given relational structure may admit multiple continuum embeddings that
  are spectrally equivalent within the admissible window.

  Operational notions of distance further reinforce this ambiguity.
  Distance is defined in terms of correlation efficiency or spectral proximity rather
  than through an underlying coordinate separation.
  Different choices of spectral filtering, coarse-graining scale, or embedding procedure
  may therefore lead to metrically equivalent but geometrically distinct descriptions.

  Importantly, this ambiguity does not compromise physical consistency.
  All equivalent metric descriptions agree on observable geometric properties within
  their shared domain of validity.
  The effective metric should thus be understood as a representative of an equivalence
  class of geometries compatible with the same relational spectral data.

  Non-uniqueness is therefore an intrinsic aspect of emergent geometry in relational
  frameworks.
  It reflects the fact that geometric descriptions encode only a subset of the underlying
  relational structure and should not be interpreted as faithful microscopic
  representations.

  \paragraph{Flux conservation and radial scaling.}
    The emergence of a $1/r$-type effective potential does not rely on an imposed
    force law.
    It follows from the conservation of relaxation flux in the relational substrate.
    In a quasi-static regime, the total relaxation flux crossing any closed
    relational surface surrounding a localized perturbation is conserved.

    In an effectively three-dimensional projectable regime, this conservation
    implies that the perturbation of connectivity must decay as $1/r^{2}$ with
    radial distance.
    When integrated along admissible paths, this scaling naturally yields an
    effective potential proportional to $1/r$, providing the structural origin
    of the Schwarzschild radius without assuming Newtonian dynamics or a prior
    gravitational law.

  \paragraph{Origin of the $1/r$ profile.}
    In the present framework, the notion of mass does not correspond to a fundamental
    source term, but to a localized inhibition of admissible relaxation in the relational
    substrate.
    Such a configuration reduces the local density of optimal relational paths, as measured
    by shortest-path connectivity under the operational distance defined via Dijkstra-type
    minimization.

    In the continuum regime where an effective geometric description becomes valid,
    this reduction acts as a source term for the scalar Laplacian governing admissible
    deformations of the effective metric.
    In three spatial dimensions, the fundamental solution of the corresponding Poisson
    equation is uniquely fixed by flux conservation and exhibits a $1/r$ decay.
    The emergence of the Schwarzschild factor $1 - r_s/r$ therefore follows directly from
    the conservation of relational relaxation flux in an isotropic three-dimensional
    effective geometry, rather than from an imposed potential or symmetry assumption.
