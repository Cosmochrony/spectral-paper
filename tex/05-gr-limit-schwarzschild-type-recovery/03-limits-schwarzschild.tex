\subsection{Limits of Schwarzschild-Type Reconstruction}
  \label{subsec:limits-schwarzschild}

  The recovery of a Schwarzschild-type metric in the previous section illustrates
  how familiar gravitational geometries can emerge from relational spectral data
  under restrictive conditions.
  This reconstruction is, however, intrinsically regime-dependent and should not be
  interpreted as universally applicable.

  The derivation relies on assumptions of approximate stationarity, spherical
  symmetry, and weak inhomogeneity.
  Outside these regimes, the effective geometric description may differ substantially
  from the Schwarzschild form or cease to be meaningful altogether.

  The characteristic length scale appearing in the static spherically symmetric
  metric arises as an integration constant of the geometric reconstruction.
  Its identification with physical parameters depends on the choice of spectral
  filtering and on the operational resolution at which the geometry is probed.
  As a result, Schwarzschild-type metrics should be understood as effective geometric
  representatives rather than as fundamental solutions.

  Classical weak-field tests confirm the consistency of the geometric approximation
  within its domain of validity.
  They do not, however, constrain the behavior of the relational system outside this
  regime, where the assumptions underlying the metric reconstruction no longer hold.

  These considerations highlight the importance of distinguishing between the
  existence of a geometric description and its range of applicability.
  Schwarzschild geometry emerges where appropriate but does not define the behavior
  of the system in regimes where a continuum spacetime interpretation breaks down.
