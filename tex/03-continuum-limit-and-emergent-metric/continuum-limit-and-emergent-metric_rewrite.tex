\clearpage
\section{Continuum Limit and Emergent Metric}
  \label{sec:continuum-limit-and-emergent-metric}

  The spectral distance introduced in the previous section provides a purely relational
  notion of proximity.
  A natural question is under which conditions such a construction admits an effective
  continuum description, allowing the use of familiar geometric tools.
  In this section, we analyze the regimes in which a relational spectral structure can be
  approximated by a smooth metric geometry.

  We consider families of relational graphs or coarse-grained networks whose Laplacian
  spectra exhibit sufficient regularity.
  No assumption is made concerning an underlying manifold or dimensionality.
  Instead, the continuum limit is defined operationally, as the regime in which spectral
  distances vary smoothly and admit consistent local approximations.

  A key requirement for the emergence of a continuum description is local
  factorizability.
  In such regimes, the relational system admits an approximate decomposition into
  weakly coupled subsystems, allowing spectral distances to be defined locally and
  independently.
  This property underlies the emergence of effective locality and permits the
  introduction of local coordinate charts as auxiliary descriptive tools.

  The continuum approximation is inherently non-unique.
  Because spectral reconstruction is generically non-injective, distinct relational
  configurations may give rise to identical effective metric descriptions.
  Conversely, a given relational structure may admit multiple equivalent continuum
  embeddings.
  This ambiguity is a structural feature of spectral geometry and does not signal a loss
  of consistency.

  When local factorizability and spectral regularity are satisfied, the spectral distance
  admits a local quadratic approximation.
  In this regime, one may introduce an effective metric tensor $g_{\mu\nu}(x)$ defined by
  the leading-order expansion
  \begin{equation}
    d_{\mathrm{spec}}(i,j)^2
    \;\approx\;
    g_{\mu\nu}(x)\,\Delta x^\mu \Delta x^\nu ,
  \end{equation}
  where $\Delta x^\mu$ denotes coordinate differences in a local embedding.
  The metric tensor is not a fundamental structure but a derived descriptor encoding the
  local behavior of spectral distances.

  The validity of the continuum approximation is limited to regimes where higher-order
  spectral corrections remain subdominant.
  Outside such regimes, no smooth geometric interpretation is assumed, and the relational
  description must be treated in its full spectral form.

  The framework developed here is purely kinematical.
  No assumptions regarding dynamics, temporal ordering, or causal structure are required
  for the emergence of an effective metric description.
  The analysis focuses solely on the conditions under which relational spectral data
  support a continuum geometric approximation.

  \subsection{Spectral Admissibility and Regularity}
  \label{subsec:spectral-admissibility}

  The emergence of an effective continuum geometry from the relational spectral data
  requires additional regularity conditions.
  Not all relational configurations admit a meaningful continuum approximation,
  even when a spectral distance can be defined.
  In this subsection, we introduce the spectral admissibility criteria that characterize
  the regimes in which the geometric reconstruction is well-defined.

  Let $L$ denote a self-adjoint relational operator acting on a suitable Hilbert space for the configurations.
  In discrete realizations, $L$ is reduced to a graph Laplacian associated with the
  relational connectivity of the system.
  In general, $L$ encodes a relational structure without reference to the background manifold or metric.

  The operator $L$ admits a spectral decomposition
  \begin{equation}
    L \psi_n = \lambda_n \psi_n ,
  \end{equation}
  with non-negative eigenvalues $\{\lambda_n\}$ and the corresponding eigenmodes $\{\psi_n\}$.
  No geometric interpretation was assumed at this stage.

  Spectral admissibility is defined by restricting the attention to a controlled spectral
  window.

  Specifically, we introduce a smooth spectral filter
  \begin{equation}
    F_{\lambda_*} \;=\; f\!\left(\frac{L}{\lambda_*}\right),
  \end{equation}
  where $f(x)$ is a fixed cutoff function and $\lambda_*$ sets a characteristic spectral scale.
  Only modes below this scale contribute significantly to the effective geometric reconstruction.

  This filtering procedure defines the admissibility purely in spectral terms.
  It does not rely the locality, coordinates, or spatiotemporal integration measures.
  Instead, it reflects that continuum geometry, when it emerges, is
  necessarily insensitive to fine-grained spectral details beyond a given resolution.

  Spectral admissibility is generally non-injective.
  Distinct relational configurations may share identical spectral content within the
  admissible window and therefore give rise to the same effective continuum geometry.
  Conversely, a given relational structure can admit multiple equivalent continuum embeddings.
  This non-uniqueness is a structural feature of spectral reconstruction and does not
  indicate inconsistency in the framework.


  In some reconstruction schemes, admissibility criteria may involve monotonic
  spectral filters or ordering parameters, reflecting the fact that effective
  continuum descriptions are typically insensitive to fine-grained spectral
  rearrangements beyond a given resolution.

  We emphasize that the spectral reconstruction of an effective metric does not
  require the introduction of additional fundamental fields or degrees of freedom;
  the geometric structures discussed here arise solely from relational spectral
  data and their continuum approximation.
