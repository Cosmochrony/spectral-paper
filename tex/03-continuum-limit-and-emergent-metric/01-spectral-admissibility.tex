\subsection{Spectral Admissibility and Regularity}
  \label{subsec:spectral-admissibility}

  The emergence of an effective continuum geometry from relational spectral data
  requires additional regularity conditions.
  Not all relational configurations admit a meaningful continuum approximation,
  even when a spectral distance can be defined.
  In this subsection, we introduce spectral admissibility criteria that characterize
  the regimes in which geometric reconstruction is well-defined.

  Let $L$ denote a self-adjoint relational operator acting on a suitable Hilbert
  space of configurations.
  In discrete realizations, $L$ reduces to a graph Laplacian associated with the
  relational connectivity of the system.
  More generally, $L$ encodes relational structure without reference to a background
  manifold or metric.

  The operator $L$ admits a spectral decomposition
  \begin{equation}
    L \psi_n = \lambda_n \psi_n ,
  \end{equation}
  with non-negative eigenvalues $\{\lambda_n\}$ and corresponding eigenmodes
  $\{\psi_n\}$.
  No geometric interpretation is assumed at this stage.

  Spectral admissibility is defined by restricting attention to a controlled spectral
  window.
  Concretely, we introduce a smooth spectral filter
  \begin{equation}
    F_{\lambda_*} \;=\; f\!\left(\frac{L}{\lambda_*}\right),
  \end{equation}
  where $f(x)$ is a fixed cutoff function and $\lambda_*$ sets a characteristic
  spectral scale.
  Only modes below this scale contribute significantly to the effective geometric
  reconstruction.

  This filtering procedure defines admissibility purely in spectral terms.
  It does not rely on locality, coordinates, or spacetime integration measures.
  Instead, it reflects the fact that continuum geometry, when it emerges, is
  necessarily insensitive to fine-grained spectral details beyond a given resolution.

  Spectral admissibility is generically non-injective.
  Distinct relational configurations may share identical spectral content within the
  admissible window and therefore give rise to the same effective continuum geometry.
  Conversely, a given relational structure may admit multiple equivalent continuum
  embeddings.
  This non-uniqueness is a structural feature of spectral reconstruction and does not
  indicate an inconsistency of the framework.
