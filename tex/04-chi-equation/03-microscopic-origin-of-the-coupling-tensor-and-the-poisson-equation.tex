\subsection{Microscopic Origin of the Coupling Tensor and the Poisson Equation}
  \label{subsec:microscopic-origin-of-the-coupling-tensor-and-the-poisson-equation}

  To ensure internal consistency, the effective coupling governing the relaxation of
  $\chi$ cannot be taken as a universal constant.
  Instead, it must reflect the local structural state of the field.
  In Cosmochrony, this dependence is encoded through a constitutive relation linking
  the effective coupling strength to variations of $\chi$ within the emergent
  geometric description.

  A convenient phenomenological form for this effective coupling is:
  \begin{equation}
    K_{\mathrm{eff}}
    = K_0 \exp\!\left(-\frac{(\Delta \chi)^2}{\chi_c^2}\right),
    \label{eq:effective_coupling_tensor}
  \end{equation}
  where $\Delta\chi$ denotes the effective local variation of $\chi$ across correlated
  configurations, $K_0$ is the vacuum coupling strength, and $\chi_c$ sets the
  characteristic scale at which structural inhomogeneities significantly reduce the
  relaxation conductivity.

  Regions exhibiting strong structural variation of $\chi$, such as localized
  solitonic configurations, therefore weaken the effective coupling and locally slow
  down the relaxation process.
  This slowdown underlies the emergent gravitational phenomenology.

  In regimes where an effective spacetime description applies, the local relaxation
  rate $\mathcal{D}_{\mathrm{loc}}\chi$ differs from its asymptotic value
  $\mathcal{D}_0$ far from localized excitations.
  An effective gravitational potential $\Phi$ may then be introduced through the
  relation:
  \begin{equation}
    \frac{\mathcal{D}_{\mathrm{loc}}\chi}{\mathcal{D}_0}
    \simeq 1 + \frac{\Phi}{c^2},
    \label{eq:relaxation_potential_relation}
  \end{equation}
  which defines $\Phi$ as a convenient parametrization of the relative slowdown of
  relaxation in the effective geometric regime.

  In the weak-structure limit, where variations of $\chi$ are small compared to
  $\chi_c$, the spatial distribution of $\Phi$ is well described by an effective
  elliptic relation of Poisson type:
  \begin{equation}
    \nabla^2 \Phi \simeq 4\pi G_{\mathrm{eff}} \rho,
    \label{eq:effective_poisson_equation}
  \end{equation}
  where $\rho$ denotes the density of localized excitations and
  $G_{\mathrm{eff}}$ is an emergent coupling parameter determined by the collective
  response of the $\chi$ field.

  This Poisson equation is not fundamental, but represents the weak-field,
  coarse-grained limit of the constrained relaxation dynamics of $\chi$.
  Gravitation thus appears not as an external interaction, but as an effective
  manifestation of reduced relaxation conductivity induced by solitonic
  configurations.

  A fully relational formulation consistent with, but not required for, the effective
  description presented here is provided in Appendix~\ref{app:relational_formulation}.
