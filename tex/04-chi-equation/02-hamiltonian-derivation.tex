\subsection{Hamiltonian Derivation of the Evolution Equation}
  \label{subsec:hamiltonian-derivation}

  \subsubsection*{Effective Constraint on Local Relaxation}

    Although the fundamental dynamics of $\chi$ is formulated without reference to
    spacetime or a Hamiltonian structure, its relaxation is subject to universal local
    constraints.
    In regimes where $\chi$ configurations admit a stable geometric interpretation,
    these constraints may be summarized in a compact, Hamiltonian-like form.

    Specifically, the local relaxation rate of $\chi$ is bounded by the invariant
    constant $c$.
    When expressed in effective geometric variables, this bound can be written as:
    \begin{equation}
    (\mathcal{D}_{\text{loc}}\chi)^2 + |\nabla \chi|_{\text{eff}}^2 = c^2 ,
    \label{eq:effective_hamiltonian_constraint}
    \end{equation}
    where $|\nabla \chi|_{\text{eff}}$ denotes the effective magnitude of spatial
    variations of $\chi$ within the coarse-grained geometric description.
    This relation does not define a fundamental Hamiltonian, but provides a concise
    summary of the admissible local relaxation configurations.

    Selecting the monotonic relaxation branch yields the effective evolution equation:
    \begin{equation}
      \mathcal{D}_{\text{loc}}\chi
      = c \sqrt{1 - \frac{|\nabla \chi|_{\text{eff}}^2}{c^2}} ,
      \label{eq:chi_dynamics}
    \end{equation}
    which encodes the slowdown of relaxation induced by structural variations of
    $\chi$.

  \subsubsection*{Emergent Gravitational Interpretation}

    Localized excitations of $\chi$ generate regions of enhanced relational structure,
    which locally reduce the relaxation rate.
    When interpreted in the effective spacetime description, this manifests as
    gravitational time dilation.
    No independent gravitational field is postulated; the effect arises directly from
    the constrained relaxation dynamics of $\chi$.

    In the weak-structure regime ($|\nabla \chi|_{\text{eff}} \ll c$), the effective
    description admits a simplified elliptic relation governing the spatial
    distribution of relaxation slowdown:
    \begin{equation}
      \nabla \cdot \left( \frac{\nabla \chi}
      {\sqrt{1 - |\nabla \chi|^2 / c^2}} \right)
      \simeq \frac{4 \pi G_{\text{eff}}}{c^2} \rho ,
      \label{eq:effective_poisson_relation}
    \end{equation}
    where $\rho$ denotes the density of localized excitations and $G_{\text{eff}}$ is
    an emergent coupling parameter characterizing the collective influence of such
    structures on the relaxation flow.

    In the Newtonian limit, this reduces to an effective Poisson equation for a
    gravitational potential $\Phi$ defined by:
    \begin{equation}
      \Phi \equiv c^2 \ln \left( \frac{\mathcal{D}_{\text{loc}}\chi}{c} \right),
      \label{eq:effective_gravitational_potential}
    \end{equation}
    recovering standard gravitational phenomenology as a coarse-grained description of
    $\chi$ dynamics.

  \subsubsection*{Interpretational Status}

    The relations above do not constitute a fundamental Hamiltonian or variational
    theory.
    They provide an effective, local summary of the constraints governing $\chi$
    relaxation in regimes where spacetime notions apply.
    The predictive content of Cosmochrony remains entirely encoded in the intrinsic,
    pre-geometric relaxation dynamics of $\chi$, while geometric and gravitational
    structures emerge only as descriptive tools.
