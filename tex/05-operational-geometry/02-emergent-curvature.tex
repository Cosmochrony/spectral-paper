\subsection{Emergent Curvature}
  \label{subsec:emergent-curvature}

  In a relational framework, curvature arises as a collective geometric descriptor
  rather than as a primitive property of an underlying spacetime.
  Spatial variations in relational connectivity and spectral coupling lead to
  non-uniform correlation patterns across extended regions.
  When a smooth geometric parametrization is applicable, these non-uniformities are
  compactly summarized by gradients of an effective metric structure.

  From this perspective, curvature does not represent an independent dynamical field.
  It functions as a macroscopic descriptor encoding how localized relational features
  modulate collective connectivity.
  The metric does not act as a causal agent; it provides a concise representation of
  constrained relational organization in regimes admitting a continuum approximation.

  Within such regimes, the resulting curvature reproduces the familiar geometric
  phenomenology associated with curved spacetime, including geodesic deviation,
  gravitational redshift, and lensing effects.
  These phenomena arise here from spatial variations in relational coupling rather
  than from a fundamental spacetime geometry.

  The use of geometric language is therefore strictly operational.
  Curvature and metric quantities are introduced only insofar as they provide an
  accurate summary of relational correlation patterns.
  Outside regimes where a smooth and slowly varying approximation is valid, no
  geometric interpretation is assumed.
