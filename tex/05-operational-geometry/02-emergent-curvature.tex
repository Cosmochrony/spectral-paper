\subsection{Emergent Curvature}
  \label{subsec:emergent-curvature}

  In a relational framework, curvature arises as a collective geometric descriptor
  rather than as a primitive property of underlying spacetime.
  Spatial variations in relational connectivity and spectral coupling lead to
  non-uniform correlation patterns across extended regions.
  When a smooth geometric parametrization is applicable, these non-uniformities are
  compactly summarized by the gradients of an effective metric structure.

  From this perspective, the curvature does not represent an independent dynamic field.
  It functions as a macroscopic descriptor that encodes how localized relational features
  modulate collective connectivity.
  The metric does not act as a causal agent; it provides a concise representation of
  the constrained relational organization in regimes that admit a continuum approximation.

  Within such regimes, the resulting curvature reproduces the familiar geometric
  phenomenology associated with curved spacetime, including the geodesic deviation,
  gravitational redshift, and lensing effects.
  These phenomena arise from spatial variations in relational coupling, rather
  than from fundamental spacetime geometry.

  Therefore, the use of geometric language is strictly operational.
  Curvature and metric quantities are introduced only insofar as they provide an
  accurate summary of the relational correlation patterns.
  Outside the regimes where a smooth and slowly varying approximation is valid, no
  geometric interpretation is assumed.
