\abstract{
  \textbf{Background:}
  We present a relational and spectral construction of effective spacetime geometry in which metric notions arise solely
  from correlation structure, without assuming a background manifold, coordinates, or fundamental geometric degrees of
  freedom.

  \textbf{Methods:}
  Starting from a purely relational substrate endowed with a symmetric connectivity operator, we define operational
  distances through minimal path functionals. A non-circular coarse-graining scheme is introduced, distinguishing
  pre-geometric combinatorial neighborhoods from geometry-aware weighted distances. Spectral admissibility criteria
  identify regimes in which relational variations become sufficiently smooth to support an effective geometric
  description.

  \textbf{Results:}
  In these projectable regimes, the resulting distance matrix admits a low-dimensional embedding, enabling the
  reconstruction of emergent coordinates and an effective metric structure. Standard geometric observables—such as proper
  time, spatial distance, and curvature—arise as descriptive summaries of relational constraints. In appropriate limits,
  the effective metric reproduces general-relativistic phenomenology, including the recovery of Schwarzschild geometry for
  isolated, approximately symmetric configurations, without postulating gravitational dynamics at the fundamental level.

  \textbf{Conclusions:}
  The framework naturally predicts the breakdown of geometric description when spectral gaps close or relational structure
  becomes non-local, providing intrinsic criteria for the limits of continuum spacetime. Numerical and analytical results
  supporting a universal spectral hierarchy are presented in the appendices. Overall, this work establishes a concrete
  pathway from relational spectral data to emergent metric geometry, positioning spacetime as an operational construct
  rather than a primitive entity.
}
