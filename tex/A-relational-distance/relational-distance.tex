\section{Relational Distance as a Minimal Path Functional}
  \label{app:relational-distance-minimal-path}

  A central step in the relational construction is the introduction of a distance notion
  defined \emph{within} the relational adjacency structure underlying the $\chi$
  description, without presupposing any embedding space or fundamental lattice.
  To avoid ambiguity and to prevent hidden circularity in subsequent coarse-graining
  procedures, we explicitly distinguish two distances that operate at different descriptive
  levels.

  \subsection{Combinatorial vs.\ weighted distance.}
    \label{subsec:combinatorial-vs.-weighted-distance}
    We define:

    \begin{enumerate}
      \item \textbf{Combinatorial distance} $d_{ij}^{C}$ (pre-geometric).
      \[
        d_{ij}^{C} \;=\; \min_{\gamma_{ij}} \sum_{(u,v)\in\gamma_{ij}} 1,
      \]
      where $\gamma_{ij}$ is any path connecting nodes $i$ and $j$ through the network links.
      This distance counts \emph{graph steps only} and is \textbf{independent of the field values of $\chi$}\citep{Diestel2017,Newman2010}.
      It is used to define the neighborhood sets employed in relational averaging\citep{Dijkstra1959,Cormen2009}.

      \item \textbf{Weighted distance} $d_{ij}^{W}$ (emergent / effective).
      \[
        d_{ij}^{W} \;=\; \min_{\gamma_{ij}} \sum_{(u,v)\in\gamma_{ij}} w_{uv},
      \]
      where $w_{uv}$ is a positive weight associated with each link. This distance is used
      for the \textbf{emergent geometry} (and in particular for spectral constructions),
      because it encodes the effective relational stiffness of the network.
      The minimization over weighted paths corresponds to the classical shortest-path
      problem, efficiently solvable by Dijkstra-type algorithms~\cite{Dijkstra1959}.
      Algorithmic aspects of weighted shortest-path computations and their
      computational complexity are discussed in standard references~\cite{Cormen2009}.

    \end{enumerate}

    This distinction ensures that the coarse-graining background $\bar{\chi}$ can be defined
    using $d_{ij}^{C}$ without any metric dependence, while the effective geometry is encoded
    by $d_{ij}^{W}$ through weights that depend only on $\bar{\chi}$ (not on instantaneous $\chi$).

    The combinatorial distance $d_{ij}^{C}$ has no physical interpretation and plays no
    direct observational role; it serves solely as an auxiliary construct for defining
    relational neighborhoods in a pre-geometric manner.

  \subsection{Weight functional and positivity.}
    \label{subsec:weight-functional-and-positivity}
    We parameterize the weights by an effective connectivity (stiffness) matrix $K_{uv}>0$:
    \[
      w_{uv} \;=\; \frac{1}{K_{uv}}.
    \]
    In the circularity-free construction used in this appendix, $K_{uv}$ is not taken as a
    direct functional of $\chi$, but as a functional of a slowly varying \emph{background} field
    $\bar{\chi}$ defined by relational averaging (see Appendix~E.5). Concretely, we use
    \begin{equation}
      w_{uv}(\bar{\chi})
      \;=\;
      \frac{1}{K_0}
      \left[
        1 + \left(\frac{\bar{\chi}_u-\bar{\chi}_v}{\chi_c}\right)^2
      \right],
      \qquad
      K_{uv}(\bar{\chi}) \;=\; \frac{1}{w_{uv}(\bar{\chi})}.
      \label{eq:weight_function_background}
    \end{equation}
    The positivity $w_{uv}(\bar{\chi})>0$ is guaranteed by construction, so $d_{ij}^{W}$ is a
    well-defined weighted path metric whenever the graph is connected on the domain considered.

  \subsection{Metric status.}
    \label{subsec:metric-status}
    Both the combinatorial distance $d_{ij}^{C}$ and the weighted distance $d_{ij}^{W}$ define
    proper metric spaces on the $\chi$-network, albeit at different descriptive levels:
    $d_{ij}^{C}$ is discrete and topological, while $d_{ij}^{W}$ encodes emergent relational structure.
    This duality is essential: $d_{ij}^{C}$ provides a pre-geometric scaffold for defining $\bar{\chi}$,
    whereas $d_{ij}^{W}$ provides the effective distance used in the emergent geometric regime.
