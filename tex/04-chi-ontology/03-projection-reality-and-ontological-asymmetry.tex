\subsection{Projection, Reality, and Ontological Asymmetry}
  \label{subsec:projection-reality-and-ontological-asymmetry}

  Within this ontology, the emergence of spacetime should be understood as a projection
  from the $\chi$ substrate, rather than as a dual or equivalent description.
  The projected universe is fully real at the physical level, but its reality is
  derivative: spacetime entities, effective fields, and dynamical laws do not possess
  ontological primacy~\cite{Rovelli2021}.

  This asymmetry is essential.
  While physical descriptions depend on the projection of $\chi$, the converse is not
  true.
  The $\chi$ structure exists independently of spacetime notions and does not admit a
  reformulation entirely in geometric or dynamical terms.
  Projection must therefore be understood as non-injective: distinct structural
  features of $\chi$ may correspond to identical or physically indistinguishable
  effective configurations.

  \paragraph{Projection and non-circularity.}
    Because geometric notions arise only after projection, the construction of effective
    descriptions must not presuppose any metric or temporal structure defined at the
    effective level.
    In particular, coarse-graining procedures used to define $\chi_{\mathrm{eff}}$ must
    avoid relying implicitly on emergent geometric quantities.
    This requirement motivates the explicit separation between pre-geometric relational
    structures and geometry-dependent observables developed in Appendix~E, where
    combinatorial and weighted distances are carefully distinguished.

  \paragraph{Projection and emergent time.}
    The parameter commonly interpreted as time in the effective description is not a
    fundamental attribute of the $\chi$ substrate.
    Instead, temporal ordering and duration arise only after projection, as part of the
    emergent spacetime representation associated with $\chi_{\mathrm{eff}}$.
    No external or fundamental time parameter is introduced at the level of $\chi$ itself.

    Accordingly, all averaging and coarse-graining operations involved in the definition
    of the background field $\bar{\chi}$ and in the construction of $\chi_{\mathrm{eff}}$
    are formulated relationally, without reference to an underlying temporal metric.
    Temporal concepts enter the framework only at the effective level, once a stable
    geometric regime has emerged through projection.

  \paragraph{Apparent fine-tuning.}
    In Cosmochrony, apparent fine-tuning does not reflect an improbable choice of initial
    conditions or a delicate adjustment of fundamental constants.
    It arises from the fact that only a restricted class of $\chi$ configurations admits
    a coherent and stable physical projection.
    Most configurations of the $\chi$ substrate do not give rise to consistent spacetime
    descriptions, observable laws, or persistent physical structures.

    The apparent delicacy of physical parameters is therefore a selection effect imposed
    by the projection itself.
    Only those relational configurations compatible with a stable emergent geometry and
    with sustained relaxation dynamics appear as physically realized universes.
    Fine-tuning is thus reinterpreted as a structural constraint on projectability rather
    than as a coincidence requiring external explanation.

  \paragraph{Absence of a multiverse.}
    Cosmochrony does not postulate a multiverse.
    While multiple configurations of the $\chi$ substrate may correspond to the same
    physical universe under projection, the framework provides no mechanism by which a
    single $\chi$ structure could sustain multiple independent physical projections.

    The universe is therefore unique at the level of physical reality, even though its
    underlying description in terms of $\chi$ may be non-unique.
    The absence of a multiverse is not an additional assumption, but a direct consequence
    of the non-injective and ontologically asymmetric character of the projection.
