\subsection{Projection, Reality, and Ontological Asymmetry}
  \label{subsec:projection-reality-and-ontological-asymmetry}

  Within this ontology, the emergence of spacetime should be understood as a projection
  from the $\chi$ substrate, rather than as a dual or equivalent description.
  The projected universe is fully real at the physical level, but its reality is
  derivative: spacetime entities, effective fields, and dynamical laws do not possess
  ontological primacy~\cite{Rovelli2021}.

  This asymmetry is essential.
  While physical descriptions depend on the projection of $\chi$, the converse is not
  true.
  The $\chi$ structure exists independently of spacetime notions and does not admit a
  reformulation entirely in geometric or dynamical terms.
  Projection must therefore be understood as non-injective: distinct structural
  features of $\chi$ may correspond to identical or physically indistinguishable
  spacetime configurations.

  \paragraph{Apparent fine-tuning.}
    In Cosmochrony, apparent fine-tuning does not reflect an improbable choice of initial
    conditions or a delicate adjustment of fundamental constants.
    It arises from the fact that only a restricted class of $\chi$ configurations admits
    a coherent and stable physical projection.
    Most configurations of the $\chi$ substrate do not give rise to consistent spacetime
    descriptions, observable laws, or persistent physical structures.

  \paragraph{Absence of a multiverse.}
    Cosmochrony does not postulate a multiverse.
    While multiple configurations of the $\chi$ substrate may correspond to the same
    physical universe under projection, the framework provides no mechanism by which a
    single $\chi$ structure could sustain multiple independent physical projections.
    The universe is therefore unique at the level of physical reality, even though its
    underlying description in terms of $\chi$ may be non-unique.
