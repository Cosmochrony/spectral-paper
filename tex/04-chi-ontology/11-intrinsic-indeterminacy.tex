\subsection{Intrinsic Indeterminacy of the $\chi$ Field}
\label{subsec:intrinsic-indeterminacy}

Although the term \emph{fluctuations} is sometimes used for intuitive purposes,
it should not be understood in a dynamical or stochastic sense.
At the fundamental level, the $\chi$ field does not undergo random temporal
variations, nor does it implement any underlying probabilistic process.
In particular, $\chi$ is not subject to noise, thermal agitation, or hidden
stochastic dynamics.

Instead, $\chi$ exhibits an \emph{intrinsic structural indeterminacy}:
certain relational degrees of freedom are not fully constrained.
This indeterminacy is ontological rather than epistemic.
It does not reflect incomplete knowledge, nor the presence of hidden variables,
but a genuine underdetermination of the fundamental relational structure.

As a consequence, the probabilistic character of quantum phenomena does not
originate from randomness within $\chi$ itself.
It emerges at the level of projection, when an underconstrained configuration
of $\chi$ admits multiple spacetime realizations compatible with the same
underlying structure.
The effective probabilities associated with different outcomes reflect the
relative structural measure of the corresponding sets of compatible
$\chi$-configurations.

Importantly, no complete initial condition or primordial ``seed'' encodes the
future evolution of the system.
Compatibility relations are not fixed in advance but emerge progressively as
successive projections resolve previously unconstrained degrees of freedom.
This process introduces irreversible structural constraints without implying
determinism.

Throughout this work, the term ``fluctuation'' is therefore used only as a
heuristic shorthand for structural underconstraint
