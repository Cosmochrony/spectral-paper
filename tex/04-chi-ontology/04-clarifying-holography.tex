\subsection{Clarifying the Relation to Holographic Descriptions}
\label{subsec:clarifying-holography}

The preceding considerations naturally invite comparison with the holographic
principle, as originally proposed by ’t~Hooft and Susskind, which suggests that the
effective information content of a spacetime region scales with its boundary rather
than its volume.

Cosmochrony is \emph{not} a holographic theory in the technical sense.
It does not posit a lower-dimensional boundary description, nor a dual equivalence
between bulk and boundary physics.
Any holographic-like behavior arises here as a consequence of projection from a
non-factorizable, pre-geometric substrate, rather than from a fundamental encoding
principle.

In particular, the limitation of physically accessible information within a
spacetime region reflects the degeneracy of $\chi$ configurations compatible with a
given projection.
This constraint is intrinsic to the emergence of spacetime itself and does not
require the introduction of boundary degrees of freedom or dimensional reduction.

Thus, while Cosmochrony reproduces certain qualitative features commonly associated
with holographic descriptions, it differs fundamentally in its ontological grounding.
Holography appears here as an emergent signature of projection, not as a guiding
principle or foundational postulate.
