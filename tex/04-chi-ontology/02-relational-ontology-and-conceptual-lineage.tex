\subsection{Relational Ontology and Conceptual Lineage}
  \label{subsec:relational-ontology-and-conceptual-lineage}

  The relational character of the $\chi$ substrate bears a conceptual affinity with
  relational approaches in physics, notably those advocated by Rovelli~\cite{Rovelli1996,Rovelli2004}.
  These approaches trace their philosophical roots to Aristotelian relational ontology,
  in which properties are defined through relations rather than as intrinsic
  attributes~\cite{AristotleCategories,Shields2016}.

  Cosmochrony shares this rejection of intrinsic, observer-independent properties.
  However, it extends relationalism to a deeper ontological level.
  In relational quantum mechanics, relations are primary at the level of quantum states
  describing interactions between systems that are themselves taken as given.
  By contrast, Cosmochrony posits that the fundamental substrate $\chi$ is itself
  relational: there are no underlying entities between which relations are defined.

  In this sense, $\chi$ configurations are not relations \emph{between} pre-existing
  objects, but relational structures that constitute the objects themselves once a
  projection into an effective spacetime description occurs.
  Relationality is therefore not a feature of physical states within spacetime, but an
  intrinsic property of the pre-geometric substrate from which spacetime and physical
  entities emerge.

  This distinction is essential for understanding the ontological asymmetry between
  $\chi$ and its spacetime projections.
  While relational quantum mechanics reformulates quantum theory without modifying
  the ontological status of spacetime itself, Cosmochrony relocates relationality at
  the level of the substrate that gives rise to spacetime, time ordering, and physical
  objects.

  \paragraph{On the absence of fundamental values.}
    In Cosmochrony, relations are not defined between pre-existing fundamental values.
    The relational structure of $\chi$ is ontologically primary and admits no intrinsic
    numerical or field-like values.
    What appear, within effective spacetime descriptions, as scalar values, entities, or
    local degrees of freedom arise only as stable invariants of this relational structure
    under projection and coarse-graining.
