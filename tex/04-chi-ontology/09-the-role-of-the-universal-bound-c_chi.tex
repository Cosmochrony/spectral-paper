\subsection{The Role of the Universal Bound $c_\chi$}
  \label{subsec:role-of-cchi}

  A central structural element of the Cosmochrony framework is the existence of a
  universal invariant bound, denoted $c_\chi$, defined at the level of the pre-temporal
  $\chi$ substrate.
  This bound does not correspond to a signal velocity, nor to the propagation of any
  field or excitation in spacetime.
  Instead, $c_\chi$ characterizes an absolute structural limit on the degree to which
  relational information can be locally constrained within admissible $\chi$ configurations.

  At the fundamental level, $c_\chi$ is non-metric and non-temporal.
  It is not associated with distances, durations, or causal cones, since none of these
  notions are defined prior to projection.
  Rather, $c_\chi$ expresses a maximal admissible rate of structural ordering, beyond
  which further confinement of relational degrees of freedom becomes impossible and
  relaxation is unavoidable.

  Crucially, $c_\chi$ is not itself an observable quantity.
  It acquires operational meaning only through projection, when $\chi$ configurations
  admit a locally injective representation in terms of effective spacetime variables.
  In such projectable regimes, the invariant structural bound $c_\chi$ manifests as an
  effective causal constraint $c$, defined through the maximal admissible local ordering
  rate of projected configurations:
  \[
    c \;\equiv\; \Pi(c_\chi),
  \]
  where $\Pi$ denotes the projection from the pre-geometric relational substrate to an
  effective spacetime description.

  The constant $c$, which coincides numerically with the observed speed of light, therefore
  has a derivative status.
  It does not represent an independent postulate, but the spacetime expression of the
  deeper structural bound $c_\chi$ once notions of locality, duration, and causal ordering
  become meaningful.
  All effective causal constraints appearing in projected descriptions are ultimately
  inherited from this underlying invariant bound.

  Accordingly, the local constraint on the effective relaxation functional,
  \[
    |D_{\mathrm{loc}}\chi_{\mathrm{eff}}| \le c,
  \]
  should be understood as the projected form of the more fundamental structural limit
  imposed by $c_\chi$.
  No causal or dynamical principle is imposed directly at the level of spacetime;
  effective causality arises solely as a consequence of the bounded projective
  realization of relational ordering.

  This distinction becomes essential in strong-gravity or near-deprojection regimes.
  While the effective constant $c$ may lose its geometric or causal interpretation when
  projection ceases to be injective, the structural bound $c_\chi$ remains invariant.
  The breakdown of spacetime description therefore signals not a violation of causality,
  but the loss of representability of relational configurations within a spacetime
  framework.

  In summary, $c_\chi$ defines a universal, pre-temporal structural bound governing the
  admissibility of $\chi$ configurations, while $c$ represents its effective spacetime
  manifestation.
  The latter inherits its value and universality entirely from the former, ensuring
  conceptual continuity between the pre-geometric substrate and emergent relativistic
  causality.
