\subsection{Quarks as Non-Projectable Internal Modes}
\label{subsec:quarks-non-projectable-modes}

The ontological status of quarks provides a clarifying example of the distinction
between the pre-geometric $\chi$ substrate and its effective spacetime projections.

In Cosmochrony, quarks are not interpreted as fundamental particle-like entities,
nor as independent localized excitations of $\chi$.
Rather, they correspond to internal structural modes of composite solitonic
configurations—modes that are necessary to characterize the internal organization
and stability of hadronic excitations, but which do not admit an autonomous and
coherent projection into spacetime.

In effective quantum field descriptions, quarks appear as elementary degrees of
freedom subject to confinement.
Within Cosmochrony, this confinement is not imposed dynamically by an external
interaction, but reflects a deeper structural constraint: isolated quark-like modes
do not correspond to admissible standalone projections of $\chi_{\mathrm{eff}}$.
Only collective configurations in which these internal modes are topologically
and relationally closed admit a stable spacetime manifestation.

In this sense, quarks are real at the structural level of $\chi$, but incomplete at
the level of physical projection.
They are neither fictitious nor fundamental objects, but non-factorizable internal
components of projected excitations.
Their observability is therefore necessarily indirect, encoded in the spectral,
dynamical, and symmetry properties of hadrons rather than in localized detection
events.

This interpretation parallels the status of internal degrees of freedom in other
collective systems: they are indispensable for an accurate effective description,
yet do not correspond to independently realizable physical entities.
Quark confinement thus appears not as a contingent feature of strong interactions,
but as a direct consequence of the non-injective nature of the projection from $\chi$
to emergent spacetime.
