\subsection{Spectral Admissibility and Regularity}
  \label{subsec:spectral-admissibility}

  The emergence of an effective continuum geometry from the relational spectral data
  requires additional regularity conditions.
  Not all relational configurations admit a meaningful continuum approximation,
  even when a spectral distance can be defined.
  In this subsection, we introduce the spectral admissibility criteria that characterize
  the regimes in which the geometric reconstruction is well-defined.

  Let $L$ denote a self-adjoint relational operator acting on a suitable Hilbert space for the configurations.
  In discrete realizations, $L$ is reduced to a graph Laplacian associated with the
  relational connectivity of the system.
  In general, $L$ encodes a relational structure without reference to the background manifold or metric.

  The operator $L$ admits a spectral decomposition
  \begin{equation}
    L \psi_n = \lambda_n \psi_n ,
  \end{equation}
  with non-negative eigenvalues $\{\lambda_n\}$ and the corresponding eigenmodes $\{\psi_n\}$.
  No geometric interpretation was assumed at this stage.

  Spectral admissibility is defined by restricting the attention to a controlled spectral
  window.

  Specifically, we introduce a smooth spectral filter
  \begin{equation}
    F_{\lambda_*} \;=\; f\!\left(\frac{L}{\lambda_*}\right),
  \end{equation}
  where $f(x)$ is a fixed cutoff function and $\lambda_*$ sets a characteristic spectral scale.
  Only modes below this scale contribute significantly to the effective geometric reconstruction.

  This filtering procedure defines the admissibility purely in spectral terms.
  It does not rely the locality, coordinates, or spatiotemporal integration measures.
  Instead, it reflects that continuum geometry, when it emerges, is
  necessarily insensitive to fine-grained spectral details beyond a given resolution.

  Spectral admissibility is generally non-injective.
  Distinct relational configurations may share identical spectral content within the
  admissible window and therefore give rise to the same effective continuum geometry.
  Conversely, a given relational structure can admit multiple equivalent continuum embeddings.
  This non-uniqueness is a structural feature of spectral reconstruction and does not
  indicate inconsistency in the framework.
