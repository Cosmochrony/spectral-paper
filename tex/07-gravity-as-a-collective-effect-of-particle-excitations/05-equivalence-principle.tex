\subsection{Equivalence Principle}
  \label{subsec:equivalence-principle}

  Within the Cosmochrony framework, particle-like excitations do not couple to a
  fundamental gravitational field.
  They are described instead, at the level of effective descriptions, as localized
  projected configurations that impose structural constraints on admissible
  relaxation ordering.

  Crucially, all admissible localized projected configurations constrain relaxation
  ordering in the same universal manner.
  Their internal composition, detailed structure, or microscopic realization plays no
  role in how they affect or respond to the admissible ordering environment.
  The collective reduction of effective relaxation ordering therefore depends only on
  the presence of localized structural constraints, not on their specific nature.

  As a consequence, all particle-like projected configurations respond identically to a
  given effective ordering environment.
  When expressed in an effective geometric language, this universal response appears
  as composition-independent acceleration in a gravitational field.

  The equivalence between inertial and gravitational behavior thus emerges as a direct
  structural necessity.
  Inertial resistance and gravitational response are not distinct physical properties,
  but two effective manifestations of the same underlying constraint on admissible
  relaxation ordering.

  In this sense, the equivalence principle is not an independent postulate within
  Cosmochrony.
  It arises inevitably as an emergent symmetry of admissible projected descriptions,
  reflecting the absence of any fundamental distinction between inertial and
  gravitational mass at the effective level.
