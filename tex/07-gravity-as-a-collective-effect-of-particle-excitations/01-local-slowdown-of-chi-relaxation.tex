\subsection{Local Slowdown of Relaxation Ordering}
  \label{subsec:local-slowdown-of-chi-relaxation}

  Within the Cosmochrony framework, gravitation does not arise from a fundamental
  interaction or from an independent dynamical field.
  It emerges, at the level of effective descriptions, from the collective influence
  of localized projected configurations on admissible relaxation ordering.

  As established in
  Sec.~\ref{sec:particles-as-localized-excitations-of-the-chi-field},
  particle-like projected configurations correspond to regions of enhanced resistance
  to admissible relaxation.
  When such configurations are present in significant number, their combined effect
  leads, in the weak-constraint regime, to a macroscopic reduction of the admissible
  ordering rate within projected descriptions.

  In an effective spacetime parametrization, this collective effect may be expressed
  schematically as
  \begin{equation}
    \mathcal{D}_{\mathrm{eff}} \chi_{\mathrm{eff}}
    \;\simeq\;
    c \left( 1 - \alpha\, \rho \right),
  \end{equation}
  where $\rho$ denotes the effective density of localized projected configurations and
  $\alpha$ encodes their average contribution to relaxation resistance.
  This expression represents a first-order approximation, valid when localized
  constraints are sufficiently dilute and weakly overlapping.

  The coupling parameter $\alpha$ is not fundamental.
  It emerges as a collective property of admissible projected descriptions and depends
  on the typical structural characteristics of localized configurations.
  In the weak-field limit, dimensional consistency relates its scaling to the observed
  gravitational constant, leading to $\alpha \propto G / c^2$ when expressed in terms
  of effective inertial mass densities.
  Within this approximation, the reduction of admissible relaxation ordering admits a
  Newtonian-like interpretation in terms of an effective gravitational potential.

  Physically, this collective slowdown manifests, within effective geometric
  descriptions, as gravitational time dilation and curvature effects.
  No independent gravitational force or mediator is introduced.
  Gravitation appears instead as a macroscopic signature of constrained admissible
  relaxation ordering induced by the presence of localized projected configurations.
