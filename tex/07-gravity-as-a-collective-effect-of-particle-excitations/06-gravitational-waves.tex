\subsection{Gravitational Waves}
  \label{subsec:gravitational-waves}

  Time-dependent variations in the distribution of localized projected configurations,
  such as accelerating masses or mergers of compact systems, induce collective and
  transient modulations of admissible relaxation ordering.
  These modulations propagate through the projected description as changes in the
  effective ordering regime and are transmitted at the maximal admissible ordering
  speed~$c$.

  When expressed in an effective spacetime language, such propagating modulations are
  described as gravitational waves.
  Unlike electromagnetic radiation, which corresponds to propagating particle-like
  projected excitations, gravitational waves represent collective variations of the
  admissible ordering and correlation structure of projected descriptions themselves.

  In this framework, gravitational waves do not introduce additional fundamental
  degrees of freedom.
  They arise as macroscopic, collective responses of admissible projected descriptions
  to time-dependent reconfigurations of localized relaxation-resistant structures,
  rather than as excitations of an independent underlying field.

  This interpretation preserves the phenomenology of general relativity in regimes
  where a smooth spacetime description applies.
  The standard properties of gravitational waves—propagation at speed~$c$,
  transverse polarization, and energy transport—are recovered as effective features
  of collective ordering dynamics.

  It should be emphasized, however, that gravitational-wave descriptions remain valid
  only within regimes where projection onto an effective spacetime remains well
  defined.
  In strong-gravity environments approaching the deprojection threshold discussed in
  Section~\ref{subsec:black-hole-deprojection-cycle}, such collective modulations are
  expected to become increasingly attenuated or to lose a clear spacetime
  interpretation altogether.
