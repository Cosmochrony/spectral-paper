\subsection{Collective Gravitational Coupling and Operational Geometry}
  \label{subsec:collective-gravitational-coupling-and-operational-geometry}

  The collective reduction of admissible relaxation ordering described above affects
  not only the local accumulation of effective time, but also the manner in which
  variations within projected descriptions influence one another across extended
  regions.
  In the presence of localized projected configurations, the relaxation resistance
  they induce modulates how efficiently structural variations can be correlated
  between different effective locations.

  At the level of effective descriptions, this collective behavior may be summarized
  by a constitutive coupling function characterizing the stiffness of projected
  configurations with respect to relative variations.
  In regions where projected descriptions are nearly homogeneous, this coupling
  approaches a uniform effective value.
  Localized projected configurations weaken it by introducing additional structural
  constraints.
  Crucially, this coupling is defined entirely within the effective descriptive
  framework and does not presuppose any fundamental spatial metric, background geometry,
  or pre-existing notion of distance.

  Because no fundamental geometry is assumed, spatial separation is defined
  operationally.
  Two effective regions are considered close if structural variations of projected
  configurations can be efficiently correlated between them, and distant otherwise.
  In the continuum and weak-constraint regime, this operational notion admits a compact
  description in terms of an effective spatial metric, which summarizes the collective
  response of admissible projected descriptions to relative variations.

  Within this framework, spacetime curvature does not arise as a primitive geometric
  property or as the effect of a fundamental gravitational field.
  It emerges instead as a descriptive manifestation of how localized projected
  configurations modulate the collective admissible ordering and correlation structure.
  Geometry therefore functions as a macroscopic encoding of constrained relational
  organization, rather than as an independent ontological entity.

  A more explicit relational construction of the coupling mechanism, and its connection
  to discrete formulations of admissible projected descriptions, is presented in
  Appendix~\ref{subsec:collective-coupling}.
