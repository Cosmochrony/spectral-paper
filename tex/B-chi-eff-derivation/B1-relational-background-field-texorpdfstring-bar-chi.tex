\subsection{Relational background field \texorpdfstring{$\bar{\chi}$}{χ̄}.}
  \label{subsec:relational-background-field-texorpdfstring-bar-chi}
  We define a background field $\bar{\chi}$ using a \textbf{relational average} that uses only the
  \textbf{combinatorial (pre-geometric) distance} $d_{ij}^{C}$. Let
  \begin{equation}
    N_i \;=\; \left\{ j \,\middle|\, d_{ij}^{C} \le \ell_0 \right\}
    \label{eq:combinatorial_neighborhood}
  \end{equation}
  be the combinatorial neighborhood of radius $\ell_0$ around node $i$. We then set
  \begin{equation}
    \bar{\chi}_i
    \;=\;
    \frac{1}{|N_i|}
    \sum_{j\in N_i} \chi_j.
    \label{eq:relational_average_bar_chi}
  \end{equation}
  Because $N_i$ depends only on $d_{ij}^{C}$, the definition of $\bar{\chi}$ is \textbf{independent
of any weighted metric} and therefore does not depend on $\chi$ through a distance function.
  This is a crucial step in the preventions of circularity.

  The scale $\ell_0$ is an auxiliary coarse-graining parameter defining the minimal
  relational neighborhood required for stability of the effective description; it does
  not correspond to a physical length scale prior to the emergence of geometry.
