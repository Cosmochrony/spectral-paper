\section{Derivation of \texorpdfstring{$\chi_{\mathrm{eff}}$}{χeff} from Relational Observables}
  \label{app:chi-eff-derivation}

  The effective field $\chi_{\mathrm{eff}}$ is introduced as an operational description of
  $\chi$-configurations once a stable projected regime exists. Since the projected regime
  admits an effective geometric interpretation, $\chi_{\mathrm{eff}}$ must be constructed in a
  way that does not implicitly assume the metric structure it is meant to support.
  In particular, if the distance $d_{ij}$ used to define coarse-graining neighborhoods depends
  on the weights that themselves depend on $\chi$, then a hidden circularity would arise.

  To remove this ambiguity, we adopted a \textbf{two-level construction} based on the explicit
  distinction between the combinatorial distance $d_{ij}^{C}$ and the weighted distance
  $d_{ij}^{W}$ introduced in Appendix~\ref{subsec:combinatorial-vs.-weighted-distance}.

  \subsection{Relational background field \texorpdfstring{$\bar{\chi}$}{χ̄}.}
  \label{subsec:relational-background-field-texorpdfstring-bar-chi}
  We define a background field $\bar{\chi}$ using a \textbf{relational average} that uses only the
  \textbf{combinatorial (pre-geometric) distance} $d_{ij}^{C}$. Let
  \begin{equation}
    N_i \;=\; \left\{ j \,\middle|\, d_{ij}^{C} \le \ell_0 \right\}
    \label{eq:combinatorial_neighborhood}
  \end{equation}
  be the combinatorial neighborhood of radius $\ell_0$ around node $i$. We then set
  \begin{equation}
    \bar{\chi}_i
    \;=\;
    \frac{1}{|N_i|}
    \sum_{j\in N_i} \chi_j.
    \label{eq:relational_average_bar_chi}
  \end{equation}
  Because $N_i$ depends only on $d_{ij}^{C}$, the definition of $\bar{\chi}$ is \textbf{independent
of any weighted metric} and therefore does not depend on $\chi$ through a distance function.
  This is a crucial step in the preventions of circularity.

  The scale $\ell_0$ is an auxiliary coarse-graining parameter defining the minimal
  relational neighborhood required for stability of the effective description; it does
  not correspond to a physical length scale prior to the emergence of geometry.

  \subsection{Emergent connectivity and weighted distance.}
  \label{subsec:emergent-connectivity-and-weighted-distance}
  Using the background field $\bar{\chi}$, we define the link weights and the corresponding
  connectivity using Eq.~\eqref{eq:weight_function_background}:
  \[
    w_{uv}(\bar{\chi})
    \;=\;
    \frac{1}{K_0}
    \left[
      1 + \left(\frac{\bar{\chi}_u-\bar{\chi}_v}{\chi_c}\right)^2
    \right],
    \qquad
    K_{uv}(\bar{\chi}) \;=\; \frac{1}{w_{uv}(\bar{\chi})}.
  \]
  The \textbf{weighted distance} used for effective geometry is then
  \begin{equation}
    d_{ij}^{W}
    \;=\;
    \min_{\gamma_{ij}}
    \sum_{(u,v)\in\gamma_{ij}} w_{uv}(\bar{\chi}),
    \label{eq:weighted_distance}
  \end{equation}
  which depends on $\bar{\chi}$ but not on instantaneous $\chi$ values through the metric definition.

  \subsection{Effective field \texorpdfstring{$\chi_{\mathrm{eff}}$}{χeff} (geometry-aware coarse-graining).}
  \label{subsec:effective-field-chi_eff-geometry-aware-coarse-graining}
  Finally, we define $\chi_{\mathrm{eff}}$ by coarse-graining $\chi$ over neighborhoods defined
  by the \textbf{weighted distance} $d_{ij}^{W}$:
  \begin{equation}
    V_{\ell_0}(i)
    \;=\;
    \left\{ j \,\middle|\, d_{ij}^{W} \le \ell_0 \right\},
    \label{eq:weighted_neighborhood}
  \end{equation}
  \begin{equation}
    \chi_{\mathrm{eff}}(i)
    \;=\;
    \frac{1}{|V_{\ell_0}(i)|}
    \sum_{j\in V_{\ell_0}(i)} \chi_j.
    \label{eq:chi_eff_definition}
  \end{equation}

  \subsection{Non-circular dependency structure.}
  \label{subsec:non-circular-dependency-structure}
  The construction is explicitly hierarchical:
  \[
    d^{C} \;\Longrightarrow\; \bar{\chi}
    \;\Longrightarrow\; w(\bar{\chi}),\,K(\bar{\chi})
    \;\Longrightarrow\; d^{W}
    \;\Longrightarrow\; \chi_{\mathrm{eff}}.
  \]
  The neighborhood used to compute $\bar{\chi}$ is defined using $d^{C}$ and is therefore $\chi$-independent.
  The effective geometry is encoded in $d^{W}$ through weights that depend only on $\bar{\chi}$, thereby breaking any
  instantaneous feedback loop.
  This makes the operational definition of $\chi_{\mathrm{eff}}$ compatible with the pre-geometric status of $\chi$,
  while still allowing an emergent geometric regime for spectral and effective-field analyses.

  No dynamical equation for $\chi$ or $\chi_{\mathrm{eff}}$ is assumed in this construction;
  the procedure is purely kinematic and defines the conditions under which an effective
  geometric regime becomes admissible.

