\subsection{Derivation of \texorpdfstring{$\chi_{\mathrm{eff}}$}{χeff} from Relational Observables}
  \label{app:chi-eff-derivation}

  The effective field $\chi_{\mathrm{eff}}$ is introduced as an operational description of
  $\chi$-configurations once a stable projected regime exists. Since the projected regime
  admits an effective geometric interpretation, $\chi_{\mathrm{eff}}$ must be constructed in a
  way that does not implicitly assume the very metric structure it is meant to support.
  In particular, if a distance $d_{ij}$ used to define coarse-graining neighborhoods depends
  on weights that themselves depend on $\chi$, then a hidden circularity would arise.

  To remove this ambiguity, we adopt a \textbf{two-level construction} based on the explicit
  distinction between the combinatorial distance $d_{ij}^{C}$ and the weighted distance
  $d_{ij}^{W}$ introduced in Appendix~E.4.

  \paragraph{(1) Relational background field \texorpdfstring{$\bar{\chi}$}{χ̄}.}
    We define a background field $\bar{\chi}$ by a \textbf{relational average} that uses only the
    \textbf{combinatorial (pre-geometric) distance} $d_{ij}^{C}$. Let
    \begin{equation}
      N_i \;=\; \left\{ j \,\middle|\, d_{ij}^{C} \le \ell_0 \right\}
      \label{eq:combinatorial_neighborhood}
    \end{equation}
    be the combinatorial neighborhood of radius $\ell_0$ around node $i$. We then set
    \begin{equation}
      \bar{\chi}_i
      \;=\;
      \frac{1}{|N_i|}
      \sum_{j\in N_i} \chi_j.
      \label{eq:relational_average_bar_chi}
    \end{equation}
    Because $N_i$ depends only on $d_{ij}^{C}$, the definition of $\bar{\chi}$ is \textbf{independent
of any weighted metric} and therefore does not depend on $\chi$ through a distance functional.
    This is the crucial step that prevents circularity.

  \paragraph{(2) Emergent connectivity and weighted distance.}
    Using the background field $\bar{\chi}$, we define the link weights and the corresponding
    connectivity through Eq.~\eqref{eq:weight_function_background}:
    \[
      w_{uv}(\bar{\chi})
      \;=\;
      \frac{1}{K_0}
      \left[
        1 + \left(\frac{\bar{\chi}_u-\bar{\chi}_v}{\chi_c}\right)^2
      \right],
      \qquad
      K_{uv}(\bar{\chi}) \;=\; \frac{1}{w_{uv}(\bar{\chi})}.
    \]
    The \textbf{weighted distance} used for effective geometry is then
    \begin{equation}
      d_{ij}^{W}
      \;=\;
      \min_{\gamma_{ij}}
      \sum_{(u,v)\in\gamma_{ij}} w_{uv}(\bar{\chi}),
      \label{eq:weighted_distance}
    \end{equation}
    which depends on $\bar{\chi}$ but not on instantaneous $\chi$ values through the metric definition.

  \paragraph{(3) Effective field \texorpdfstring{$\chi_{\mathrm{eff}}$}{χeff} (geometry-aware coarse-graining).}
    Finally, we define $\chi_{\mathrm{eff}}$ by coarse-graining $\chi$ over neighborhoods defined
    with the \textbf{weighted distance} $d_{ij}^{W}$:
    \begin{equation}
      V_{\ell_0}(i)
      \;=\;
      \left\{ j \,\middle|\, d_{ij}^{W} \le \ell_0 \right\},
      \label{eq:weighted_neighborhood}
    \end{equation}
    \begin{equation}
      \chi_{\mathrm{eff}}(i)
      \;=\;
      \frac{1}{|V_{\ell_0}(i)|}
      \sum_{j\in V_{\ell_0}(i)} \chi_j.
      \label{eq:chi_eff_definition}
    \end{equation}

  \paragraph{Non-circular dependency structure.}
    The construction is explicitly hierarchical:
    \[
      d^{C} \;\Longrightarrow\; \bar{\chi}
      \;\Longrightarrow\; w(\bar{\chi}),\,K(\bar{\chi})
      \;\Longrightarrow\; d^{W}
      \;\Longrightarrow\; \chi_{\mathrm{eff}}.
    \]
    The neighborhood used to compute $\bar{\chi}$ is defined using $d^{C}$ and is therefore
    $\chi$-independent. The effective geometry is encoded in $d^{W}$ through weights that depend
    only on $\bar{\chi}$, breaking any instantaneous feedback loop. This makes the operational
    definition of $\chi_{\mathrm{eff}}$ compatible with the pre-geometric status of $\chi$, while still
    allowing an emergent geometric regime for spectral and effective-field analyses.
