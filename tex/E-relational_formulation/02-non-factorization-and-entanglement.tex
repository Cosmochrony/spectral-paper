\subsection{Non-Factorization and Entanglement}
  \label{subsec:non-factorization-entanglement}

  This appendix provides a formal and ontological clarification of the relational
  mechanism underlying quantum entanglement, as introduced at a phenomenological
  level in Section~\ref{subsec:entanglement-nonlocality}.
  Its purpose is not to restate the physical interpretation given there, but to make explicit the non-factorization
  properties of the $\chi$ substrate from which those effective behaviors arise.

  Within the relational formulation of Cosmochrony, configurations of the $\chi$ field
  do not generically decompose into independent subsystems.
  Factorization, understood as a decomposition preserving internal relaxation structure while isolating disjoint
  subsets of relations, is therefore not fundamental.
  It emerges only in restricted regimes where relational couplings are weak, hierarchically organized, or dynamically
  suppressed.

  A relational configuration is said to be \emph{non-factorizable} when no decomposition
  exists that preserves its internal relaxation structure while yielding independent
  subconfigurations.
  In such cases, what appear as multiple subsystems at the effective geometric level correspond, at the relational
  level, to a single indivisible configuration.

  Quantum entanglement arises as a direct manifestation of persistent non-factorization.
  When a non-factorizable relational configuration admits an effective projection onto
  spatially separated degrees of freedom, its components may become geometrically
  distant while remaining relationally inseparable.
  Measurement projections acting on one effective subsystem therefore constrain the set of admissible projections
  associated with others, independently of their spatial separation.

  These constraints do not arise from dynamical updates propagating through spacetime.
  They reflect the incompatibility of certain relational patterns with the selected
  projection and are fully determined by global relational consistency conditions.

  Importantly, non-factorization should not be conflated with dynamical nonlocality.
  As emphasized in Section~9.6, all dynamical evolution of $\chi$ remains governed by
  bounded relaxation constraints that respect the invariant speed $c$ once an effective
  causal structure emerges.
  Entanglement correlations therefore do not enable superluminal signaling and do not violate relativistic causality.

  This appendix thus provides the ontological underpinning of the entanglement
  phenomenology discussed in Section~9.6, framing quantum correlations as structural
  properties of the relational configuration space of $\chi$, rather than as consequences
  of nonlocal dynamics or measurement-induced collapse.
