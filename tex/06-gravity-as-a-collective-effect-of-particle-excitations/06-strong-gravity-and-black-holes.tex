\subsection{Strong Gravity and Black Holes}
  \label{subsec:strong-gravity-and-black-holes}

  In regions where the density of localized excitations becomes sufficiently high,
  the relaxation dynamics of the $\chi$ field may become extremely constrained.
  In effective spacetime descriptions, this corresponds to a regime in which the
  local relaxation rate is strongly suppressed relative to distant observers,
  defining an effective horizon.

  Within Cosmochrony, such regions are interpreted as black holes.
  Rather than being characterized by a fundamental spacetime singularity, black
  holes correspond to domains where the unfolding of physical processes becomes
  asymptotically inaccessible from the exterior due to the collective inhibition
  of $\chi$ relaxation.
  This naturally accounts for extreme time dilation effects without requiring
  divergent curvature invariants.

  \subsubsection{Gravitational and Temporal Shadows}

    In the strong-gravity regime, the increasing concentration of excitations induces
    large structural constraints in the $\chi$ field.
    As a result, the effective rate at which $\chi$ relaxes relative to external
    parametrizations is progressively reduced, approaching an asymptotic freeze-out
    in effective geometric descriptions.

    This behavior reproduces the phenomenon commonly referred to as a
    \emph{gravitational shadow}.
    In general relativity, such shadows arise from the absence of escaping null
    geodesics within a characteristic angular region.
    In Cosmochrony, an equivalent observational signature emerges because propagating
    excitations of the $\chi$ field, including radiation-like modes, cannot be
    sustained in regions where the relaxation dynamics is effectively frozen.
    External observers therefore perceive a dark angular region corresponding to the
    projection of this dynamically inaccessible domain.

    Beyond this optical effect, the framework predicts a deeper phenomenon, which may
    be termed a \emph{temporal shadow}.
    As the local relaxation of $\chi$ becomes increasingly inhibited, the effective
    progression of time within the region slows asymptotically with respect to the
    external environment.
    From the external perspective, internal processes appear indefinitely delayed,
    providing a natural interpretation of horizon-induced time dilation.

    In this view, the observed gravitational shadow corresponds to the visible
    manifestation of an underlying temporal shadow.
    Both effects arise from the same collective relaxation dynamics of the $\chi$
    field and need not be attributed to a fundamental spacetime singularity or to
    divergent tensorial curvature.
