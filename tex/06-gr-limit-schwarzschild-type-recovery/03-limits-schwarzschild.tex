\subsection{Limits of Schwarzschild-Type Reconstruction}
  \label{subsec:limits-schwarzschild}

  The recovery of the Schwarzschild-type metric in the previous section illustrates
  how familiar gravitational geometries can emerge from relational spectral data
  under restrictive conditions.
  However, this reconstruction is intrinsically regime-dependent and should not be
  interpreted as being universally applicable.

  The derivation relies on the assumptions of approximate stationarity, spherical
  symmetry, and weak inhomogeneity.
  Outside these regimes, the effective geometric description may differ substantially
  from the Schwarzschild form or cease to be meaningful.

  The characteristic length scale appearing in the static spherically symmetric
  metric is an integration constant of the geometric reconstruction.
  Its identification using physical parameters depends on the choice of spectral
  filtering and on the operational resolution at which the geometry is probed.
  Consequently, Schwarzschild-type metrics should be understood as effective geometric
  representatives rather than fundamental solutions.

  Classical weak-field tests confirm the consistency of the geometric approximation within its validity domain.
  However, they do not constrain the behavior of the relational system outside this
  regime, where the assumptions underlying metric reconstruction no longer hold.

  These considerations highlight the importance of distinguishing between the
  existence and range of applicability of a geometric description.
  Schwarzschild geometry emerges where appropriate but does not define the behavior
  of the system in regimes where a continuum spacetime interpretation breaks down.

  This limitation reflects the fact that Schwarzschild geometry captures only the
  long-wavelength, weak-field imprint of a localized relational obstruction.
  Beyond this regime, the effective geometric description ceases to faithfully encode
  the underlying connectivity constraints of the substrate.
