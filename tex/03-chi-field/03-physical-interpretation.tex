\subsection{Physical Interpretation}
  \label{subsec:physical-interpretation}

  The central interpretative assumption of Cosmochrony is that spacetime is not a
  static background, but an effective macroscopic description arising from the
  continuous relaxation of the scalar quantity $\chi$.
  In regimes where a geometric interpretation becomes meaningful, variations of
  $\chi$ are reflected simultaneously in several emergent observables.

  Specifically, an increase of $\chi$ is encoded in:
  \begin{itemize}
    \item the accumulation of local proper time along physical processes,
    \item the emergence of effective spatial separation between correlated
    configurations,
    \item the large-scale expansion of the universe when the relaxation of $\chi$
    is considered globally.
  \end{itemize}

  Within this effective description, spatial distance and temporal duration are
  not independent primitives, but complementary aspects of the same underlying
  dynamics.
  Heuristically, distance may be viewed as accumulated (or ``frozen'') relaxation,
  while time corresponds to locally ongoing (or ``active'') relaxation.
  These expressions are intended as interpretative analogies rather than literal
  definitions, emphasizing that both quantities originate from the same scalar
  process.

  This unified interpretation is not imposed ad hoc, but follows from identifying
  temporal ordering and spatial relations as distinct coarse-grained manifestations
  of the relaxation dynamics of $\chi$.
