\subsection{Initial Conditions and Global Structure}
  \label{subsec:initial-conditions-and-global-structure}

  The Cosmochrony framework does not postulate initial conditions in the conventional
  temporal sense.
  Instead, it assumes that the relational substrate $\chi$ admits a minimal admissible
  ordering state, denoted $\chi_0$, corresponding to configurations of maximal effective
  relaxation density.
  This reference state does not represent an initial condition imposed at a distinguished
  moment in time, but a structural boundary of admissible projected descriptions,
  characterizing the earliest configurations that can meaningfully admit an effective
  ordering interpretation.

  In effective geometric regimes, the characteristic scale associated with projected
  descriptions in the vicinity of $\chi_0$ coincides numerically with the Planck scale.
  This correspondence reflects the loss of projectability and the breakdown of
  coarse-grained spacetime descriptions below this regime, rather than the presence of a
  fundamental cutoff, microscopic discreteness, or underlying spacetime lattice.

  From this perspective, cosmic history is interpreted as the progressive and irreversible
  ordering of projected $\chi$ configurations away from this minimal admissible state.
  No spacetime singularity is required at the fundamental level.
  Apparent singular behavior arises only when classical notions of time, distance, or
  curvature are extrapolated beyond the regime in which projected $\chi$ configurations
  admit a stable geometric and causal interpretation.

  The global structure of admissible projected descriptions is therefore constrained by
  the ordering properties of the underlying $\chi$ substrate, rather than by arbitrarily
  specified initial data or boundary conditions.
  In the following section, we derive a minimal effective dynamical equation governing
  the ordering of projected $\chi$ configurations and explore its immediate physical
  consequences.
