\subsection{Initial Conditions and Global Structure}
  \label{subsec:initial-conditions-and-global-structure}

  The Cosmochrony framework assumes the existence of a lower bound $\chi_0$ for the
  scalar quantity $\chi$, corresponding to a regime of maximal relaxation density.
  This bound characterizes the earliest physically meaningful configurations of the
  field and does not presuppose a fundamental temporal origin.

  In effective geometric descriptions, the scale associated with $\chi_0$ coincides
  numerically with the Planck scale.
  This identification reflects the breakdown of coarse-grained spacetime concepts
  below this regime, rather than the presence of a fundamental cutoff or discrete
  structure.

  Cosmic history is thus interpreted as the progressive global relaxation of $\chi$
  away from this minimal bound.
  No spacetime singularity is required in the fundamental description; apparent
  singular behavior arises only when classical notions of time and distance are
  extrapolated beyond the domain in which $\chi$ admits a stable geometric
  interpretation.

  In the next section, we derive a minimal dynamical equation governing the relaxation
  of $\chi$ and explore its immediate consequences.
