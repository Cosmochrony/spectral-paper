\subsection{Monotonicity and Arrow of Time}
  \label{subsec:monotonicity-and-arrow-of-time}

  A fundamental postulate of Cosmochrony is that the scalar quantity $\chi$ undergoes a
  monotonic relaxation process~\cite{Prigogine1997,Penrose1989Weyl}:
  \begin{equation}
    \mathcal{D}_{\lambda} \chi \ge 0 .
  \end{equation}

  This monotonicity is not derived from statistical considerations, nor imposed as a
  thermodynamic boundary condition.
  Rather, it reflects a structural property of the $\chi$ field associated with the
  irreversible character of its relaxation dynamics.
  Within Cosmochrony, energy is not treated as a fundamental conserved substance, but as
  a measure of the residual capacity of a given $\chi$ configuration to relax.
  Only relaxation processes can dissipate this capacity, whereas a decrease of $\chi$
  would correspond to a spontaneous reintroduction of contraction or tension into the
  field, for which no physical mechanism exists within the framework.

  Irreversibility therefore follows naturally.
  A decrease of $\chi$ would imply a contraction of both temporal ordering and spatial
  relations, effectively recreating relaxation capacity, which is dynamically forbidden.
  The arrow of time is thus identified with the irreversible expenditure of the
  relaxation capacity of the $\chi$ field, rather than with a statistical entropy
  principle imposed at the outset.
