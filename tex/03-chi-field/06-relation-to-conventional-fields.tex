\subsection{Relation to Conventional Fields}
  \label{subsec:relation-to-conventional-fields}

  Although $\chi$ may exhibit formal similarities with scalar fields employed in
  cosmology (such as inflaton-like fields) when expressed in effective spacetime
  descriptions, its ontological role is fundamentally different.
  The quantity $\chi$ is not a physical field propagating on spacetime, but a
  pre-geometric substrate from which spacetime notions themselves emerge.

  Accordingly, $\chi$ does not carry energy in the conventional field-theoretic
  sense, nor is it subject to quantization at the fundamental level.
  Quantization arises only at the effective level, where localized, stable
  excitations of $\chi$ admit a particle-like interpretation and can be described
  using standard quantum field-theoretic tools.

  Within this framework, matter, radiation, and interactions do not correspond to
  independent fields coupled to $\chi$.
  They emerge instead as localized excitations, constraints, or topological
  features of $\chi$ configurations, providing an effective description of familiar
  physical degrees of freedom in regimes where a spacetime interpretation applies.
