\subsection{Mass as Resistance to $\chi$ Relaxation}
  \label{subsec:mass-as-resistance-to-chi-relaxation}

  Within the Cosmochrony framework, mass is not an intrinsic attribute of matter
  but an emergent measure of the degree to which a localized excitation resists
  the relaxation of the $\chi$ field.
  Mass therefore characterizes the persistence of a non-relaxed configuration
  embedded within an otherwise globally relaxing substrate.

  Localized particle-like excitations introduce structural constraints in the
  configuration of $\chi$, which reduce the effective local rate of relaxation.
  When described using an effective spacetime parametrization, this reduction
  appears as a local slowing of the $\chi$ evolution, corresponding to time
  dilation in geometric terms.
  At larger scales, the cumulative effect of such localized constraints gives rise
  to what is interpreted as inertial and gravitational mass.

  From an energetic perspective, this resistance to relaxation corresponds to the
  storage of relaxation potential within the $\chi$ field.
  Maintaining a stable solitonic configuration requires continuously opposing the
  global relaxation flow, and the associated cost is what is operationally
  identified as energy.
  In this sense, mass quantifies the amount of relaxation potential that remains
  locally trapped within a persistent excitation.

  The quantitative mapping between solitonic energy and the masses of observed
  particles is discussed in Sec.~\ref{subsec:soliton_energy_mass}.
