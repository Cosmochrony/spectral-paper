\subsection{Collective Coupling and Operational Geometry}
  \label{subsec:collective-coupling-operational-geometry}

  The emergence of curvature in a relational framework can be understood as a
  collective effect.
  Rather than postulating a fundamental gravitational field, geometric properties
  arise from the way relational variations influence one another across extended
  regions of the system.

  In regimes where the relational structure is approximately homogeneous, variations
  propagate uniformly and admit a simple effective description.
  Localized irregularities in the relational connectivity modify this collective
  response, altering how variations can be correlated between neighboring regions.
  This modulation can be summarized by an effective coupling function characterizing
  the stiffness of the relational structure with respect to relative deformations.

  Importantly, this coupling is defined without reference to any background metric,
  coordinate system, or pre-existing notion of spatial separation.
  Distance is instead defined operationally: two regions are considered close if
  relational variations can be efficiently correlated between them, and distant
  otherwise.

  In the continuum and weakly inhomogeneous regime, this operational notion of
  proximity admits a compact geometric representation.
  An effective metric can then be introduced as a descriptive tool summarizing the
  collective response of the relational structure to local variations.
  The metric does not constitute an independent degree of freedom but encodes
  coarse-grained relational regularities.

  From this perspective, curvature is not a primitive geometric property nor the
  result of a dynamical field equation.
  It emerges as a macroscopic descriptor of how localized relational features
  modulate the collective connectivity of the system.
  Geometry thus functions as an effective encoding of constrained relational
  organization rather than as a fundamental ontological entity.

  \paragraph{Spectral convergence and continuum limit.}
    The emergence of a continuous geometric description relies on the spectral
    consistency of the relational Laplacian in appropriate large-scale regimes.
    Under mild regularity assumptions—uniform node density, bounded degree growth,
    and approximate isotropy of local connectivity—the graph Laplacian
    $\Delta_G$ converges, in the strong resolvent sense, toward the
    Laplace--Beltrami operator $\Delta_{\mathcal M}$ associated with an effective
    manifold $\mathcal M$.

    This convergence is understood in the operational sense relevant to the present
    framework: low-lying eigenmodes of $\Delta_G$ approximate those of
    $\Delta_{\mathcal M}$ up to a spectral cutoff $\lambda_*$, beyond which the
    notion of smooth geometry ceases to be meaningful.
    Such convergence results are well established in spectral graph theory and
    manifold learning, and justify interpreting the admissible spectral sector of
    the relational substrate as an effective continuum geometry.

  \paragraph{Spectral dimension as an emergent observable.}
    The effective dimensionality of the reconstructed geometry is not postulated
    but inferred from the spectral properties of the Laplacian.
    A standard operational probe is provided by the spectral heat kernel
    \begin{equation}
      K(t) = \sum_n e^{-t \lambda_n},
    \end{equation}
    from which the spectral dimension is defined as
    \begin{equation}
      d_s(t) = -2 \frac{d \log K(t)}{d \log t}.
    \end{equation}

    In regimes where the relational configuration is sufficiently smooth and
    admissible, $d_s(t)$ exhibits a stable plateau at large scales, indicating an
    effective manifold-like behavior.
    In particular, configurations relevant to the gravitational regime display a
    robust spectral dimension $d_s \simeq 4$, independently of microscopic graph
    details, providing an intrinsic justification for the emergence of a
    four-dimensional spacetime description.

  \paragraph{Physical interpretation of the spectral scale $\lambda_*$.}
    The spectral cutoff $\lambda_*$ plays a central physical role in the framework.
    It defines the upper limit of spectral admissibility beyond which relational
    modes can no longer be interpreted geometrically.
    Operationally, $\lambda_*$ functions as an ultraviolet cutoff, analogous to the
    Planck scale in effective approaches to quantum gravity.

    Above this scale, spectral locality breaks down and the reconstruction of smooth
    coordinates, distances, or curvature becomes ill-defined.
    The breakdown of the geometric description at high spectral energies is not a
    pathology but a physical prediction of the theory, signaling a transition to a
    genuinely pre-geometric regime.
