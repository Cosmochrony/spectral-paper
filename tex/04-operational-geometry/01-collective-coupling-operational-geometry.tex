\subsection{Collective Coupling and Operational Geometry}
  \label{subsec:collective-coupling-operational-geometry}

  The emergence of curvature in a relational framework can be understood as a
  collective effect.
  Rather than postulating a fundamental gravitational field, geometric properties
  arise from the way relational variations influence one another across extended
  regions of the system.

  In regimes where the relational structure is approximately homogeneous, variations
  propagate uniformly and admit a simple effective description.
  Localized irregularities in the relational connectivity modify this collective
  response, altering how variations can be correlated between neighboring regions.
  This modulation can be summarized by an effective coupling function characterizing
  the stiffness of the relational structure with respect to relative deformations.

  Importantly, this coupling is defined without reference to any background metric,
  coordinate system, or pre-existing notion of spatial separation.
  Distance is instead defined operationally: two regions are considered close if
  relational variations can be efficiently correlated between them, and distant
  otherwise.

  In the continuum and weakly inhomogeneous regime, this operational notion of
  proximity admits a compact geometric representation.
  An effective metric can then be introduced as a descriptive tool summarizing the
  collective response of the relational structure to local variations.
  The metric does not constitute an independent degree of freedom but encodes
  coarse-grained relational regularities.

  From this perspective, curvature is not a primitive geometric property nor the
  result of a dynamical field equation.
  It emerges as a macroscopic descriptor of how localized relational features
  modulate the collective connectivity of the system.
  Geometry thus functions as an effective encoding of constrained relational
  organization rather than as a fundamental ontological entity.
