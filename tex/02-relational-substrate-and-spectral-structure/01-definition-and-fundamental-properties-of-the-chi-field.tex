\subsection{Relational Structure}
  \label{subsec:relational-structure}

  We introduce a minimal relational framework intended to capture the weakest
  structural assumptions required for spectral reconstruction of the geometry.
  No spacetime manifold, metric tensor, coordinate system, or causal structure was postulated at this level.
  The framework was formulated deliberately in non-geometric terms.

  The system is described by a set of abstract relational elements together with the admissible relations among them.
  These relations define a connectivity structure, that may be represented discretely or in a coarse-grained form.
  No embedding space or background geometry was assumed or required.

  The relational elements are not assigned local values, dimensional quantities, or order parameters.
  Such quantities arise only at an effective level, when relational configurations
  permit a stable spectral reconstruction.
  In particular, notions of distance, dimension, and curvature are not fundamental
  primitives but are reconstructed descriptors.

  A key feature of the reconstruction problem is its inherent non-injectivity,
  which may give rise to identical effective geometric
  descriptions, whereas a single configuration may admit multiple equivalent spectral representations.
  This loss of information is a structural property of spectral reconstruction and
  does not rely on any physical coarse-graining mechanism.

  Throughout this work, geometric and field-theoretic languages were employed strictly
  for effective descriptive convenience.
  Such a language refers to reconstructed structures that become meaningful only in
  regimes in which the relational system supports a consistent spectral continuum approximation.
  This does not imply the existence of underlying spacetime entities or dynamic fields.

  This relational starting point provides a neutral basis for defining spectral
  operators and investigating how effective geometric properties can emerge from the connectivity data alone.
