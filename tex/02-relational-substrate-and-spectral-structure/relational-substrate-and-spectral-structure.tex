\clearpage

\section{Relational Substrate and Spectral Structure}
  \label{sec:relational-substrate-and-spectral-structure}

  In this section, we introduce the minimal relational framework underlying the
  spectral constructions developed in the remainder of the paper.
  The purpose is not to postulate a spacetime manifold, a metric tensor, or a set of
  dynamical fields, but to specify the weakest structural assumptions required to
  define spectral operators and to extract effective geometric information from them.

  We consider systems described by relational connectivity data, represented by
  discrete or coarse-grained graphs.
  The vertices of the graph correspond to abstract relational elements, while edges
  encode admissible relations between them.
  Such relational networks may be weighted or unweighted and are widely used
  to model connectivity structures across many domains~\cite{Newman2010}.
  No embedding space, coordinate chart, or background geometry is assumed.
  In particular, notions of distance, dimension, or curvature are not introduced at
  this stage.

  The only primitive structure required is the existence of a well-defined
  adjacency relation, possibly weighted, from which a Laplacian operator can be
  constructed.
  The use of graphs as minimal representations of relational structure follows
  standard constructions in graph theory~\cite{Diestel2017}.
  This operator encodes the local connectivity properties of the relational system
  and provides access to its spectral data.
  The spectrum of the Laplacian constitutes the primary object of interest in what
  follows.

  We emphasize that the relational graphs considered here are not assumed to be
  fundamental in a physical sense.
  Rather, they are used as minimal mathematical representations of relational
  structure, suitable for spectral analysis.
  Different graph realizations may correspond to the same effective geometric
  description, reflecting the fact that spectral reconstruction is generally
  non-injective.

  Effective geometric notions are introduced only at a secondary level, through
  spectral constructions applied to families of admissible relational graphs.
  When appropriate consistency and regularity conditions are satisfied, these
  constructions allow for the emergence of continuum-like geometric descriptions.
  The criteria for such admissibility and the associated reconstruction procedures
  are developed in the following sections.

  No assumptions concerning dynamics, temporal ordering, or causal structure are
  made in this section.
  The framework is entirely kinematical at this stage.
  Any reference to evolution or ordering will be introduced later only as an
  effective or auxiliary notion, when required by specific reconstruction schemes.

  This relational and spectral starting point provides a neutral and flexible basis
  for addressing the geometric reconstruction problem.
  It allows us to investigate how metric properties can arise from spectral data
  alone, without committing to a specific underlying ontology or dynamical theory.

  \subsection{Relational Structure}
  \label{subsec:relational-structure}

  We introduce a minimal relational framework intended to capture the weakest
  structural assumptions required for spectral reconstruction of geometry.
  No spacetime manifold, metric tensor, coordinate system, or causal structure is
  postulated at this level.
  The framework is deliberately formulated in non-geometric terms.

  The system is described by a set of abstract relational elements together with
  admissible relations among them.
  These relations define a connectivity structure, which may be represented
  discretely or in a coarse-grained form.
  No embedding space or background geometry is assumed or required.

  The relational elements themselves are not assigned local values, dimensional
  quantities, or order parameters.
  Such quantities arise only at an effective level, when relational configurations
  admit a stable spectral reconstruction.
  In particular, notions of distance, dimension, and curvature are not fundamental
  primitives but reconstructed descriptors.

  A key feature of the reconstruction problem is its inherent non-injectivity:
  distinct relational configurations may give rise to identical effective geometric
  descriptions, while a single configuration may admit multiple equivalent spectral
  representations.
  This loss of information is a structural property of spectral reconstruction and
  does not rely on any physical coarse-graining mechanism.

  Throughout this work, geometric and field-theoretic language is employed strictly
  as an effective descriptive convenience.
  Such language refers to reconstructed structures that become meaningful only in
  regimes where the relational system supports a consistent spectral continuum
  approximation.
  It does not imply the existence of underlying spacetime entities or dynamical
  fields.

  This relational starting point provides a neutral basis for defining spectral
  operators and for investigating how effective geometric properties can emerge
  from connectivity data alone.

