\subsection{Particles as Stable Wave Configurations}
  \label{subsec:particles-as-stable-wave-configurations}

  Within the Cosmochrony framework, particles are not fundamental point-like entities.
  They arise only at the level of effective descriptions, as stable and localized
  configurations within projected $\chi$ descriptions~\cite{Rajaraman1982}.
  Such configurations correspond to persistent patterns that locally constrain the
  admissible relaxation ordering, rather than to elementary objects propagating on a
  pre-existing spacetime background.

  In effective geometric regimes, these structures may be described using a wave-like
  or soliton-like language.
  Their stability reflects the existence of localized regions in which further
  relaxation is strongly inhibited, allowing the configuration to persist under
  interactions and effective displacement.
  Apparent particle propagation corresponds to a continuous reorganization of
  admissible projected descriptions, not to the motion of an object through a
  fundamental spacetime.

  Particle-like behavior therefore does not originate from intrinsic degrees of
  freedom of the $\chi$ substrate.
  It emerges instead as a stable relational configuration within the effective projected description
  once a spacetime interpretation becomes applicable.
  In this sense, particles are best understood as emergent, dynamically sustained
  patterns of constrained relaxation, rather than as ontologically primitive objects.
