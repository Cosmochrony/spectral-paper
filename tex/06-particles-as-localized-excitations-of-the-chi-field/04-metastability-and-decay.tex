\subsection{Metastability, Projection, and Particle Decay}
  \label{subsec:metastability-and-decay}

  In the Cosmochrony framework, particle-like entities are not elementary objects but
  stabilized collective regimes of the projected field \(\chi_{\mathrm{eff}}\).
  Their persistence reflects a balance between structural constraints—topological,
  geometric, or relational—and the capacity of the surrounding substrate to accommodate
  relaxation.

  A stable particle corresponds to a deep basin of admissible projected configurations.
  An unstable particle, by contrast, occupies only a shallow or fragile basin: it remains
  admissible, but concentrates structural constraints in a way that impedes efficient
  relaxation.
  Such configurations are therefore metastable rather than forbidden.

  Particle decay is interpreted as the structural reorganization of such a metastable
  configuration.
  When the concentration of constraints exceeds what can be sustained by a single
  projected entity, admissibility cannot be maintained at the level of a unified
  description.
  Stability is recovered only through factorization into several less constrained
  localized configurations, possibly accompanied by weakly structured excitations.

  This mechanism is closely related to the non-injective character of the projection from
  the \(\chi\)-substrate to effective observables.
  In the case of quantum entanglement, a single underlying \(\chi\)-configuration admits a
  non-factorizable but stable projected description.
  In the case of decay, the same non-factorizability becomes unstable: no single projected
  description remains admissible, and fragmentation becomes unavoidable.

  In this sense, entanglement and decay represent two regimes of the same projection
  structure.
  Entanglement corresponds to non-factorizability without fragmentation, while decay
  corresponds to non-factorizability that forces fragmentation.
  The distinction lies not at the level of the \(\chi\)-substrate, but in the stability
  properties of the projected description under admissible fluctuations.

  The finite lifetime of unstable particles does not reflect a fundamental indeterminacy.
  Projected configurations are continuously subject to small admissible variations, and
  decay occurs when such variations cross a structural reorganization threshold.
  If the probability of reaching such a threshold is approximately constant, the observed
  exponential decay law follows as a statistical signature of metastability.

  The instability responsible for particle decay originates from the intrinsic
  susceptibility of certain projected configurations to their own admissible fluctuations.
  Such fluctuations are continuously present in the \(\chi\)-substrate and, in
  metastable configurations, may drive the system across an admissibility threshold,
  triggering structural fragmentation.

  A more technical characterization of metastability, admissible factorization channels,
  and decay widths is provided in Appendix~\ref{subsec:metastability-decay-channels-and-exponential-lifetimes}.

  \begin{figure}[t]
      \centering
      \begin{tikzpicture}[
        box/.style={draw, rounded corners, align=center, inner sep=6pt},
        arr/.style={->, thick},
        lab/.style={font=\small, align=center}
      ]
        % Parent
        \node[box] (A) {$\chi_{\mathrm{eff},A}$\\[-2pt]\footnotesize metastable\\[-2pt]\footnotesize (single knot)};

        % Barrier / threshold
        \node[box, right=2.7cm of A] (T) {Admissible\\[-2pt]\footnotesize factorization\\[-2pt]\footnotesize threshold};

        % Daughters
        \node[box, right=2.9cm of T, yshift=1.1cm] (B1) {$\chi_{\mathrm{eff},1}$\\[-2pt]\footnotesize stable knot};
        \node[box, right=2.9cm of T] (B2) {$\chi_{\mathrm{eff},2}$\\[-2pt]\footnotesize stable knot};
        \node[box, right=2.9cm of T, yshift=-1.1cm] (B3) {$\chi_{\mathrm{eff},3}$\\[-2pt]\footnotesize stable /\\[-2pt]\footnotesize sub-threshold};

        % Radiation channel
        \node[box, below=1.6cm of B2] (R) {$\chi_{\mathrm{eff,rad}}$\\[-2pt]\footnotesize light excitations\\[-2pt]\footnotesize (photon/$\nu$/...)};

        % Arrows
        \draw[arr] (A) -- node[lab, above] {projective variability\\\footnotesize (threshold crossing)} (T);
        \draw[arr] (T) -- (B1);
        \draw[arr] (T) -- (B2);
        \draw[arr] (T) -- (B3);
        \draw[arr] (T) -- node[lab, right] {\footnotesize mismatch evacuation} (R);

        % Conservation annotation
        \node[lab, above=0.35cm of B2] (Q) {$Q(\chi_A)=\sum_i Q(\chi_i)$\\[-2pt]\footnotesize structural invariants redistributed};
      \end{tikzpicture}
      \caption{Structural interpretation of particle decay in Cosmochrony.
      A metastable localized projected configuration can transition, via an admissible
      factorization threshold, into several more stable localized configurations plus weak
      excitations that evacuate the residual structural mismatch.}
      \label{fig:decay-fragmentation}
    \end{figure}
