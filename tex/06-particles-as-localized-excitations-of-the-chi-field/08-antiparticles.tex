\subsection{Antiparticles}
  \label{subsec:antiparticles}

  Within the Cosmochrony framework, antiparticles are not interpreted as independent
  fundamental entities, nor as excitations propagating backward in time.
  They arise instead, at the level of effective descriptions, as relationally conjugate
  counterparts of particle-like projected configurations.

  A particle and its antiparticle correspond to projected configurations belonging to
  distinct but conjugate topological classes within the space of admissible projected
  descriptions.
  These classes are related by an internal reversal of relational organization, rather
  than by the inversion of a fundamental dynamical variable, temporal orientation, or
  spacetime trajectory.

  This conjugation reflects a symmetry of admissible projected structures under
  relational reversal.
  It does not presuppose the existence of intrinsic charges, fundamental fields, or
  time-oriented dynamics at the level of the $\chi$ substrate.

  Annihilation processes occur when a particle-like projected configuration and its
  conjugate combine into a composite projected description that no longer supports
  localized structural constraints.
  Such a configuration admits a continuous deformation toward a more homogeneous
  effective description, in which localized resistance to relaxation disappears.

  In effective geometric and quantum descriptions, this transition manifests as the
  conversion of particle--antiparticle structure into delocalized radiation-like
  projected excitations.
  No fundamental structure is destroyed in this process.
  The underlying relational substrate $\chi$ remains intact, while localized
  topological organization is redistributed into admissible projected configurations
  with extended support.

  In this sense, particle--antiparticle annihilation does not represent the destruction
  of matter, but a reorganization of effective relational structure from localized to
  delocalized forms within the space of admissible projected descriptions.

  \paragraph{Interpretative Remark: Annihilation as Structural Unknotting}

    Within the solitonic interpretation of matter developed throughout this work,
    particle-like configurations correspond to localized topological organizations of the
    $\chi$ field that inhibit its global relaxation.
    From this perspective, antimatter may be understood not merely as a conjugate
    configuration, but as a relational mode that enables the release of such constraints.

    Particle--antiparticle annihilation can thus be interpreted as a process of
    \emph{structural unknotting} (in the topological and descriptive sense), in which
    mutually conjugate topological organizations combine into a configuration that no
    longer sustains localized resistance.
    The relaxation flux previously stored in the constrained solitonic structure is then
    freed to redistribute into delocalized, radiation-like projected excitations.

    This interpretation clarifies why annihilation releases an energy proportional to the
    inertial mass of the particle: within Cosmochrony, mass itself measures the degree of
    topological obstruction to relaxation.

  \paragraph{Why Antimatter Does Not Require Time Reversal}
    \label{subsec:why-antimatter-no-time-reversal}

    In conventional relativistic quantum field theory, antiparticles are often
    heuristically interpreted as particles propagating backward in time.
    While this picture is computationally useful in perturbative formalisms, it does
    not reflect a fundamental physical process.
    Within the Cosmochrony framework, such an interpretation is neither required nor
    meaningful.

    At the fundamental level, the $\chi$ substrate admits no temporal parameter and no
    notion of temporal reversal.
    The monotonic ordering structure intrinsic to $\chi$ defines an absolute arrow of
    admissible projection, which cannot be inverted.
    All effective descriptions compatible with Cosmochrony therefore inherit a
    unidirectional ordering, corresponding operationally to the observed arrow of time.

    Antiparticles do not arise from reversing this ordering.
    Instead, they correspond to projected configurations belonging to topologically
    conjugate classes within the space of admissible projected descriptions.
    The distinction between particles and antiparticles reflects an internal relational
    reversal of projected structure, not a reversal of temporal succession.

    The apparent association between antimatter and time reversal in standard formalisms
    originates from the structure of relativistic wave equations and their symmetry
    properties under complex conjugation.
    In Cosmochrony, this correspondence is reinterpreted as a feature of the effective
    representation, not as an indication of physical evolution backward in time.

    In particular, processes involving antiparticles always occur within the same
    monotonic ordering of projected configurations as their particle counterparts.
    Antimatter participates in relaxation, interaction, and annihilation processes
    according to the same intrinsic arrow of ordering.
    No admissible projected description involves a reversal of the effective relaxation
    parameter or a decrease of $\chi_{\mathrm{eff}}$.

    Annihilation processes further clarify this point.
    Particle--antiparticle annihilation corresponds to the merging of conjugate projected
    configurations into a delocalized admissible description, not to the cancellation of
    forward and backward temporal trajectories.
    The effective outcome is a redistribution of relational structure into radiation-like
    projected excitations, all evolving within the same monotonic ordering framework.

    In summary, antimatter in Cosmochrony does not require time reversal.
    It reflects a structural conjugation in the space of admissible projected
    configurations, fully compatible with a unique and irreversible ordering of
    projected descriptions.
    Time-reversal interpretations therefore belong to the mathematical convenience of
    certain effective formalisms, not to the ontological structure of the underlying
    relational substrate.

  \paragraph{Matter--Antimatter Asymmetry without Fundamental CP Violation}
    \label{subsec:matter-antimatter-asymmetry-without-cp}

    The observed dominance of matter over antimatter in the universe is commonly
    attributed to CP violation occurring in the early universe, supplemented by
    additional dynamical assumptions.
    Within the Cosmochrony framework, matter--antimatter asymmetry admits a more direct
    structural interpretation, without requiring fundamental CP violation at the level
    of the $\chi$ substrate.

    As previously established, particles and antiparticles correspond to conjugate
    classes of admissible projected configurations.
    These classes are structurally paired at the pre-geometric level, but their
    realization in effective spacetime descriptions depends on the properties of the
    projection.

    If the projection from $\chi$ to effective spacetime is chiral, conjugate classes
    need not be realized with equal stability, persistence, or projectability.
    One class may preferentially admit long-lived, localized projected configurations,
    while its conjugate may predominantly yield delocalized or short-lived realizations.
    This imbalance arises without any fundamental asymmetry in the underlying relational
    substrate.

    In such a scenario, matter--antimatter asymmetry is not generated dynamically through
    time-dependent processes, nor through explicit CP-breaking interactions.
    It emerges instead as a selection effect imposed by the structure of admissible
    projections.
    Only those configurations compatible with stable, persistent projection contribute
    significantly to the effective physical universe.

    Importantly, this mechanism does not require a departure from a unique and monotonic
    ordering of projected configurations.
    Both matter and antimatter evolve within the same intrinsic arrow of ordering defined
    by $\chi$.
    The asymmetry reflects differential projectability, not temporal reversal or
    dynamical bias.

    In this sense, the observed matter-dominated universe is interpreted as a consequence
    of projective asymmetry rather than of fundamental CP violation.
    The apparent violation of CP symmetry in effective descriptions encodes the geometry
    of projection, not an intrinsic imbalance in the underlying relational ontology.
