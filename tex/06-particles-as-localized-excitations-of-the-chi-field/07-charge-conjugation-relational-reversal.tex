\subsection{Charge Conjugation as Relational Reversal}
  \label{subsec:charge-conjugation-relational-reversal}

  In conventional quantum field theory, charge conjugation is introduced as an
  operation that reverses the sign of internal charges associated with a field.
  While this description is operationally effective, it presupposes the existence
  of fundamental charge degrees of freedom.
  Within the Cosmochrony framework, no such assumption is required.

  At the level of the pre-geometric $\chi$ substrate, there are no intrinsic charges,
  gauge fields, or conserved internal labels.
  What are described in effective theories as charges arise only after projection,
  as stable invariants characterizing admissible projected configurations.
  Charge is therefore not a fundamental attribute, but a relational property of
  projected descriptions.

  Charge conjugation is correspondingly reinterpreted as a transformation between
  \emph{relationally conjugate} projected configurations.
  A particle and its charge-conjugate counterpart belong to distinct but paired
  topological classes within the space of admissible projected descriptions.
  These classes are related by an internal reversal of relational organization,
  rather than by the inversion of a primitive scalar quantity.

  This relational reversal does not involve time reversal, energy inversion, or
  dynamical evolution at the level of the $\chi$ substrate.
  It is instead a symmetry of the admissible projection structure itself.
  The existence of such conjugate classes reflects the internal duality of the
  projection fiber, not an independent physical operation acting on spacetime fields.

  Within effective quantum descriptions, this relational reversal manifests as
  complex conjugation of wavefunctions and as the inversion of effective charge
  labels.
  These representations encode the topological distinction between conjugate
  projected configurations, without implying that charge is fundamental or that
  conjugation corresponds to a physical process occurring in time.

  Importantly, charge conjugation symmetry is not guaranteed within this framework.
  Because charge is a projective and relational invariant, the symmetry between
  conjugate classes may be broken by asymmetries in the projection structure itself.
  Violations of charge conjugation symmetry therefore admit a natural structural
  interpretation, without invoking explicit symmetry-breaking terms at the level
  of the $\chi$ substrate.

  In summary, charge conjugation in Cosmochrony is not the reversal of a fundamental
  charge, but a relational reversal between paired classes of admissible projected
  configurations.
  It reflects a symmetry of the effective descriptive structure, not an ontological
  operation acting on primitive physical entities.

\paragraph{CP Symmetry as Projective Chirality}
  \label{subsec:cp-as-projective-chirality}

  In conventional particle physics, CP symmetry combines charge conjugation and
  spatial parity reversal.
  Its observed violation is typically introduced through complex phases in effective
  Lagrangians, without a deeper structural explanation.
  Within the Cosmochrony framework, CP symmetry admits a natural reinterpretation in
  terms of projective chirality.

  As established in the preceding sections, charge conjugation corresponds to a
  relational reversal between conjugate classes of admissible projected configurations.
  Parity, in turn, is not interpreted as a fundamental inversion of spatial coordinates,
  but as a reversal of orientation within the effective geometric description that
  emerges after projection.

  CP symmetry therefore corresponds to a combined transformation acting on the
  \emph{orientation of the projection itself}.
  In Cosmochrony, projected configurations may admit an intrinsic chirality: an
  orientation asymmetry in the mapping from the pre-geometric $\chi$ substrate to
  effective spacetime descriptions.
  This chirality is not imposed dynamically, nor encoded in a fundamental interaction,
  but arises from the geometry of the projection fiber.

  When the projection is achiral, conjugate configurations are mapped symmetrically,
  and CP symmetry is preserved at the effective level.
  When the projection is chiral, however, the relational reversal associated with charge
  conjugation is no longer equivalent to a parity-reversed description.
  CP symmetry is then violated as a direct consequence of projective asymmetry.

  Importantly, this violation does not require the introduction of explicit CP-violating
  terms at the level of the $\chi$ substrate.
  It reflects a geometric asymmetry in how relational structures are realized as
  effective spacetime configurations.
  CP violation is therefore reinterpreted as a manifestation of projective chirality,
  not as a fundamental breaking of symmetry in the underlying ontology.

  In this view, CP symmetry is effective, contingent, and emergent.
  Its violation signals an asymmetry of the projection itself, rather than a failure of
  fundamental physical laws.
