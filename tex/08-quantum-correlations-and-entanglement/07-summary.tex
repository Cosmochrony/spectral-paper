\subsection{Summary}
  \label{subsec:summary-quantum}

  Within the Cosmochrony framework, the entire phenomenology of the Standard Model—from gauge interactions to quantum
  correlations—emerges as a consequence of the spectral properties of the substrate $\chi$ and its projection $\Pi$.

  \begin{itemize}
    \item \textbf{Gauge Mediators as Projection Dynamics:}
    Interactions are not mediated by autonomous fields but by fluctuations in the projection process itself. While the
    photon represents scalar transmittance, the $W$ and $Z$ bosons emerge as \textit{shear modes}
    , their mass being a direct manifestation of the spectral rigidity of the projection fiber.
    \item \textbf{Topological Origin of Matter:}
    Strong interactions and confinement are reinterpreted through the lens of topological stability. The ``color'' force
    is a macroscopic expression of the energy required to maintain the coherence of knotted solitonic configurations
    (like the $Q=3$ proton) under deformation.
    \item \textbf{Mass as Spectral Overlap:}
    The Higgs mechanism is replaced by the principle of spectral overlap. Mass is no longer an intrinsic coupling
    constant but a dynamical resonance between a configuration’s stability spectrum and the global relaxation flux of
    the substrate.
    \item \textbf{Quantum Emergence:}
    Entanglement and non-locality do not require superluminal signaling. They reflect the persistence of
    non-factorizable configurations across the projection. Quantum mechanics thus emerges as an effective statistical
    framework describing the limits of local projectability for globally consistent spectral descriptions.
  \end{itemize}

  In this sense, the Standard Model is not a collection of arbitrary particles and forces, but an effective theory
  describing the \textbf{harmonics of relaxation}
      . The transition from the substrate to spacetime is a filter that organizes these harmonics into what we perceive
      as discrete symmetries and fundamental constants.
