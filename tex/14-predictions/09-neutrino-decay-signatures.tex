\subsection{Neutrino-Mediated Relaxation and Decay Signatures}
  \label{subsec:neutrino-decay-signatures}

  In Cosmochrony, particle decay and neutrino emission are manifestations of structural
  reorganization rather than independent microscopic processes.
  This interpretation leads to distinct observational signatures.

  Because neutrino-like excitations act as non-local relaxation channels, decay processes
  in the early universe contribute to an irreversible smoothing of admissible
  configurations.
  This smoothing is expected to leave detectable imprints across multiple observational
  domains.

  Specifically, the framework predicts:
  \begin{itemize}
    \item an enhanced role of neutrino backgrounds in suppressing large-scale coherence
    without behaving as conventional radiation pressure,
    \item a weak coupling between decay rates and late-time environmental conditions,
    reflecting their origin in early structural metastability,
    \item possible correlations between decay-driven neutrino emission and large-scale
    anisotropies observed in the cosmic microwave background.
  \end{itemize}

  At the particle-physics level, this interpretation suggests that decay lifetimes and
  branching ratios encode information about the stability landscape of admissible
  projected configurations rather than fundamental stochasticity.
  Future precision measurements of rare decays may therefore provide indirect probes of
  the structural relaxation dynamics underlying the Cosmochrony framework.
