\subsection{Relation to the Higgs Mechanism}
  \label{subsec:relation-to-the-higgs-mechanism}

  In the Standard Model, particle masses arise through spontaneous symmetry
  breaking of the electroweak sector, mediated by the Higgs field.
  In the Cosmochrony framework, mass is not introduced as a fundamental parameter,
  but emerges as a measure of the resistance of localized $\chi$ configurations to
  the global relaxation flow.

  These two descriptions are not in contradiction.
  Rather, the Higgs field can be interpreted as an effective low-energy
  manifestation of interactions between localized excitations and the surrounding
  $\chi$ background.
  From this perspective, the Higgs condensate encodes how particle-like
  configurations acquire inertial properties within an already structured,
  relaxing substrate.

  Cosmochrony does not question the empirical success of the Higgs mechanism, nor
  does it seek to modify its phenomenology at experimentally accessible energies.
  Instead, it suggests that the Higgs field, like the spacetime metric or quantum
  wavefunctions, should be regarded as an emergent effective degree of freedom
  rather than as a fundamental constituent.
  Within this interpretation, the observed Higgs boson corresponds to a collective
  excitation associated with the stabilization of mass-generating configurations.

  In this framework, the Higgs vacuum expectation value (VEV) is not treated as an
  independent fundamental scale, but as a quantity indirectly determined by the
  local properties of the $\chi$ background.
  This provides a conceptual link between the microphysical origin of particle
  masses and the large-scale relaxation dynamics of the underlying field.

  A detailed derivation of the electroweak sector as an effective theory emerging
  from $\chi$ dynamics lies beyond the scope of the present work and is left for
  future investigation.

\subsection{Electroweak Symmetry Breaking as a Phase Transition in $\chi$}
  \label{subsec:ewsb_phase_transition}

  The spontaneous breaking of electroweak symmetry is reinterpreted in the Cosmochrony
  framework as a phase transition of the underlying $\chi$ field.
  Rather than invoking a fundamental Higgs potential, this transition reflects a change
  in the relaxation behavior of $\chi$ as it crosses a characteristic structural scale
  $\chi_c$.

  \paragraph{Phases of the $\chi$ field.}
    Two qualitative regimes may be distinguished:
    \begin{itemize}
      \item \emph{Symmetric phase} ($\chi \ll \chi_c$): the $\chi$ field relaxes essentially
      homogeneously, and localized excitations dissipate rapidly. In this regime, no
      persistent resistance to relaxation exists, and particle-like excitations are
      effectively massless.
      \item \emph{Broken phase} ($\chi \gtrsim \chi_c$): nonlinear self-interactions of $\chi$
      stabilize localized excitations. These solitonic configurations resist relaxation
      and correspond to massive particles.
    \end{itemize}

  \paragraph{Qualitative mechanism.}
    The transition is driven by the intrinsic nonlinear dynamics of $\chi$, which favors
    the emergence of stable solitonic configurations once the field exceeds the critical
    scale $\chi_c$.
    No explicit potential $V(\chi,\phi_H)$ is postulated.
    Instead, electroweak symmetry breaking is described phenomenologically as a qualitative
    change in the relaxation resistance of $\chi$:
    \begin{itemize}
      \item for $\chi < \chi_c$, excitations relax efficiently (massless phase);
      \item for $\chi > \chi_c$, excitations resist relaxation and acquire inertial mass
      (massive phase).
    \end{itemize}

  \paragraph{Avoiding fine-tuning.}
    The scale $\chi_c$ is not introduced as a free adjustable parameter.
    It is constrained indirectly by independent considerations, including:
    \begin{itemize}
      \item cosmological observations sensitive to relaxation dynamics (e.g.\ the Hubble
      tension and large-scale anisotropies; see Section~9.7),
      \item particle mass hierarchies, such as the proton-to-electron mass ratio
      (see Section~B.3).
    \end{itemize}
    Within this framework, the proportionality between the Higgs vacuum expectation value
    $\langle \phi_H \rangle$ and the inverse length scale $\hbar c / \chi_c$ is not fixed
    numerically, but arises from the topological stability of solitonic excitations, up to
    dimensionless factors of order unity that depend on the soliton class.

\subsection{Mass Generation via Topological Solitons}
\label{subsec:mass_generation_solitons}

In the broken phase ($\chi \gtrsim \chi_c$), fermions and gauge bosons acquire mass
through their association with topological solitons of the $\chi$ field.

\paragraph{Fermion masses.}
  Fermions correspond to skyrmion-like solitons (see Section~B.2.3), whose mass is set by
  the resistance of the configuration to relaxation.
  At leading order, this yields the scaling
  \begin{equation}
    m_f \;\propto\; y_f^{\mathrm{eff}} \, \frac{\hbar}{\chi_c},
  \end{equation}
  where $y_f^{\mathrm{eff}}$ is an effective Yukawa-like coupling encoding the topological
  class and internal structure of the soliton.
  The observed hierarchy of fermion masses is interpreted as arising from distinct
  topological invariants (e.g.\ winding or linking numbers) associated with different
  soliton families.

\paragraph{Gauge boson masses.}
  Gauge bosons are associated with vortex-like solitonic excitations of $\chi$
  (see Section~B.2.2).
  Their masses scale as
  \begin{equation}
    m_W \;\propto\; g \, \frac{\hbar}{\chi_c},
  \end{equation}
  where $g$ is the relevant gauge coupling.
  The weak mixing angle $\theta_W$ is not derived here; it is interpreted instead as
  reflecting a ratio of topological charges between neutral and charged solitonic sectors.

\subsection{Phenomenological Implications and Open Questions}
\label{subsec:ewsb_implications}

\paragraph{Testable implications.}
  This emergent interpretation of electroweak symmetry breaking suggests potential,
  though highly suppressed, deviations from Standard Model expectations in extreme
  regimes, including:
  \begin{itemize}
    \item small variations of the effective Higgs vacuum expectation value in strong
    gravitational fields, where $\chi$ relaxation may be locally slowed,
    \item modifications to Higgs production or decay rates at very high energies, due to
    non-minimal couplings between $\chi$ and Higgs-like collective modes.
  \end{itemize}
  In weak-field and low-energy regimes, any such effects are expected to be strongly
  suppressed, consistent with current experimental bounds.

\paragraph{Open challenges.}
  Several important issues remain open:
  \begin{itemize}
    \item deriving the precise form of the coupling between $\chi$ and solitonic
    excitations that reproduces the full Standard Model mass spectrum,
    \item understanding the origin of gauge couplings ($g$, $g'$) from the internal
    symmetry structure of $\chi$,
    \item quantifying the stability of the electroweak phase under cosmological evolution,
    including whether $\langle \phi_H \rangle$ exhibits any residual dependence on the
    global relaxation state of $\chi$.
  \end{itemize}

\subsection{Summary: Higgs as an Emergent Phenomenon}
\label{subsec:higgs_summary}

In Cosmochrony, the Higgs field is interpreted as an effective description of a
structured phase of the $\chi$ field, emerging when $\chi \gtrsim \chi_c$.
The electroweak scale ($\sim 246\,\mathrm{GeV}$) is associated with the inverse
correlation length $\hbar c / \chi_c$ that characterizes stable solitonic excitations.

Mass generation arises from the topological stability of $\chi$ solitons rather than
from a fundamental scalar potential, avoiding fine-tuning.
No direct connection to the Planck scale is assumed: the electroweak scale is independent
of gravitational scales and originates from the internal dynamics of $\chi$.

In this sense, the Higgs mechanism, gravity, and cosmology are unified at a conceptual
level, as emergent phenomena arising from the same underlying $\chi$ dynamics, but
manifesting at distinct structural scales.
