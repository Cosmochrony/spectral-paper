\subsection{Relation to the Higgs Mechanism}
  \label{subsec:relation-to-the-higgs-mechanism}

  In the Standard Model, particle masses arise through spontaneous symmetry
  breaking of the electroweak sector, mediated by the Higgs field.
  In the Cosmochrony framework, mass is not introduced as a fundamental parameter,
  but emerges as a measure of the resistance of localized $\chi$ configurations to
  the global relaxation flow.

  These two descriptions are not in contradiction.
  Rather, the Higgs field can be interpreted as an effective low-energy
  manifestation of interactions between localized excitations and the surrounding
  $\chi$ background.
  From this perspective, the Higgs condensate encodes how particle-like
  configurations acquire inertial properties within an already structured,
  relaxing substrate.

  Cosmochrony does not question the empirical success of the Higgs mechanism, nor
  does it seek to modify its phenomenology at experimentally accessible energies.
  Instead, it suggests that the Higgs field, like the spacetime metric or quantum
  wavefunctions, should be regarded as an emergent effective degree of freedom
  rather than as a fundamental constituent.
  Within this interpretation, the observed Higgs boson corresponds to a collective
  excitation associated with the stabilization of mass-generating configurations.

  In this framework, the Higgs vacuum expectation value (VEV) is not treated as an
  independent fundamental scale, but as a quantity indirectly determined by the
  local properties of the $\chi$ background.
  This provides a conceptual link between the microphysical origin of particle
  masses and the large-scale relaxation dynamics of the underlying field.

  A detailed derivation of the electroweak sector as an effective theory emerging
  from $\chi$ dynamics lies beyond the scope of the present work and is left for
  future investigation.
