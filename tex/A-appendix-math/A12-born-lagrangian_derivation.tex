\subsection{Relational Consistency of the Effective Lagrangian}
  \label{sec:born-lagrangian_derivation_v171}

  The effective Lagrangian for the \(\chi\) field is \textbf{not postulated arbitrarily} but constructed as a \textbf{canonical representation} of the relational dynamics introduced in Section~\ref{sec:dynamical-equation-for-the-chi-field}. This appendix demonstrates how its Born--Infeld-like form emerges from fundamental principles, clarifies its systematic selection, and addresses the continuum limit with full mathematical rigor.

  \subsubsection*{Step 1: Relational Constraint and Bounded Variations}
    \label{subsec:A12-relational-constraint_v171}

    At the fundamental level, the dynamics of \(\chi\) are governed by a \textbf{discrete relational constraint}:
    \begin{equation}
      \mathcal{C}_i[\chi] \equiv \sum_j K_{ij} (\chi_i - \chi_j)^2 \le \chi_c^2,
      \label{eq:A12-relational-constraint_v171}
    \end{equation}
    where \(K_{ij} = K_{ji}\) is a symmetric connectivity matrix and \(\chi_c\) is the correlation scale. This constraint enforces bounded relative variations without assuming pre-existing spacetime, acting as a \textbf{structural causality condition}.

  \subsubsection*{Step 2: Variational Formulation with Global Order}
    \label{subsec:A12-variational-structure_v171}

    The dynamics are described by a constrained action with KKT conditions:
    \begin{equation}
      S[\{\chi_i\}, \{\mu_i\}] = \int d\lambda \left[ \sum_i \frac{1}{2} \left(\frac{d\chi_i}{d\lambda}\right)^2 - U[\{\chi_i\}] - \sum_i \mu_i(\lambda) \left(\mathcal{C}_i[\chi] - \chi_c^2 \right) \right],
      \label{eq:A12-action_v171}
    \end{equation}
    where:
    \begin{itemize}
      \item The kinetic term \(\frac{1}{2}(d\chi_i/d\lambda)^2\) is the \textbf{leading-order expansion} of any smooth functional governing ordered relaxation, with \(U[\{\chi_i\}]\) encoding additional constraints.
      \item The \textbf{global order} is ensured by the functional \(\Xi[\chi(\lambda)] \equiv \sum_i \chi_i(\lambda)\), with \(\frac{d\Xi}{d\lambda} \ge 0\).
      \item KKT conditions guarantee \(\mu_i(\lambda) \ge 0\) and \(\mu_i(\lambda)(\mathcal{C}_i[\chi] - \chi_c^2) = 0\).
    \end{itemize}

  \subsubsection*{Step 3: Continuum Limit and Canonical Form}
    \label{subsec:A12-continuum-limit_v171}

    In \textbf{projectable regimes}, the discrete constraint maps to a continuum bound:
    \begin{equation}
      |\nabla \chi|^2 \le c^2,
      \label{eq:A12-continuum-bound_v171}
    \end{equation}
    where \(\nabla\) is an emergent operator. For a lattice of spacing \(a\), we define:
    \begin{equation}
      (\nabla \chi)^2 \approx \frac{1}{a^2} \sum_{\langle i,j \rangle} (\chi_i - \chi_j)^2,
    \end{equation}
    yielding \(|\nabla \chi|^2 \le c^2\) with \(c^2 \equiv a^2 \chi_c^2 / K_0\). The continuum limit \(a \to 0\) is well-defined if \(K_0 \sim a^{-2}\).

    \paragraph*{Canonical Selection.}
      We seek a functional \(L_{\text{eff}} = f(|\nabla \chi|^2/c^2)\) satisfying:
    \begin{itemize}
      \item Free theory normalization: \(f(0) = -c^2\),
      \item Saturation: \(f(1) = 0\),
      \item Monotonicity: \(f'(x) > 0\) for \(x \in [0,1]\),
      \item Regularity: \(f''(x)\) finite.
      \end{itemize}
      The \textbf{canonical representation} is:
    \begin{equation}
      f(x) = -c^2 \sqrt{1 - x},
      \label{eq:A12-born-infeld-derivation_v171}
      \end{equation}
      yielding the Born--Infeld-like Lagrangian:
    \begin{equation}
      \mathcal{L}_{\text{eff}} = -c^2 \sqrt{1 - \frac{|\nabla \chi|^2}{c^2}} + \partial_t \chi.
      \label{eq:A12-effective-lagrangian_v171}
      \end{equation}

    \paragraph*{Selection Criteria.}
      The Born--Infeld form corresponds to the \textbf{minimal non-polynomial functional} satisfying boundedness, smooth saturation, and finite characteristic speeds. Other choices are admissible but introduce either degeneracies, non-saturating behavior, or additional scales.

  \subsubsection*{Step 4: Role of the Potential \(U[\{\chi_i\}]\)}
    \label{subsec:A12-potential-role_v171}

    The potential \(U[\{\chi_i\}]\) encodes additional relational constraints (e.g., topological terms).
    In the continuum limit:
    \begin{equation}
      U[\{\chi_i\}] \to \int d^3x \, V(\chi),
    \end{equation}
    where \(V(\chi)\) is an effective potential.
    The connection to the main text's \(V(\chi)\) is established via
    coarse-graining, ensuring consistency with solitonic solutions (Section~\ref{subsec:stability-analysis}).

  \subsubsection*{Step 5: Connection to Emergent Geometry}
    \label{subsec:A12-emergent-geometry_v171}

    A coarse-graining procedure \emph{admits} an auxiliary effective Lagrangian representation of the form:
    \begin{equation}
      \mathcal{L}_{\text{eff}} = -c^2 \sqrt{1 - \frac{|\nabla \chi|^2}{c^2}} + \partial_t \chi,
    \end{equation}
    where the linear term \(\partial_t \chi\) does not affect the equations of motion but fixes the orientation of the
    effective evolution parameter.

    The effective metric is defined via the Hessian of \(\mathcal{L}_{\text{eff}}\):
    \begin{equation}
      g_{\mu\nu}^{\text{eff}} \propto \frac{\partial^2 \mathcal{L}_{\text{eff}}}{\partial (\partial_\mu \chi) \partial (\partial_\nu \chi)},
      \label{eq:A12-emergent-metric_v171}
    \end{equation}
    up to conformal rescalings.
    This construction is valid in projectable regimes where \(K_{ij}\) approximates a
    continuum Laplacian (Section~\ref{subsec:numerical-validation-of-the-chi-rightarrow-chi_eff-transition}).

  \subsubsection*{Summary of Key Improvements}
    \begin{itemize}
      \item \textbf{Systematic selection} of the Born--Infeld form from first-principle constraints (boundedness, monotonicity, regularity).
      \item \textbf{Explicit continuum limit} with spectral Laplacian connection.
      \item \textbf{Clarified role of \(U[\{\chi_i\}]\)} and its continuum counterpart.
      \item \textbf{No circularity}: \(\mathcal{L}_{\text{eff}}\) is consistent with (not derived from) relational dynamics.
    \end{itemize}

  \subsubsection*{Scope and Limitations}
    The Born--Infeld-like Lagrangian is a \textbf{canonical representation} valid in projectable regimes.
    Outside these regimes:
    \begin{itemize}
      \item No spacetime description exists,
      \item Alternative functionals may be required,
      \item The discrete dynamics (Eq.~\eqref{eq:A12-relational-constraint_v171}) remain fundamental.
    \end{itemize}
