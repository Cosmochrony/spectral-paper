\subsection{Stability Analysis of the \texorpdfstring{$\chi$}{χ}-Field Dynamics}
  \label{subsec:stability-analysis}

  The stability of the $\chi$-field dynamics is a central requirement for Cosmochrony
  to define a physically consistent framework.
  Since $\chi$ is interpreted as a fundamental pre-geometric substrate, its effective
  descriptions must remain well-behaved under perturbations, without runaway growth
  or singular behavior.

  \noindent
  \emph{The stability analysis presented in this subsection does not concern the
    $\chi$ substrate itself, but the admissibility and robustness of its effective
    projected descriptions once a hydrodynamic regime is assumed.}

  In regimes where a smooth geometric description is applicable, the effective
  relaxation dynamics of $\chi$ may be written as
  \begin{equation}
    \partial_t \chi
    =
    c \sqrt{1 - \frac{|\nabla \chi|^2}{c^2}},
  \end{equation}
  where $\partial_t$ denotes an effective ordering parameter associated with the
  relaxation process, not a fundamental time variable.
  This representation is introduced solely for analytical convenience in the
  hydrodynamic regime.
  The ordering parameter labels successive admissible configurations in the projected
  description and does not correspond to an underlying temporal evolution of the
  $\chi$ substrate.

  Below we analyze the response of this effective dynamics to small deviations around
  homogeneous relaxation states.

  \subsubsection*{Perturbative Structure and Marginal Linear Stability}

    Consider a spatially homogeneous background solution
    \[
      \chi_0(t) = c\,t + \chi_{0,0},
    \]
    satisfying $\nabla \chi_0 = 0$ and $\partial_t \chi_0 = c$.
    We introduce a small perturbation
    \[
      \chi(x,t) = \chi_0(t) + \delta \chi(x,t),
      \qquad
      |\nabla \delta \chi| \ll c .
    \]

    Substituting into the evolution equation and expanding the square root yields
    \begin{equation}
      \partial_t \delta \chi
      =
      - \frac{1}{2c}\,|\nabla \delta \chi|^2
      + \mathcal{O}\!\left(|\nabla \delta \chi|^4\right).
    \end{equation}

    Importantly, no term linear in $\delta \chi$ appears.
    The homogeneous relaxation solution is therefore \emph{marginally stable at linear
order}: infinitesimal perturbations neither grow nor propagate dynamically at first
    order.
    This reflects the purely relaxational character of the effective $\chi$
    description and the absence of fundamental propagating modes at the linearized
    level.

  \subsubsection*{Nonlinear Stability and Dissipative Behavior}

    Although linear perturbations are marginal, the leading nonlinear correction is
    strictly negative.
    Any spatial inhomogeneity in the effective $\chi$ description therefore reduces
    the local relaxation rate and is dynamically suppressed.

    To make this explicit, consider the functional
    \begin{equation}
      E[\delta \chi]
      =
      \frac{1}{2}
      \int |\nabla \delta \chi|^2 \, d^3x ,
    \end{equation}
    which measures the degree of spatial inhomogeneity in the projected description.
    This functional has no energetic or variational meaning at the fundamental level
    and serves only as a diagnostic measure of geometric tension within the effective
    hydrodynamic regime.

    Using the evolution equation, one finds that $E[\delta \chi]$ is a non-increasing
    function of the ordering parameter.
    This follows from the fact that the relaxation flow is negative-definite in the
    presence of spatial gradients, acting systematically to reduce
    $|\nabla \delta \chi|^2$.

    Spatial gradients are therefore progressively smoothed, and perturbations remain
    bounded for all values of the ordering parameter.
    The effective dynamics is dissipative and contractive in configuration space,
    with no mechanism for amplification of perturbations.

    This establishes \emph{nonlinear stability} of the effective relaxation
    description.

  \subsubsection*{Special Configurations}

    For simple classes of perturbations, the qualitative behavior is transparent:
    \begin{itemize}
      \item \textbf{Planar perturbations:}
      Spatial oscillations do not propagate as waves, but are progressively flattened
      as the local relaxation rate decreases in regions of nonzero gradient.

      \item \textbf{Spherically symmetric perturbations:}
      Radial inhomogeneities decay monotonically, corresponding to a diffusion-like
      relaxation of geometric tension.
    \end{itemize}

    In all cases, the effective dynamics suppresses sharp gradients and prevents the
    formation of singular structures within the hydrodynamic description.

  \subsubsection*{Conclusion}

    The effective $\chi$-field dynamics is marginally stable at linear order and
    strictly stable once nonlinear effects are taken into account.
    This guarantees that the irreversible relaxation of $\chi$ admits stable and
    well-defined effective descriptions, providing a robust basis for the emergence of
    spacetime geometry, gravitation, and quantum phenomena within the Cosmochrony
    framework.

    Notably, this stability property is inseparable from the monotonic character of
    the relaxation process: the same structural constraint that defines the arrow of
    time also precludes dynamical instabilities in the projected description.
