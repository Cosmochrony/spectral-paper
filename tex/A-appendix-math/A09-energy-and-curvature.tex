\subsection{Energy and Curvature}
  \label{subsec:energy-and-curvature}

  In the Cosmochrony framework, energy is not introduced as a fundamental conserved
  quantity.
  Instead, it emerges as an effective and relational measure of how strongly a given
  configuration of the $\chi$ substrate resists the global relaxation process.
  Energy is therefore not a primitive substance, but a diagnostic of constrained
  relaxation within an otherwise monotonically ordered substrate.

  Once an effective geometric description becomes applicable, this resistance may be
  summarized by quantities that resemble familiar energy densities.
  At this phenomenological level, it is convenient to introduce the functional
  \begin{equation}
    \mathcal{E}_\chi^{\mathrm{eff}}
    \;=\;
    \frac{1}{2}
    \left[
      (\partial_t \chi)^2 + (\nabla \chi)^2
    \right],
  \end{equation}
  which provides a coarse-grained measure of temporal and spatial deformation of the
  projected $\chi$ configuration.

  This functional has no fundamental Hamiltonian or variational status.
  It is neither conserved nor associated with a symmetry of the underlying
  pre-geometric substrate, and it does not generate equations of motion.
  Its role is purely diagnostic, allowing constrained relaxation patterns to be
  characterized using a familiar field-theoretic language within the emergent
  geometric regime.

  Regions in which $\mathcal{E}_chi^{\mathrm{eff}}$ is large correspond to projected
  configurations where $\chi$ exhibits strong internal gradients, extended spatial
  modulation, or reduced local relaxation rates.
  Such configurations store a significant amount of \emph{relaxation resistance} and
  are interpreted as localized impediments to the global ordering process.
  In the effective spacetime description, these regions are naturally identified
  with particle-like excitations carrying inertial and gravitational properties.

  \paragraph{Orbital extension and effective energy ordering.}
    Within the Cosmochrony framework, the apparent ordering of atomic orbitals by
    increasing energy does \emph{not} arise from spatial distance to a nucleus as such.
    Instead, it reflects the structural cost associated with sustaining increasingly
    extended and oscillatory configurations of the $\chi$ field.

    An ``outer orbital'' corresponds, in the effective description, to a configuration
    in which the relaxation-resistant pattern associated with a particle excitation
    occupies a larger spatial domain and exhibits a higher degree of internal
    modulation.
    Maintaining such an extended pattern requires constraining the global relaxation
    flow over a wider region, thereby increasing the total relaxation resistance.

    In terms of the diagnostic functional $\mathcal{E}_\chi^{\mathrm{eff}}$, this
    manifests as an increase in the integrated contribution of spatial gradients
    and inhibited ordering across the configuration.
    The higher effective energy of more distant orbitals therefore reflects a greater
    \emph{global cost of constraint}, not a larger local potential or an intrinsic
    preference for spatial separation.

    Importantly, this energetic ordering is independent of the geometric visibility
    of orbitals discussed in Appendix~\ref{app:level_sets_orbitals}.
    While orbital-like shapes emerge from thresholding and projection effects applied
    to a continuous field, their energetic hierarchy is determined by the degree to
    which the corresponding configurations frustrate relaxation at the structural
    level.

  \paragraph{Curvature as constrained relaxation.}
    Within this context, the notion of ``curvature'' associated with $\chi$ must be
    interpreted with care.
    It does not refer to spacetime curvature as a fundamental geometric object, nor to
    an intrinsic curvature of a background manifold.
    Instead, it characterizes the internal deformation and non-uniformity of projected
    $\chi$ configurations and the way these deformations modulate the propagation of
    relaxation and correlations.

    Effective spacetime curvature arises only secondarily, as a macroscopic descriptor
    summarizing how constrained configurations influence admissible ordering and
    correlation structures over extended regions.
    No independent geometric degree of freedom is introduced at the fundamental level.

    Stable solitonic configurations arise when nonlinear self-interaction effects of
    the $\chi$ substrate balance the dispersive tendency associated with spatial
    gradients in the projected description.
    This balance allows localized resistance to relaxation to persist over extended
    ordering intervals, providing a dynamical and geometric origin for long-lived
    particle-like excitations without invoking fundamental energy conservation laws,
    quantized potentials, or primitive geometric dynamics.
