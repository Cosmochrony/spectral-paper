\subsection{Analytical Solutions of the \texorpdfstring{$\chi$}{χ}-Field Dynamics}
  \label{subsec:analytical-solutions}

  To illustrate the qualitative behavior of the $\chi$ field, we derive a set of
  explicit analytical solutions of the effective relaxation equation
  \begin{equation}
    \partial_t \chi
    =
    c \sqrt{1 - \frac{|\nabla \chi|^2}{c^2}},
    \label{eq:chi_effective_evolution}
  \end{equation}
  valid in regimes where a smooth geometric description is applicable.
  Here, $\partial_t$ denotes an effective ordering parameter associated with the
  relaxation process, not a fundamental time derivative.

  \noindent
  \emph{The solutions presented in this subsection do not describe trajectories or
  time evolution of a fundamental field, but characterize admissible configurations
  of the effective projected description once a hydrodynamic regime is assumed.}

  These solutions are not intended to exhaust the full dynamics of $\chi$, but to
  clarify its causal structure, limiting configurations, and relaxational character
  within the effective description.

  \subsubsection*{Homogeneous Relaxation Solution}

    In a spatially homogeneous configuration, spatial variations vanish,
    \[
      \nabla \chi = 0 ,
    \]
    and the effective evolution equation reduces to
    \begin{equation}
      \partial_t \chi = c .
    \end{equation}

    Integration yields
    \begin{equation}
      \chi(t) = \chi_0 + c\,t ,
    \end{equation}
    where $\chi_0$ is a constant labeling the homogeneous relaxation state.
    This solution represents a reference configuration corresponding to uniform
    global relaxation in the projected description.

    When interpreted within an effective spacetime description, this homogeneous
    relaxation underlies the emergence of cosmological expansion and leads
    naturally to a Hubble-like relation, as discussed in
    Section~\ref{subsec:homogeneous-cosmological-limit}.
    No fundamental notion of expansion is assumed at the level of the $\chi$
    substrate itself.

  \subsubsection*{Spherically Symmetric Gradient-Saturated Profiles}

    Consider a spherically symmetric configuration $\chi = \chi(r,t)$ within the
    effective description.
    The evolution equation becomes
    \begin{equation}
      \partial_t \chi
      =
      c \sqrt{1 - \frac{(\partial_r \chi)^2}{c^2}} .
    \end{equation}

    Configurations satisfying
    \[
      |\partial_r \chi| = c
    \]
    correspond to complete local saturation of the relaxation bound.
    In this limit,
    \[
      \partial_t \chi = 0 ,
    \]
    indicating a local freezing of the effective temporal ordering parameter.

    Such profiles take the form
    \begin{equation}
      \chi(r) = \chi_0 \pm c\,r ,
    \end{equation}
    and represent limiting admissible configurations in which relaxation is entirely
    inhibited by maximal structural gradients.

    Although these configurations cannot be realized globally in a regular manner,
    they play an important conceptual role as idealized models of horizons and
    maximally constrained regions within the effective description.
    In geometric language, they correspond to boundaries beyond which spacetime
    notions cease to be operationally meaningful, rather than to physical singularities
    of the underlying $\chi$ substrate.

  \subsubsection*{Linear Relaxation Fronts}

    A simple class of exact solutions is given by linear fronts of the form
    \begin{equation}
      \chi(x,t) = \chi_0 + c\,t \pm v\,x ,
    \end{equation}
    with $|v| < c$.
    For such configurations,
    \[
      |\nabla \chi| = |v| < c ,
    \]
    and the evolution equation~\eqref{eq:chi_effective_evolution} is satisfied
    identically.

    These solutions describe admissible relaxation fronts separating regions of
    different projected $\chi$ values.
    They do not correspond to propagating waves or oscillatory modes, but to kinematic
    boundaries determined by the maximal admissible relaxation rate.

    The parameter $v$ characterizes the spatial steepness of the front within the
    effective description rather than a signal propagation speed, which remains
    bounded by $c$.

  \subsubsection*{Absence of Linear Wave Solutions}

    A crucial structural feature of the $\chi$ relaxation dynamics is the absence of
    linear wave solutions.
    Small perturbations around homogeneous relaxation configurations do not propagate
    as oscillatory modes.

    As shown in Section~\ref{subsec:stability-analysis}, infinitesimal perturbations
    are marginal at linear order and are damped once nonlinear effects are taken
    into account.
    The effective dynamics is therefore purely relaxational.

    Apparent wave-like phenomena, such as gravitational or electromagnetic radiation,
    arise only at the effective level, through collective excitations associated with
    structured matter configurations.
    These emergent phenomena should not be confused with fundamental propagating modes
    of the $\chi$ substrate.

  \subsubsection*{Conclusion}

    These analytical solutions illustrate the central features of the effective
    $\chi$ description:
    homogeneous relaxation underpins cosmological expansion at the projected level,
    gradient saturation defines causal and horizon-like limits, and relaxation fronts
    clarify the kinematic structure imposed by the universal bound $c$.

    Together, they confirm the internal consistency of the hydrodynamic regime of
    Cosmochrony and prepare the ground for the emergence of effective geometric,
    gravitational, and radiative phenomena at macroscopic scales.
