\section{Relation to the Effective Geometric Description}
  \label{sec:relation-to-the-effective-geometric-description}

  The effective geometric structures introduced in the main text, such as metric
  fields, spatial gradients, connection-like objects, and Poisson-type equations,
  do not represent fundamental degrees of freedom in the present relational framework.
  They arise as coarse-grained summaries of relational configurations of the
  $\chi$ network, once a projectable regime becomes applicable.

  In the pre-geometric formulation, the relational substrate is defined purely in
  terms of adjacency, spectral properties, and admissibility constraints, without
  reference to coordinates, distances, or differential structures.
  Geometric notions become meaningful only after a stable effective field
  $\chi_{\mathrm{eff}}$ has been constructed (Appendix~\ref{app:chi-eff-derivation})
  and an operational distance $d^W$ emerges from relational stiffness
  (Appendix~\ref{app:relational-distance-minimal-path}).

  Within this regime, smooth variations in $\chi_{\mathrm{eff}}$ over neighborhoods
  defined by $d^W$ admit a continuum approximation.
  Metric components, gradients, and connection-like quantities were then introduced
  as \emph{descriptive tools} that compactly encode how admissible relational
  correlations respond to local perturbations.
  They summarized the collective response properties of the projected description rather
  than encoding independent dynamic degrees of freedom.

  Importantly, these geometric objects are valid only insofar as the projection
  remains locally injective and relational variations remain weak.
  When admissibility breaks down near the deprojection thresholds or in strongly
  constrained regions, the effective geometric description loses its operational meaning.
  In such regimes, no failure of geometric dynamics is implied; rather, the geometric
  language itself ceases to be applied.

  Therefore, the effective geometric description employed in the main text should be
  understood as a regime-dependent and operational representation of relational
  organization, exactly within its domain of validity and silent outside it.
