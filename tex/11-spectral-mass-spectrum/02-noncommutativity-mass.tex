\subsection{Non-Commutativity as a Source of Mass}
  \label{sec:noncommutativity-mass}

  In the Cosmochrony framework, torsion is promoted from a geometric feature to a
  \emph{dynamical constraint}.
  It does not merely reshape the spectrum of admissible modes, but actively
  competes with the transport of relaxation through the projection fiber.
  The key structural transition responsible for mass generation is the loss of
  commutativity between relaxation diffusion and torsional constraints.

  \subsubsection*{Inhibition of Relaxation}
    \label{sec:inhibition-relaxation}

    Let $\Delta^{(0)}_G$ denote the scalar (0-form) Laplacian induced by the relational
    graph structure, and let $\Omega_w$ be the effective torsion operator associated
    with winding number $w$ within the projection fiber $\Pi$.
    For the fundamental lepton configuration ($w=1$), the relaxation flow remains
    spectrally compatible with torsion, such that
    \begin{equation}
      [\Delta^{(0)}_G,\Omega_{1}] = 0 ,
    \end{equation}
    and relaxation modes can be chosen as simultaneous eigenstates.

    For higher-winding configurations ($w\ge 2$), torsional constraints become
    spectrally frustrated with respect to diffusion:
    \begin{equation}
      [\Delta^{(0)}_G,\Omega_{w}] \neq 0 \qquad (w\ge 2).
    \end{equation}
    This non-commutativity inhibits uniform relaxation across the fiber and induces
    an irreducible spectral compression.
    At the effective level, this inhibition manifests as an amplification of inertial
    mass, reflecting increased resistance to relaxation rather than additional
    substance.

    To quantify this effect without introducing adjustable parameters, we define the
    torsional action as a purely spectral invariant:
    \begin{equation}
      \mathcal{A}(w) \;\equiv\;
      \frac12 \ln\!\left(
                     \frac{\det(\Delta^{(0)}_G+\Omega_{w})}{\det(\Delta^{(0)}_G)}
      \right),
      \label{eq:Aw-fredholm}
    \end{equation}
    where the determinant is understood in the zeta-regularized (Fredholm) sense.

  \subsubsection*{The Pisano Ratio as a Stability Fixed Point}
    \label{sec:pisano-ratio}

    For $w=2$, the projection fiber ceases to be spectrally isotropic.
    The relaxation modes split into two dynamically competing sectors,
    \begin{equation}
      \Pi \;=\; \Pi_{\parallel} \oplus \Pi_{\perp},
    \end{equation}
    where $\Pi_{\parallel}$ aligns with the Hopf-like direction selected by torsion,
    and $\Pi_{\perp}$ spans the orthogonal frustrated modes.

    Dynamical stability requires simultaneously avoiding internal resonances and
    maximizing relaxation throughput.
    This condition selects the most irrational admissible ratio between the spectral
    frequencies of the two sectors, in direct analogy with KAM-type stability
    criteria:
    \begin{equation}
      \frac{\lambda_{\parallel}}{\lambda_{\perp}} \;=\; \varphi
      \qquad\Longrightarrow\qquad
      \beta \;\equiv\; \frac{1}{\varphi},
      \label{eq:beta-golden}
    \end{equation}
    where $\varphi=(1+\sqrt5)/2$ is the golden ratio.
    In this framework, $\beta$ is not a phenomenological fit parameter, but the unique
    compression invariant associated with non-integrable torsional frustration.

  \subsubsection*{Leptonic Spectrum Synthesis}
    \label{sec:leptonic-synthesis}

    In the geometric regime, rest masses scale with the effective spectral cut-off
    frequencies of admissible configurations.
    For the muon, non-commutative torsion yields the following parameter-free
    prediction:
    \begin{equation}
      \boxed{
        \frac{m_\mu}{m_e}
        = \sqrt{\frac{\lambda_{2}}{\lambda_{1}}}
        \cdot \frac{3}{2\alpha}
        \cdot \frac{1}{\varphi}
      }
      \qquad
      \text{with}
      \qquad
      \frac{\lambda_2}{\lambda_1} = \frac{8}{3}.
      \label{eq:muon-ratio-final}
    \end{equation}
    Here $\alpha$ is interpreted as the spectral transmittance of relaxation through
    the projected regime.

    The appearance of the fixed ratio $\lambda_2/\lambda_1 = 8/3$ is not assumed.
    Its robustness is demonstrated independently in
    Appendix~\ref{sec:spectral_ratio_derivation}, where the same invariant emerges
    both from stochastic relational sampling on the underlying graph and from the
    spectral response of a discrete Laplacian defined on the same relational support.
    This confirms that the ratio reflects an intrinsic property of the relational
    geometry, rather than a fitted spectral input.
