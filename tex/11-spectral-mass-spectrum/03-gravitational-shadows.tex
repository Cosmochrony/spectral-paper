\subsection{Gravitational Shadows and the Spectral Wake}
  \label{sec:gravitational-shadows}

  The torsional constraint $\Omega_w$ responsible for mass generation is not strictly
  confined to the localized projected configuration.
  While the obstruction itself remains topologically localized within the projection
  fiber, it induces an extended \emph{spectral deformation} of the surrounding
  relational substrate $\chi$.
  This deformation takes the form of a persistent modification of the local spectral
  density of relaxation modes, here referred to as a \emph{spectral wake} or
  gravitational shadow.

  The gravitational shadow does not correspond to additional matter or propagating
  degrees of freedom.
  Rather, it reflects a long-lived redistribution of relaxation capacity in the
  substrate induced by the presence of a topological obstruction.
  In effective spacetime descriptions, this manifests as an extended gravitational
  influence exceeding that associated with the localized baryonic projection alone.

  This mechanism provides an ontological basis for phenomena usually attributed to
  dark matter, without introducing new particles or fields.
  The spectral shadow exhibits two characteristic features:

  \begin{itemize}

    \item \textbf{Elastic Remanence:}
    The spectral deformation persists in the relational substrate even under
    displacement of the baryonic projected configuration.
    As a result, the effective gravitational potential need not remain spatially
    coincident with visible matter, naturally accounting for the observed offsets in
    systems such as the Bullet Cluster.
    This persistence reflects the finite relaxation time of large-scale spectral
    rearrangements rather than the motion of an independent mass component.

    \item \textbf{Non-Local Susceptibility:}
    Effective gravitational acceleration arises from gradients in the global
    relaxation flow.
    When the local gradient falls below a characteristic threshold
    $a_0 \sim c H_0$, the substrate response transitions from linear to non-linear.
    In this low-acceleration regime, the elastic properties of $\chi$ dominate,
    recovering MOND-like phenomenology as an emergent phase of spectral response,
    rather than as a modification of fundamental dynamics.

  \end{itemize}

  In this perspective, dark matter effects are reinterpreted as manifestations of
  persistent spectral memory in the relational substrate.
  Gravitation probes not only localized topological obstructions, but also the
  extended relaxation shadow they imprint on $\chi$.
