\subsection{Spectral Stability and the Unit of Mass}
  \label{sec:spectral-stability}

  In the Cosmochrony framework, rest mass does not arise from an intrinsic property
  of particles nor from a coupling to an external field.
  It is defined as a \emph{spectral invariant} associated with the stability of
  admissible projected configurations.

  More precisely, the rest mass $m$ of a localized excitation is identified with
  the fundamental eigenmode of the scalar Laplacian $\Delta^{(0)}_G$ acting on the
  projection fiber $\Pi$, subject to a set of topological constraints
  $\mathcal{T}$ characterizing the configuration:
  \begin{equation}
    m^2 c^2 \;=\; \lambda_{\mathcal{T}}
    \;\equiv\;
    \mathrm{Eig}\!\left(\Delta^{(0)}_G\right)\Big|_{\mathcal{T}} .
    \label{eq:mass_spectral_def}
  \end{equation}

  Here, $\lambda_{\mathcal{T}}$ is a dimensionless spectral eigenvalue encoding the
  minimal energetic cost required to maintain the configuration against the global
  relaxation of the substrate $\chi$.
  Mass therefore quantifies resistance to relaxation, rather than inertial response
  to force.

  The electron mass $m_e$ corresponds to the lowest non-trivial admissible eigenmode
  $\lambda_1$, associated with the simplest stable topological constraint compatible
  with projectability.
  It represents the fundamental resonance of the substrate $\chi$ within the
  finite-volume geometry $\Pi \cong S^3$.

  The conversion from dimensionless spectral values to physical mass units is fixed
  by identifying this fundamental mode with the observed electron mass:
  \begin{equation}
    m \;=\; m_e \sqrt{\frac{\lambda_{\mathcal{T}}}{\lambda_1}} .
  \end{equation}
  In this sense, the electron mass does not enter the theory as an arbitrary parameter,
  but as the natural unit of mass emerging from the lowest stable spectral excitation
  of the projection fiber.
