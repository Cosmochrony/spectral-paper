\clearpage

\section{Spectral Distance and Embedding}
  \label{sec:spectral-distance-and-embedding}

  A central step in the reconstruction of geometry from relational data is the
  definition of a notion of distance that does not rely on a pre-existing metric or
  coordinate structure.
  In this section, we introduce spectral notions of proximity and distance derived
  directly from relational connectivity and Laplacian operators.

  We consider a relational system represented by a graph or coarse-grained network
  endowed with a Laplacian operator $\Delta$.
  No embedding space or background geometry is assumed.
  The spectrum and eigenfunctions of $\Delta$ encode the connectivity structure of
  the system and constitute the only input for the constructions below.
  The relation between Laplacian spectra and global connectivity properties
  of graphs is a central result of spectral graph theory~\cite{Chung1997}.

  \subsection{Spectral proximity}

    Given the Laplacian spectrum $\{\lambda_n, \phi_n\}$, one may define spectral kernels
    that quantify relational proximity between nodes or abstract elements of the
    relational system.
    A generic example is provided by heat-kernel–type constructions,
    \begin{equation}
      K(i,j;\alpha) \;=\; \sum_{n} e^{-\alpha \lambda_n}\,\phi_n(i)\,\phi_n(j),
    \end{equation}
    where $\alpha$ is a spectral scale parameter.
    Such kernels measure the degree of connectivity between elements $i$ and $j$ through
    the spectrum of $\Delta$, without reference to any metric distance.

    Spectral kernels of this type are widely used in clustering and manifold
    learning to extract geometric information from graph spectra~\cite{VonLuxburg2007}.
    They are invariant under relabeling of nodes and do not presuppose any embedding.
    They provide a natural notion of relational proximity that depends solely on the
    spectral properties of the Laplacian.

  \subsection{Spectral distance}

    From spectral proximity measures, one may define an effective distance by monotonic
    transformations.
    A convenient choice is
    \begin{equation}
      d_{\mathrm{spec}}(i,j)
      \;=\;
      -\log\!\left(
               \frac{K(i,j;\alpha)}{\sqrt{K(i,i;\alpha)\,K(j,j;\alpha)}}
      \right),
    \end{equation}
    which defines a symmetric, non-negative quantity vanishing when $i=j$.
    This definition is purely spectral and does not invoke any geometric or physical
    interpretation.

    The resulting distance is generically non-injective:
    distinct relational configurations may induce identical spectral distances, and
    multiple embeddings may correspond to the same distance matrix.
    This non-uniqueness is a structural feature of spectral reconstruction rather than a
    limitation of the formalism.

    Explicit constructions of local embeddings, intrinsic dimension selection,
    and the breakdown of manifold reconstruction outside the projectable regime
    are presented in Appendix~\ref{subsec:emergent-coordinates}.

    \paragraph{Emergent spectral dimension.}
      In Eq.~(2), the exponent $d$ should not be interpreted as a fixed or postulated
      topological dimension.
      Within the present framework, $d$ is an \emph{emergent spectral quantity}
      characterizing the asymptotic scaling of the eigenvalue counting function
      $N(\lambda)$ in the low-energy regime.
      Operationally, $d$ is extracted from the slope of $\log N(\lambda)$ versus
      $\log \lambda$ and may take non-integer values at finite spectral resolution.

      In particular, the convergence of $d$ toward an integer value reflects the
      stabilization of the relational substrate into a smooth effective geometric
      regime.
      Numerical evidence presented in this work indicates that, under admissibility
      and regularity conditions, the substrate converges toward a four-dimensional
      spectral behavior, $d \to 4$, without this value being imposed \emph{a priori}.
      This convergence is observed to persist across multiple spectral decades, indicating
      that the four-dimensional behavior is not a transient finite-size effect but a stable
      property of the admissible relational regime.

      At smaller spectral scales, deviations from integer dimensionality encode
      local curvature and connectivity distortions, providing a direct bridge
      between spectral observables and effective geometric structure.

    \paragraph{Stability of the spectral dimension.}
      Numerical evaluations of the spectral counting function, performed on distinct
      relational realizations and resolutions of the substrate, indicate that the extracted
      spectral dimension $d_s$ remains stable and converges toward $d_s \simeq 4$ over a broad
      range of spectral scales.
      Deviations appear only near the ultraviolet cutoff, where the notion of effective
      geometry itself ceases to be applicable.

      This stability across scales and discretizations supports the interpretation of
      $d_s$ as an emergent property of the relational organization rather than as an imposed
      dimensional parameter.

    \paragraph{From graph distance to effective geodesics.}
      The operational distance defined via shortest weighted paths on the relational
      graph plays the role of an effective geodesic distance.
      While combinatorial distances count edge hops, the weighted distance incorporates
      local variations of connectivity through edge weights, thereby encoding the
      inhomogeneous relational structure of the substrate.

      In the continuum limit, the local density of nodes contributing to admissible
      paths controls the scaling of volumes and distances.
      This node density acts as the discrete analogue of the metric determinant
      $\sqrt{-g}$, governing how relational neighborhoods are mapped onto effective
      geometric volumes.
      Geometric notions thus arise from connectivity statistics rather than from
      a postulated metric field.

\subsection{Local embedding and quadratic approximation}

  When the relational system admits a sufficiently regular spectral structure, the
  spectral distance matrix may be locally approximated by a low-dimensional embedding.
  In such regimes, one may introduce local coordinates $x^\mu$ as auxiliary variables
  parametrizing the embedding space.

  To leading order, the spectral distance between nearby elements admits a quadratic
  approximation of the form
  \begin{equation}
    d_{\mathrm{spec}}(i,j)^2
    \;\approx\;
    g_{\mu\nu}(x)\,\Delta x^\mu \Delta x^\nu ,
  \end{equation}
  where $g_{\mu\nu}$ is a symmetric tensor encoding the local geometry of the embedding.
  This tensor is not postulated as a fundamental object, but arises as a local summary
  of spectral distances.

  Local quadratic approximations of distance functions and their relation
  to effective metric tensors are standard results in metric geometry~\cite{Burago2001}.

  The introduction of $g_{\mu\nu}$ should be understood as a descriptive convenience,
  valid only in regimes where a smooth spectral embedding exists.
  Outside such regimes, no metric interpretation is assumed or required.

\subsection{Scope and limitations}

  The constructions presented in this section are entirely kinematical.
  They do not rely on assumptions about dynamics, temporal ordering, or causal
  structure.
  Their purpose is solely to establish how effective geometric notions may be
  reconstructed from spectral data associated with relational connectivity.

  The emergence of continuum geometry, curvature, and comparison with known geometric
  solutions are addressed in the following sections.
