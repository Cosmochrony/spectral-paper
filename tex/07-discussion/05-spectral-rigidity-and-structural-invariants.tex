\subsection{Spectral rigidity and structural invariants.}
  \label{subsec:spectral-rigidity-and-structural-invariants.}
  While the reconstruction of the effective geometry from relational spectral data is
  intrinsically non-unique and regime-dependent, this non-uniqueness does not extend
  to all spectral observables.
  The relational substrate exhibits robust spectral invariants that persist across
  discretizations, graph realizations, and numerical schemes.

  A notable example is the emergence of the universal ratio
  $\lambda_2/\lambda_1 = 8/3$ in the scalar Laplacian spectrum, as illustrated in
  Appendix~\ref{subsec:emergent-coordinates}.
  Although many relational microstates may project indistinguishable effective
  metrics, only a restricted class of spectral organizations is compatible with a
  stable projectable geometric regime.

  Therefore, such invariants constrain the space of admissible emergent geometries
  more strongly than metric reconstruction alone.
  They provide an intermediate level of structure between microscopic relational
  descriptions and macroscopic geometric observables, and may serve as distinguishing
  signatures with respect to other background-independent approaches that reproduce
  similar effective geometries without exhibiting comparable spectral constraints.

  In this sense, the existence of robust spectral invariants such as the
  $\lambda_2/\lambda_1 = 8/3$ ratio suggests that relational substrates may be
  more tightly constrained than effective metric reconstructions alone.
