\subsection{Non-uniqueness and regime dependence.}
  \label{subsec:non-uniqueness-and-regime-dependence.}
  An important implication of the spectral approach is the intrinsic non-uniqueness of geometric reconstruction.
  Different relational configurations may give rise to indistinguishable effective
  metrics within a given regime, reflecting the non-injective character of the projection.
  Therefore, geometric descriptions are intrinsically approximate and regime-dependent.
  This non-uniqueness should not be interpreted as an ambiguity of the theory but as a
  structural feature of the emergent geometry itself.
